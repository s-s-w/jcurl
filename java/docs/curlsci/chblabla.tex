
\chapter*{Preface}\addcontentsline{toc}{chapter}{Preface}

During my days as an active Curler I spent quite a time thinking about
what's happening when a rock slides over the ice or hits another one.
And the longer I thought and the more people I asked the mystery did
nothing but grow.

Some situations I could predict by experience rather fair. Some more
skilled fellows could others. But some outcomes were completely
miraculous.

So I started to gather scientific material to cover the topic
from a theoretical base. Because once you got a model working properly
for known situations, you can start to examine unknown ones and compare
the model's predictions with reality. This way you get a fundamental
understanding of what's going on.

At the beginning I considered this to be a kind of brain-jogging, but now I
think the sport has developed so far yet that a team can't seriously
practice competitive curling without at least noticing some theoretical
knowledge e.g.\ about psychology but also physics.

Because to throw an \emph{impossible} matchwinner you first have to
recognise the option, then you need to dare it and surely, you need skill
\&\ luck to succeed.

\chapter*{Introduction}\addcontentsline{toc}{chapter}{Introduction}

This paper wants to examine the theoretical basics of running rocks and hits at
first and later apply these laws to real curling situations and get some useful
hints \&\ clues.
