
\section{With spin and loss (Try I)\label{lossII}}

The approach \ref{lossI} doesn't give proper results, so we try a new method.
We assume the rock B not to experience acceleration until the initial friction
$ F_H $ is overcome. This causes both, a loss of energy as well as as loss of
momentum and spin. Besides we get a natural splitting of the loss of energy to
the components $ \vY{}, \vX{} $ and $ \w{} $. 

For this purpose we need to know the forces \emph{during} the time of contact $
T $. To achieve this we need to make some assumptions about the rocks'
elasticity mechanism. Here we assume the rocks to behave like springs and
fulfilling Hook's law of elasticity. Also we assume the friction rock/rock to
be constant $ |\FX{}| = \mu \cdot |\FY{} | $. This way we get to know the
forces and everything's just fine.

In this section all letters (if not stated differently) mean rock A before the
hit. E.g. $ x $ really is $ \bef{x^\mathrm{a}} $.

\subsection{Contact behaviour}

Hook says:
{%\samepage
\begin{eqnarray}
F &=& H \cdot x \\
E_\mathrm{Pot} &=& \frac{1}{2} H \cdot x^2
\end{eqnarray}}

{%\samepage
\begin{eqnarray}
H \cdot x &=& m \cdot a = m \frac{ \diff{}^2 x }{ \dt } \\
\Longrightarrow\quad
x(t) &=& A \cdot \sin \bar\omega t \\
\mbox{with}\quad
    \bar\omega &=& \sqrt{ \frac{H}{m} } = \frac{ \pi }{ T }  = \const !\\
\mbox{and}\quad
    A &=& \sqrt{ \frac{m}{H} \vY{}{}^2 } = \frac{ -\vY{} }{ \bar\omega } =
    \frac{ -\vY{} T }{ \pi } 
\end{eqnarray}}
%
{%\samepage
\begin{eqnarray}
 x(t) &=& - \frac{ \vY{} }{ \bar\omega } \cdot \sin \bar\omega t \\
\ddt x(t) &=& -\vY{} \cdot \cos \bar\omega t \\
\frac{ \diff{}^2 }{ \dt^2 } x(t) &=& \vY{} \bar\omega \sin \bar\omega t 
\end{eqnarray}}
%
{%\samepage
\begin{eqnarray}
\DelTa{1,2} \vY{} &=& -\vY{} \left[ \cos\bar\omega t' \right]_0^t = 
   -\vY{} \left( \cos\bar\omega t - 1 \right) 
\end{eqnarray}}


O.k.\ so far. But what happens during such a hit? We split the whole contact
$ 0 \le t \le T $ into three periods:
\begin{enumerate}
\item Overcome the initial friction $ t < t_1 $
\item Level the surface speeds $ t < t_0 $
\item Exchange the rest of the parallel momentum $ t \le T $
\end{enumerate}

\begin{figure}[htb]
{\footnotesize \hfill\input{contacta.emt}\hfill\input{contactb.emt}\hfill }
\caption[$ \FY{}(t) $ \emph{during} the hit]{$ \FY{}(t) $. The parallel forces
    during the hit. As you see, B ain't accelerated until $ t \ge t_1
    $.\figlab{forceABhit}} 
\end{figure}

\subsection{Find $ t_0 $}

$ t_0 $ is the time, when A's effective surface speed becomes 0 (=B's).
{%\samepage
\begin{eqnarray}
    \aft{\v{\eff}} & \stackrel{!}{=} & 0 \\
    \aft{\vX{}} - \Rad{}\aft{\w{}} &=& 0 \\
    \vX{} + \DelTa{1,2}\vX{} - \Rad{}( \w{} + \DelTa{1,2}\w{} ) &=& 0 \\
\end{eqnarray}}
with \eqref{AA} \& \eqref{CC}:
{%\samepage
\begin{eqnarray}
    \DelTa{1,2}\w{} &=& \DelTa{1,2}\vX{} \frac{ - \m{}\Rad{} }{ \J{} } \\
    \DelTa{1,2}\vX{} &=& \frac{ - \v{\eff} }{ \frac{ \J{} + \m{}\Rad{}^2 }{\J{}}} 
\end{eqnarray}}
%
{%\samepage
\begin{eqnarray}
\FX{} &=& -\sgn(\v{\eff}) \cdot \mu \cdot | \FY{} | \\
\Longrightarrow\quad
\DelTa{1,2}\vX{} &=& -\sgn(\v{\eff}) \cdot \mu \cdot \left| -\vY{} \cdot(\cos\bar\omega t_0 - 1) \right| 
\end{eqnarray}}
%
{%\samepage
\begin{eqnarray}
\Longrightarrow\quad
\frac{ - \v{\eff} \frac{ 1 }{ \frac{ \J{} + \m{}\Rad{}^2 }{ \J{} } } }{
- \sgn(\v{\eff}) \mu | \vY{} | } &=&
| \underbrace{\cos\bar\omega t_0 - 1}_{ \le 0 \forall t_0 } | \\
- \left| \frac{ \frac{ \v{\eff} }{ \frac{ \J{} + \m{}\Rad{}^2 }{ \J{} } } }{
\vY{} \mu } \right| + 1 &=& \cos\bar\omega t_0
\end{eqnarray}}

\subsection{Find $ t_1 $}

$ t_1 $ is the time, when the Hook-force equals the friction.
{%\samepage
\begin{eqnarray}
\FY{}{}^2 + \FX{}{}^2 &=& F_H^2 \\
\FY{}{}^2 + \mu^2\FY{}{}^2 &=& F_H^2 \\
\FY{} &=& \frac{ F_H }{ \sqrt{ 1+\mu^2 } } \\
\m{} \vY{} \bar\omega \sin\bar\omega t_1 &=& \frac{ F_H }{ \sqrt{1+\mu^2 } } \\
\sin\bar\omega t_1 &=& \frac{ F_H }{ \m{} \vY{} \bar\omega \sqrt{1+\mu^2} }
\end{eqnarray}}

\subsection*{Result}

The lost momentum is:
{%\samepage
\begin{eqnarray}
\Delta\v{} &=& -\vY{} \left( \sqrt{1-\sin^2\bar\omega t_1} - 1 \right)
{-\sgn(\v{\eff})\cdot\mu \choose -1}
\end{eqnarray}}

If $ \v{\eff} $ becomes zero \emph{before} the friction ends we need to find 
the friction parallel to $ \eY{} $ only, but include the lost momentum
in $ \eX{} $ direction as well.

{%\samepage
\begin{eqnarray}
\mbox{if}\quad  t_0 < t_1 & \Longleftrightarrow & \sin\bar\omega t_0 < \sin\bar\omega t_1 \\
\sin\bar\omega t_1 &=& \frac{ F_H }{ \m{} \vY{} \bar\omega \cdot 1 } 
\end{eqnarray}}
%
Lost momentum:
{%\samepage
\begin{eqnarray}
\perp &:& - \sgn(\v{\eff}) \mu \cdot \left| -\vY{} \cdot (\cos\bar\omega t_0 - 1) \right| \\
\parallel &:& \vY{} \cdot (\sqrt{1-\sin^2\bar\omega t_1} - 1)
\end{eqnarray}}

After applying the above loss of momentum to the \emph{hitting} rock and a 
loss of spin of
{%\samepage
\begin{eqnarray}
\DelTa{1,2}\w{} &=& - \DelTa{1,2}\vX{} \frac{ \m{}\Rad{} }{ \J{} }
\end{eqnarray}}
we can continue with a normal hit, e.g.\ like described in \ref{spini}.

\subsection{Loss of energy \& $ F_H $}

The energy stored in a spring is
{%\samepage
\begin{eqnarray}
E &=& \frac{1}{2} H x^2 \\
&=& \frac{1}{2} \bar\omega^2 \cdot \m{} \cdot x^2 \\
\mbox{with}\quad x &=& - \vY{}/\bar\omega \cdot \sin\bar\omega t_H \\
\mbox{and}\quad F_H &=& \m{}\vY{}\bar\omega \cdot \sin\bar\omega t_H \\
E &=& \frac{1}{2} \bar\omega^2 \cdot \m{} \cdot (- \vY{}/\bar\omega\cdot\sin\bar\omega t_H )^2 \\
E &=& \frac{1}{2} \frac{F_H^2}{\m{}\bar\omega^2} \\
\Longrightarrow\quad
F_H &=& \bar\omega\sqrt{2\m{}E} = \const !
\end{eqnarray}}


That tells us for a given loss of energy we'll get a constant initial friction.
(Which, indeed, is a very nice result.)

\subsection{Resumee}

This model depends on Coulomb-friction rock/rock and a constant friction
rock/ice for non-moving rocks. This friction can be very big, about $ 10^4 $ N!

The results presented by this model seem quite sensible. The only problem
remaining is what happens if hitting a freeze? Do we have to overcome the
friction two times? If yes this model works fine.

