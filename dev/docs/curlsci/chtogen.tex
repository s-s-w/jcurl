
\chapter{Rock collission}

Yet many brains boiled, there's poor secure knowledge still. Which directions
take the involved rocks? How much momentum (or energy) is lost? How is the spin
transferred? 

Maybe this paper can show some ideas and suggest solutions. 

\vspace{1em}
Some things are going to be pretty much same with most of the following models.
They are defined here and will be just referred in the latter. The variables'
and constants' naming also is defined here.

\begin{table}[htb]
\begin{center}
\begin{tabular}{|ll|}
\hline
\Tk         & Time of contact                                              \\
\m{}        & Mass                                                         \\
\v{}        & Speed                                                        \\
\p{}        & Momentum $ \p{} = \m{} \cdot \v{} $                          \\
\F{}        & Force  $ \F{} = \m{} \cdot \ddt \v{} = \ddt \p{} $           \\
\J{}        & Moment of inertia                                            \\
\w{}        & Angular speed, spin                                          \\
\vL{}       & Angular momentum $ \vL{} = \J{} \cdot \vw{} $                \\
\vM{}       & Torque $ \vM{} = \J{} \cdot \ddt \vw{} = \ddt \vL{} = \vec{R} \times \F{} $ \\
\eY{}       & unit vector in (parallel) direction of hit (see \figref{coordsys}) \\
\eX{}       & unit vector perpendicular to direction of hit (see \figref{coordsys}) \\
\FY{}       & $ \FY{} = \F{} \cdot \eY{} $ and $ \F{} = \FX{} + \FY{} $    \\
\FX{}       & $ \FX{} = \F{} \cdot \eX{} $ and $ \F{} = \FX{} + \FY{} $    \\
\v{a}       & $\v{}$ of rock A                                             \\
\v{b}       & $\v{}$ of rock B                                             \\
\bef{\v{}}  & $\v{}$ before hit                                            \\
\aft{\v{}}  & $\v{}$ after hit                                             \\
$\DelTa{1,2} x $ & $ \aft{x} - \bef{x} $ (difference in time)              \\
$\DELTA{a,b}x$& $ x^\mathrm{b} - x^\mathrm{a} $ (difference in space)      \\
\hline
\end{tabular}
\caption[Naming of Variables and Constants]{%
         \tablab{names}%
         Naming of variables (italic) and constants (roman)}
\end{center}
\end{table}

\section*{The setup}

Rock A hits rock B (\figref{setup}), which equals A in all phys.\ properties.
Both can be in motion, but usually only A is.

According to the geometric setup we change to a local coordinate-system with
the $y$-axis pointing from A's center to B's. The $x$-axis is set perpendicular
to get a right-handed system. See \figref{coordsys}.

\begin{figure}[htb]
\setlength{\unitlength}{15mm} % the rock's radius
\begin{center}\large
\setcoordinatesystem units <\unitlength,\unitlength>
\mbox{\beginpicture
    % \setplotarea x from -1.3 to 2, y from -3 to 3.3
    %%% Stein A:
    \iffinalplot%
    \circulararc 360 degrees from 0.447 -0.894 center at  0    0
    \fi%
    \put{{\huge A}}                     [cc]  at 0.9 0.9
    %%% Stein B:
    \iffinalplot%
    \circulararc 360 degrees from 0.447 -0.894 center at  0.894 -1.788
    \fi%
    \put{{\huge B}}                           at  2      -1
    %% the speeds
    \put {\vector(0,-1){3.3}}       [Bl]      at  0      3.3
    \put {\bef{\v{a}}}              [l]       at  0.1    2.0
    \put {\vector(-2,-1){1.3}}      [Bl]      at  0      0
    \put {\aft{\v{a}}}              [r]       at -1.4   -0.6
    \put {\vector(1,-2){1.3}}       [Bl]      at  0.894 -1.788
    \put {\aft{\v{b}}}              [l]       at  2.0   -3.8

    \iffinalplot
    \circulararc -30 degrees from 0.5 -0.4 center at  0 0
    \fi
    \put {\vector(1,1){0.1}}        [Bl]      at 0.5 -0.4
    \put {\w{a}}                    [B]       at 0.45 -0.3
    \iffinalplot
    \circulararc 30 degrees from 0.394 -1.388 center at  0.894 -1.788
    \fi
    \put {\vector(1,1){0.1}}        [Bl]      at 0.394 -1.388
    \put {$-\w{b}$}                 [Bl]      at 0.394 -1.288
\endpicture}
\end{center}
\caption[The Setup]{The setup: Rock A hits B.
                        Both are \emph{identical}
                        in mass, size etc. After the hit i.g.\
                        both move.
                        \figlab{setup}}
\end{figure}

\begin{figure}[htb]
\setlength{\unitlength}{15mm} % the rock's radius
\begin{center}\large
\setcoordinatesystem units <\unitlength,\unitlength>
\mbox{\beginpicture
    \setplotarea x from -1.6 to 1.9, y from -2.8 to 1.5
    \iffinalplot%
    \circulararc 360 degrees from 0.447 -0.894 center at  0    0
    \fi%
    \iffinalplot%
    \circulararc 360 degrees from 0.447 -0.894 center at  0.894 -1.788
    \fi%
    \put{{\huge A}}                 [cc]   at  0.9    0.9
    \put{{\huge B}}                        at  2.0   -1.0
%    \put {\vector(0,-1){1.5}}       [Bl]   at  0      1.5
    \put {\vector(-2,-1){1.4}}      [Bl]   at  0.0    0.0
    \put {$\eX{}$}                  [r]    at -1.5   -0.6
    \put {\vector(1,-2){0.7}}       [Bl]   at  0.0    0.0
    \put {$\eY{}$}                  [Tl]   at  0.8   -1.4
\endpicture}
\end{center}
\caption[The Coordinate-System]{\figlab{coordsys}%
    The coordinate-system. Positive turn is conterclockwise.}
\end{figure}


\section{Without spin, without loss of energy}

This first model is the most primitive and simple one. But it's a proper
approximation in most cases. Also it's a good basis for later refinement.


\subsection{General thoughts}

Energy must be conserved\index{energy!convervation}:
%
\begin{eqnarray}
\bef{E} &=& \aft{E} \\
E &:=& \sum_\mathrm{all~rocks} E^i_{\mathrm{Kin}} +
                               E^i_{\mathrm{Rot}} +
                               E^i_{\mathrm{Fric}} \\
\mbox{here:} \quad 0 &=& E_{\mathrm{Rot}} = E_{\mathrm{Fric}}
\end{eqnarray}

\subsection{Parallel component $\pY{}$\label{para}}

If neglecting friction rock/ice\index{friction!rock/ice} and rock/rock only two
forces appear:
%
\begin{eqnarray}
\FY{a}(t) &=& - \FY{b}(t) \\
\Int_T\FY{a}(t)\dt &=& - \Int_T\FY{b}(t)\dt \\
\DelTa{1,2}\pY{a}  &=& - \DelTa{1,2}\pY{b}
\end{eqnarray}

With conservation of energy and assumption of equal masses we get:
\begin{eqnarray}
\aft{\vY{a}} &=& \bef{\vY{b}} \\
\aft{\vY{b}} &=& \bef{\vY{a}}
\end{eqnarray}

I.e.\ the speed-components along the hitting direction just exchange.

