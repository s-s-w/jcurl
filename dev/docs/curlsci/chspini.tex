
\section{With spin, without loss (Try I)\label{spini}}

The parallel component works like stated in \ref{para}. Much more interesting
turns out to be the

\subsection{Perpendicular component $\pX{}$ and spin $\w{}$}

According to the friction rock/rock\index{friction!rock/rock} in the touching
spot a tangential force appears.

\subsubsection{Getting the equations}

In the touching spot appear forces affecting A and B, speed and spin.
The relations are (with \figref{kraftzerleg}):

\begin{eqnarray}
\FX{b} &=& -\FX{a} \quad\mbox{(actio=reactio)} \\
\M{a}  &=&  \Rad{a}\cdot\eY{} \times \F{a} =
          - \FX{a} \cdot \Rad{a} \\
\M{b}  &=& -\Rad{b}\cdot\eY{} \times \F{b} = 
	    \Rad{b}\cdot\eY{} \times \F{a} = 
          - \FX{a} \cdot \Rad{b}
\end{eqnarray}

Integration of the left sides:
\begin{eqnarray}
\FX{a}(t) &\rightarrow& \Int_\Tk \FX{a}(t)\dt = \DelTa{1,2}\pX{a} = \m{a}\DelTa{1,2}\vX{a} \\
\FX{b}(t) &\rightarrow& \Int_\Tk \FX{b}(t)\dt = \DelTa{1,2}\pX{b} = \m{b}\DelTa{1,2}\vX{b} \\
\M{a}(t)  &\rightarrow& \Int_\Tk \M{a}(t)\dt  = \DelTa{1,2}\L{a}  = \J{a}\DelTa{1,2}\w{a}  \\
\M{b}(t)  &\rightarrow& \Int_\Tk \M{b}(t)\dt  = \DelTa{1,2}\L{b}  = \J{a}\DelTa{1,2}\w{b}
\end{eqnarray}

Giving:
\begin{eqnarray}
\eqlab{AA} \DelTa{1,2}\vX{a} &=& + \frac{1}{\m{a}}           \Int_\Tk\FX{a}(t) \dt \\
\eqlab{BB} \DelTa{1,2}\vX{b} &=& - \frac{1}{\m{b}}           \Int_\Tk\FX{a}(t) \dt \\
\eqlab{CC} \DelTa{1,2}\w{a}  &=& - \frac{ \Rad{a} }{ \J{a} } \Int_\Tk\FX{a}(t) \dt \\
\eqlab{DD} \DelTa{1,2}\w{b}  &=& - \frac{ \Rad{b} }{ \J{b} } \Int_\Tk\FX{a}(t) \dt
\end{eqnarray}

So much about how the transferred forces interact. Now let's say something
about it's size. Because it's a friction, it exists only as long, as there
is a difference of the rocks' surface-speeds. A point on a rotating circle's
edge has the speed\footnote{speed (along the path) + rotation}:
%
\begin{eqnarray}
    \v{\eff} &=& \v{} + \vec{\omega} \times \vec{R}(\alpha)
\end{eqnarray}
%
Or in our case:
%
\begin{eqnarray}
  {\vX{a}}^\eff &=& \vX{a} + \w{a} \times \Rad{a}\eY{} =
                    \vX{a} - \Rad{a}\w{a} \eqlab{EEa} \\
  {\vX{b}}^\eff &=& \vX{b} + \w{b} \times -\Rad{b}\eY{} =
                    \vX{a} + \Rad{b}\w{b} \eqlab{FFa}
\end{eqnarray}
and
\begin{eqnarray}
\eqlab{GG}
  \DELTA{a,b}\vX{\eff} &=& {\vX{b}}^\eff - {\vX{a}}^\eff
\end{eqnarray}

The friction stops when\index{friction!rock/rock!max}:
\begin{eqnarray}
  \aft{\DELTA{a,b}\vX{\eff}} &=& 0 \\
  \aft{\DELTA{a,b}\vX{\eff}} - \bef{\DELTA{a,b}\vX{\eff}} &=&
  - \bef{\DELTA{a,b}\vX{\eff}} \\
\eqlab{HH}
 \DELTA{a,b}\DelTa{1,2}\vX{\eff} &=& \underbrace{%
                                      - \bef{\DELTA{a,b}\vX{\eff}}
                                      }_{ \displaystyle =: \Veff } \\
\eqlab{EE}
\Veff =
 \DELTA{a,b}\DelTa{1,2}\vX{\eff} &=&
      ( \DelTa{1,2}\vX{b} + \Rad{a}\DelTa{1,2}\w{b} ) -
        \DelTa{1,2}\vX{a} + \Rad{a}\DelTa{1,2}\w{a}
\end{eqnarray}

\eqref{AA} -- \eqref{DD} and \eqref{EE} build a solvable set of equations.

But the extremal $ \DELTA{a,b}\DelTa{1,2}\vX{\eff} $ isn't reached always.
Both rocks might already separate while still rubbing. The maximum is limited
by the pressure-force or $ \DelTa{1,2}\pY{a} $:\footnote{%
%
This linear relation between \DelTa{1,2}\pY{} and the left side is only valid for
pure Coulomb-friction\index{friction!Coulomb-}:
%
\begin{eqnarray}
\FX{}(t) &=& -\sgn( v ) \cdot \mu \cdot |\FY{}(t)| \\
\Int_\Tk \FX{}(t) \dt &=& -\sgn( v ) \cdot \mu \cdot \Int_\Tk |\FY{}(t)| \dt
\end{eqnarray}
%
no altering sign of $ \FY{}(t) $:
%
\begin{eqnarray}
\Int_\Tk \FX{}(t) \dt &=& -\sgn( v ) \cdot \mu \cdot \left| \Int_\Tk \FY{}(t) \dt \right| \\
\Int_\Tk \FX{}(t) \dt &=& -\sgn( v ) \cdot \mu \cdot | \DelTa{1,2}\pY{}|
\end{eqnarray}
}

\begin{eqnarray}
\eqlab{FF}
   \left|\Int_\Tk\FX{a}(t)\dt \right| &\le& \mu \cdot | \DelTa{1,2}\pY{a} | \\
\textrm{otherwise}\quad
   \Int_\Tk\FX{a}(t)\dt &=& - \sgn (\Veff) \cdot \mu \cdot | \DelTa{1,2}\pY{a} |
\end{eqnarray}

\subsubsection{Solution}

\eqref{EEa}, \eqref{FFa} \& \eqref{GG}:
\begin{eqnarray}
\Veff &=& - \bef{\vX{b}} - \Rad{b}\bef{\w{b}} +
          ( \bef{\vX{a}} - \Rad{a}\bef{\w{a}} )
\end{eqnarray}
%
\eqref{EE}:
\begin{eqnarray}
\Veff &=& - \DelTa{1,2}\vX{a} + \Rad{a}\DelTa{1,2}\w{a}
          + \DelTa{1,2}\vX{b} + \Rad{a}\DelTa{1,2}\w{b}
\end{eqnarray}
%
Substituting \eqref{AA} through \eqref{DD}:
%
\begin{eqnarray}
\Veff &=& - \Int_\Tk \FX{a}(t) \cdot
          \left(
            \frac{1}{\m{a}} + \frac{ \Rad{a}^2 }{ \J{a} } +
            \frac{1}{\m{b}} + \frac{ \Rad{b}^2 }{ \J{b} }
          \right)
          \dt \\
\eqlab{Xdef}
\underbrace{\Int_\Tk \FX{a}(t) \dt}_{\displaystyle =: X} &=& \frac{ -\Veff }{
            \frac{1}{\m{a}} + \frac{ \Rad{a}^2 }{ \J{a} } +
            \frac{1}{\m{b}} + \frac{ \Rad{b}^2 }{ \J{b} }
          } \\
  \textrm{If}\quad | X | &>& \mu | \DelTa{1,2}\pY{a} | \\
  \textrm{then}\quad  X &=& - \sgn( \Veff ) \cdot \mu \cdot | \DelTa{1,2}\pY{a} |\\
%
  \DelTa{1,2}\vX{a} &=& + \frac{1}{\m{a}}           \cdot X \\
  \DelTa{1,2}\vX{b} &=& - \frac{1}{\m{b}}           \cdot X \\
  \DelTa{1,2}\w{a}  &=& - \frac{ \Rad{a} }{ \J{a} } \cdot X \\
  \DelTa{1,2}\w{b}  &=& - \frac{ \Rad{b} }{ \J{b} } \cdot X
\end{eqnarray}

\subsection{Check it out}

O.k.\ so far, but what about the conservation of energy? This still should be
ensured, so let's test it.

The energy consists of two parts, movement and spin and the motion part can
be split into it's two components.

Beyond that, by using the form $z_2 = z_1 + \DelTa{1,2}z $ for all speeds and
taking their squares we can collect the mixed parts into a remainder $\xi$
%
\begin{eqnarray}
E_{Kin_1} + E_{Rot_1} & \stackrel{!}{=} & E_{Kin_2} + E_{Rot_2} \\
&=& E_{Kin_1} + E_{Rot_1} + \xi(x) 
\end{eqnarray}
%
This remainder $ \xi(x) $ must, for energy might be lost, but not gained,
be less or equal 0.

To test this, we discuss the function $ \xi(x) $.
%
\begin{eqnarray}
\eqlab{xa}
\xi(x) &=& 2 x (\vX{a} - \vX{b} - \Rad{}\w{a} - \Rad{}\w{b} ) +
           2 x^2 \frac{\J{}+\m{}\Rad{}^2}{\m{}\J{}} \\
\eqlab{xb}
\xi''(x) &=& 4 \frac{\J{}+\m{}\Rad{}^2}{\m{}\J{}} \\
\eqlab{xc}
\xi(0) &=& 0
\end{eqnarray}

The symbols $ \vX{a} $ through $ \w{b} $, strictly spoken, should be 
$ \bef{\vX{a}} \dots \bef{\w{b}} $. But for readability reasons in this 
section we skip the index.

Now let's check $ \xi(X) $ with $X$ from \eqref{Xdef}:
\begin{eqnarray}
X &=& \frac{ -\Veff }{
            \frac{1}{\m{}} + \frac{ \Rad{}^2 }{ \J{} } +
            \frac{1}{\m{}} + \frac{ \Rad{}^2 }{ \J{} }
          } \nonumber\\
&=& - \frac{ \vX{a} - \vX{b} - \Rad{}(\w{a} + \w{b}) }{ 2\frac{ \J{} + \m{}\Rad{}^2 }{ \m{}\J{} } } \\
\xi(X) 
&=& 2 X (\vX{a} - \vX{b} - \Rad{}(\w{a} + \w{b}) ) +
    2 X^2 \frac{\J{}+\m{}\Rad{}^2}{\m{}\J{}} \\
&=& - \frac{ 
    (\vX{a} - \vX{b} - \Rad{}(\w{a} + \w{b}) )^2 
}{
    \frac{ \J{}+\m{}\Rad{}^2 }{ \m{}\J{} }
} \nonumber\\
&+& \frac{ 1 }{ 2 }\cdot \frac{ 
    (\vX{a} - \vX{b} - \Rad{}(\w{a} + \w{b}) )^2 
}{
    \frac{ \J{}+\m{}\Rad{}^2 }{ \m{}\J{} } 
} \\
&=& - \frac{ 1 }{ 2 } \cdot \frac{
    (\vX{a} - \vX{b} - \Rad{}(\w{a} + \w{b}) )^2 
}{
    \frac{ \J{}+\m{}\Rad{}^2 }{ \m{}\J{} }
} < 0 \\
\Longrightarrow \xi(2\cdot X) &=& 0 \\
\Longrightarrow \min_{\forall x}\xi(x) &=& \xi(X)
\end{eqnarray}

That means, as long as our values for $x$ lie between 0 and $X$, we've got $\xi
< 0$ and we're on the sunny side of life: We don't gain energy. Worst case we
loose. I don't exactly understand where in the above equations this energy gets
lost, but for the first we seem to have a plausible, more or less simple and
handleable model for what happens with take-outs. The physical effect causing
this loss surely is the friction rock/rock and $ \xi $ equals the energy lost
here.


\subsection{Resumee}

The model seems to work rather fair and tells that both rocks don't separate
under $ 90^0 $ in general. Spin\index{spin!influence on momentum}
has influence on direction of movement and vice versa.

But still there's a friction rock/ice that isn't included in this model, so 
it cannot be perfect in all situations.

\textbf{Improved models are to be developed!}

