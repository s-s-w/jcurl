\section{Denny Model}
The curl model by Mark Denny published in \cite{denny:98} uses polynomes of 
fourth degree to model a decent curl.

\subsection{Equations of Motion}
\begin{eqnarray}
\eqlab{denX}
x(t) &\approx& - \frac{ b v_0^2 }{ 4 \epsilon R \tau } \left( \frac{ t^3 }{ 3 } 
- \frac{ t^4 } { 4 \tau } \right) \\
\eqlab{denY}
y(t) &\approx& v_0 \left( t - \frac{ t^2 }{ 2\tau } \right) \\
\eqlab{denW}
\omega(t) &\approx& \omega_0 \left(1 - \frac{t}{\tau}\right)^{1/\epsilon} \\
\eqlab{denA}
\alpha(t) &\approx& \int_0^t \omega(t) \dt = 
-{{\omega_{0}\,\varepsilon\,\left(\tau\,\left({{\tau-t}\over{\tau}} 
\right)^{{{1}\over{\varepsilon}}}-t\,\left({{\tau-t}\over{\tau}} 
\right)^{{{1}\over{\varepsilon}}}-\tau\right)}\over{\varepsilon+1}}
\end{eqnarray}

\bigskip
\begin{tabular}{lll}
$t$		& [s] & time \\
$x$		& [m] & curl \\
$y$		& [m] & distance along the track \\
$b$ 	& [1] & parameter for the curl's magnitude \\
$\epsilon$	& [1] & parameter $ \frac{ I }{ mR^2 } $\\
$\tau$	& [s] & total time until $v=0$ \\
$R$		& [m] & radius of the touching area rock/ice $\approx$ 6.3e-2 \\
$\mu$	& [1] & friction coefficient rock/ice \\
$I$		&  & moment of inertia (z-direction) \\
$g$		& [N/kg] & 9.81 gravitation \\
\end{tabular}
\bigskip

The final properties are:
\begin{eqnarray}
\eqlab{denTau}
\tau    & = &     \frac{ v_0 }{ \mu g } \\
x(\tau) &\approx& - \frac{ b_\mathrm{LR} }{ 12\epsilon } \frac{y^2(\tau) }{R} \\
y(\tau) &\approx& \frac{ v_0^2 }{ 2\mu g }
\end{eqnarray}

\subsection{Ice Friction from Draw-to-Tee Time}
For a draw-to-tee we have -- with distance Hog-to-Tee $Y$ and Draw-to-Tee time 
$T$ --
\begin{eqnarray}
y(T) &=& Y \\
\dot y(T) &=& 0
\end{eqnarray}
an thus

\subsection{Initial speed from split time}
Setting the coordinate origin to Center Line $\times$ Hog Line and playing 
along the center line we get $v_0$ at the Hog-Line for the Split Time $t_S$ and 
the distance Back-Line to Hog-Line $Y$:

\begin{eqnarray}
y(-t_S) &=& -Y \\
y(0) &=& 0
\end{eqnarray}


