% $Id$
\section{Without spin, with loss of energy\label{lossI}}

If we look at the rocks hitting each other, what can we see? I mean which
equations do occur? First of all only the components along the
hitting-direction interact at all, the rest remains unchanged. For these
parallel components we state:

\begin{eqnarray}
E_{Kin_1} &=& E_{Kin_2} + U \quad\mbox{with $U$ = lost energy} \\
\vec{F}_a + \vec{F}_b + \vec{\xi} &=& 0
\end{eqnarray}

If we assume the lost momentum to be small, e.g.\ because of a very short time
of contact ($\ll 1$) and a \emph{not} very huge friction rock/ice ($\not\gg
1$), we can assume $ \xi \approx 0 $ and solve this set of equations.

\iffalse
We get
\begin{eqnarray}
\aft{\pY{a}} &=& \bef{\pY{a}} + \bef{\pY{b}} - \aft{\pY{b}} \\
\aft{\pY{b}} &=&
    \frac{
	\sgn(\bef{\pY{a}}-\bef{\pY{b}})
    }{
	\m{a} + \m{b}
    }
    \sqrt{
    	\bef{\pY{a}}{}^2 \m{b}^2
	- 2 \bef{\pY{a}} \bef{\pY{b}} \m{a} \m{b}
	+ \m{a} ( \m{a} \bef{\pY{b}}{}^2 - 2 \m{a} \m{b} U - 2 \m{b}^2 U)
    } \nonumber\\
&+&
    \frac{
	\m{b} (\bef{\pY{a}} + \bef{\pY{b}})
    }{
	\m{a} + \m{b}
    }
\end{eqnarray}
\else
If we assume equal masses, things end up pretty simple --- quite as we like it!
We get
\begin{eqnarray}
\aft{\pY{a}} &=& \bef{\pY{a}} + \bef{\pY{b}} - \aft{\pY{b}} \\
\aft{\pY{b}} &=&
    \frac{
	\sgn(\bef{\pY{a}}-\bef{\pY{b}})
    }{ 2 }
    \sqrt{
	\left( \bef{\pY{a}} - \bef{\pY{b}} \right)^2 - 4 U
    }
    + \frac{
	\bef{\pY{a}} + \bef{\pY{b}}
    }{ 2 } \\
\aft{\pY{a}} &=&
    \frac{
	\sgn(\bef{\pY{b}}-\bef{\pY{a}})
    }{ 2 }
    \sqrt{
	\left( \bef{\pY{a}} - \bef{\pY{b}} \right)^2 - 4 U
    }
    + \frac{
	\bef{\pY{a}} + \bef{\pY{b}}
    }{ 2 }
\end{eqnarray}
\fi

\subsection{The loss' amount}

How big has the loss of energy $ U $ to be, to reduce the remaining path for a
given distance $ \Delta s $? It's got to be the friction's work along $ \Delta
s $. If assuming constant coulomb friction $ |F| = \mu m g $ we get
%
\begin{eqnarray}
|F| &=& \mu m g \\
W   &=& F \cdot s \\
\Delta E &=& W \\
\Longrightarrow
U &=& \Delta s \cdot \mu m g
\end{eqnarray}

\subsection{Resumee\label{lossproblem}}

This model workes fine for rather full hits. But as the hits become extreme
thin --- the speed-component in hitting direction very small --- we get weird
results, e.g.: The hitter changes it's direction, but the hitting rock remains
unmoved! This tells me to include all $ \pY{} $, $ \pX{} $ and $ \L{} $ into
the model, but how?

