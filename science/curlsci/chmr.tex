% $Id$
\section{My own considerations}

\subsection{Introduction}

The rocks are not running straight in general: Besides slowing down they
\emph{curl} to one side. So the force $ \vec{F} $ caused by friction
rock/ice\index{friction!rock/ice} is not parallel to the speed $ \vec{v}
$. Neihter it is constant. The most general form is:
\begin{eqnarray}
    \vec{ F } &=& \vec{ F }( \vec{v}, \omega, \vec{x}, \alpha )
\end{eqnarray}

If ice and rock are homogenous and free of debri it simplifies a little
bit:
\begin{eqnarray}
    \vec{ F } &=& \vec{ F }( \vec{v}, \omega )
\end{eqnarray}

The components of $ \vec{ F } $ are:
\begin{equation}
\begin{array}{ll}
F_y \parallel \vec{v} & \textrm{force parallel running direction} \\
F_x \perp     \vec{v} & \textrm{force perpendicular to running direction}
\end{array}
\end{equation}

\subsection{Parameters to describe the friction}

Statements:
\begin{equation}\begin{array}{l@{\textrm{ for }}ll}
F_x \approx 0 & \omega \gg \omega_0 (\approx \frac{2.5\pi}{25s}) & \textrm{Straight running spinners} \\
F_x \approx 0 & \omega \approx 0    & \textrm{Straight if without handle} \\
F_x \max      & \omega = \omega_0   & \textrm{max.\ curl for a smooth handle} \\
F_y \ll       & \omega \gg \omega_0 & \textrm{few friction for spinners} \\
F_y \gg       & \omega \approx 0    & \textrm{much friction if without handle} \\
\end{array}\end{equation}

Practicable parameters to describe the ice might be:
\begin{itemize}
\item time hog to tee for a draw-to-tee,
\item curl for a draw-to-tee,
\item curl for a take-out of defined speed (e.g.\ 9s hog-to-tee),
\item do the rocks slow down quick once they started to or don't they.
\end{itemize}

\subsection{Base for $ \vec{ F } $}

The area of contact rock/ice is a fine ring on the rock's bottom. $ \vec{ F } $
is the sum of the forces affecting this ring. For a given spot $ (x,y)^T $
on this ring this force is in general:
\begin{equation}
    \vec{f} = \vec{f}(\vec{v}, \omega, x, y )
\end{equation}

Then $ \vec{ F } $ is:
\begin{equation}
    \vec{F} = \Int_R^{R+\Delta R} \Int_0^{2\pi}
                  \vec{f}(\vec{v}, \omega, x, y )
              \diff{\varphi} \diff{r}
\end{equation}
