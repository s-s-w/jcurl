\documentclass[titlepage,11pt]{article}
\usepackage{pslatex}
\usepackage[T1]{fontenc}
\usepackage{graphicx}

\author{Canadian Curling Association\\
	Level 4 Research Project\\
	Ron Mills}
\date{September 1992}
\title{Planning Takeouts and Raises}

\newlength\leftcolumnwid % width of picture left column
\leftcolumnwid = \textwidth
\advance\leftcolumnwid by -\columnsep
\advance\leftcolumnwid by -75mm

\newlength\picwidth
%\picwidth=\textwidth\advance\picwidth by 3cm
\picwidth=\textwidth

\parindent=0mm
\parskip=1em
\renewcommand{\baselinestretch}{1.2}

\setcounter{topnumber}{5}
\setcounter{bottomnumber}{5}
\setcounter{totalnumber}{10}
\renewcommand{\topfraction}{0.75}
\renewcommand{\bottomfraction}{0.75}
\renewcommand{\textfraction}{0.25}
\renewcommand{\floatpagefraction}{0.75}
\floatsep=1.5\parskip
\textfloatsep=1.5\parskip
\intextsep=1.5\parskip

\newcommand{\myfrac}[2]{\ensuremath{ \,{}^{#1}\!/_{#2} }}

\newcommand{\img}[1]{\includegraphics{#1}}

\newcommand{\pict}[4]{%
\begin{figure}[htp]%
    \centering%
	{#4}
    \def\boxX{#3}\ifx\boxX\empty
	\caption{\label{#1} #2}%
    \else
	\caption[#3]{\label{#1} #2}%
    \fi
\end{figure}%
}

\newcommand{\picfile}[4]{\pict{#1}{#2}{#3}{\img{#4}}}

\newcommand{\picb}[5]{%
    \def\BoXx{#4}
    \ifx\BoXx\empty
	\pict{#1}{#2}{#3}{#5}
    \else
	\pict{#1}{#2}{#3}{\makebox[0mm]{\makebox[\picwidth]{%
	\begin{minipage}{\leftcolumnwid}\sloppy
	#4%
	\end{minipage}\hspace{\columnsep}\raisebox{-.5\totalheight}{#5}}}}%
    \fi
}
\newcommand{\prtfloats}{\clearpage}

\begin{document}
\DeclareGraphicsExtensions{.eps}
\maketitle
\listoffigures
\section{Introduction}

The purpose of this presentation is to give curlers and coaches a better
understanding of the principles involved in the selection and execution of shots
when other rocks are involved in take-outs and raises. This knowledge will
become more important when more rocks come into play as a result of any rule
changes.

\section{Dimensions and Geometry}

Figure \ref{size}\ reviews the dimensions of the house area used in this project.

A curling rock must have a maximum circumference of 36", giving a maximum
diameter of $ 36 / \pi = 11.5" $ $(\rm C = \pi D )$.

Rocks at curling clubs locally were in the 10.5" to 10.75" range (can navigate
smaller ports!). Rocks in this study were $ 10 \myfrac{5}{8}" $ in diameter.

\picb{formula}{Formulae}{}{%
  \begin{math}\begin{array}[t]{lll}
   \displaystyle \frac{a}{d} &=& \displaystyle\frac{r}{l} \\[1.5em]
   \displaystyle a &=& \displaystyle\frac{r \cdot d}{l}
  \end{array}\end{math}%
}{%
 \unitlength=1mm
 \begin{picture}(20,35)%(0,16)
 \put(-10,0){\line(1,0){20}}
 \put(-10,0){\line(1,2){16}}
 \put( 10,0){\line(-1,2){16}}
 \put( -6,32){\line(1,0){12}}
 \put(	0,0){\line(0,1){32}}
 \put(	0,33){\makebox(0,0)[cb]{a}}
 \put(	1,29){\makebox(0,0)[lc]{r}}
 \put(	1, 7){\makebox(0,0)[lc]{l}}
 \put(	1,-1){\makebox(0,0)[ct]{d}}
 \end{picture}%
}

\picfile{size}{Dimensions}{}{size.eps}

\subsection*{Assumptions}

\begin{itemize}
\item The striking band of the rocks are free from pitting and chips.
\item The ice surface is free of debri.
\item The ice is broken in over all areas and is 13--14 seconds hog to hog for a
      tee weight draw (23-24 seconds hog to tee).
\end{itemize}

\subsection*{Formulae}

The formula in figure \ref{formula}\ will be used later in this presentation.

\prtfloats

\section{Proper angles}

Many curlers judge angles by instinct with varying degrees of success. Judging
angles does not come naturally and is learned by experience. A common mistake is
shown in figure \ref{fig-2}. Rock B is to be raised onto rock C and rock A is the thrown
rock. "$\times$" is the proper spot to hit since it is a direct line between the
centres of rocks B and C. If an attempt is made to hit mark $\times$ with the centre of
the thrown rock, then rock B will be too full since the contact point is wrong.

\picfile{fig-2}{A Common Mistake}{}{fig02.eps}

\subsection*{Line up the centres}

The proper way to play the shot is shown in figure \ref{fig-3}. Visualize the thrown rock
A touching rock B at spot "$\times$" so that the three centres are lined up, then
visualize an \emph{area of overlap} between the thrown rock and rock B. The skip
and vice skip can then call the line properly. Figure \ref{fig-4}\ shows five common areas
of overlap that curlers should become familiar with.

\picfile{fig-3}{Rock A coming from above hitting B, B moves on to position C.
Touching spot A/B is on line with centre of A, B, C.}{Area of Overlap}{fig03.eps}

\pict{fig-4}{Common Areas of Overlap}{}{%
\begin{minipage}{.8\textwidth}
\hspace*{\fill}\img{fig04a.eps}
\hspace*{\fill}\img{fig04b.eps}
\hspace*{\fill}\img{fig04c.eps}
\hspace*{\fill}\\[1.5em]
\hspace*{\fill}\img{fig04d.eps}
\hspace*{\fill}\img{fig04e.eps}
\hspace*{\fill}
\end{minipage}}

\subsection*{The "T" principle}

When playing a horizontal double as in figure \ref{fig-5}, the $\times$ point
to hit rock B is determined by visualizing the letter "T" as shown, then
visualize the thrown rock in position A to then determine the area of overlap
to make the shot.

\picfile{fig-5}{The T-priciple}{}{fig05.eps}

This "T" principle is very important and can be used in many crucial situations.
An example is shown in figure \ref{fig-6}. Figure \ref{fig-7}\ is Scotland's famous triple against
the Canadian men's team at the 1992 World Championship (Hammy McMillan vs Vic
Peters). The spot to hit rock A can be determined by using the "T" rule starting
with rock C. This will prove why the shot was possible.

\picb{fig-6}{T-priciple in action}{}{%
\begin{itemize}
\item Last rock on last end
\item A, B are opposition rocks
\item You need a 3 ender
\item C, D are your rocks
\item Where will A go on a full or \myfrac{3}{4} hit?
\end{itemize}}{\img{fig06.eps}}

\picfile{fig-7}{\emph{The} Triple Scotland vs Canada, 1992 world mens
		semi final.}{\emph{The} Triple}{fig07.eps}

\clearpage

\subsection*{Turn selection}

Figures \ref{fig-8}\ and \ref{fig-9}\ demonstrate the options of playing an in turn or an out turn
double on rocks B and C assuming the ice curls quite a bit. \textbf{Regardless of
which turn to play you must hit rock B at exactly the same spot.} The out turn
in figure \ref{fig-8}\ gives you a larger area of overlap and thus is the better percentage
shot. Similarly, the in turn in figure \ref{fig-9}\ is the better shot.


\picb{fig-8}{Turn decisions}{}{%
\begin{itemize}
\item Out turn the best here
\item Ice curls a lot
\end{itemize}
}{\img{fig08.eps}}

\picb{fig-9}{Turn decisions}{}{%
\begin{itemize}
\item In turn the best here
\item Ice curls a lot
\end{itemize}
}{\img{fig09.eps}}


This \emph{area of overlap} is the most important factor in determining which
turn to play. Many curlers believe the out turn in figure \ref{fig-8}\ causes the shooter
to \emph{"jump"} across at a bigger angle than an in turn. I believe this effect
to be very minimal with a normal rotation. The major consideration should be
the larger area of overlap in the different turns. On straight ice either turn
can be used.

Another interesting fact is shown in figure \ref{fig-10}. Rocks A and B are both on the
same angle from rock C, but rock B has a smaller \emph{area of overlap} than A to
make the double. You can not make a fixed rule that a $ 30^\circ $ double for
example is always a half hit on the first rock.

\picb{fig-10}{Doubles}{}{Area of overlap decreases as rocks on the same
angle get closer together.}{\img{fig10.eps}}

\prtfloats

\section{Accuracy required for raises}

Figure \ref{fig-11}\ shows two rocks a distance "L" apart. Assume rock B will be raised
onto rock C with at least half hit on rock C. This will ensure C is removed from
play.

\picb{fig-11}{Accuracy}{}{%
\begin{math}\begin{array}[t]{lll}
\displaystyle a &=& \displaystyle\frac{ r \cdot d }{ l } \\[1em]
		&\approx& \displaystyle\frac{ 5.5'' \cdot 11'' }{ l } \\[1em]
		&=& \displaystyle\frac{ 60.5 }{ l } \\
\end{array}\end{math}
}{\img{fig11.eps}}

\iffalse
From the formula in the introduction $(a/d = r/L)$

$ a = r \times \frac{ d }{ L } $ where $ r $ = radius = 5.5" and $ d $ =
diameter = 11".

Thus $ a = \frac{60.5"}{L} $ when $ L $ = distance apart.

In figure \ref{fig-12}\ the rocks are 20 feet apart, thus $ a = \frac{60.5}{240} = 0.25"$

In figure \ref{fig-13}\ the rocks are 10 feet apart, thus $ a = \frac{60.5}{120} = 0.5"$

In figure \ref{fig-14}\ the rocks are  5 feet apart, thus $ a = \frac{60.5}{ 60} = 1.0"$

Figure \ref{fig-15}\ is an %actual
relative size drawing of the accuracy required at the three distances.
\fi

\picb{fig-12}{Accuracy --- 20 foot raise}{}{%
\begin{math}\begin{array}[t]{lll}
    a &=& \mbox{area to hit} \\[1.5em]
%      &=& \mbox{radius} \cdot \mbox{diameter} / \mbox{length} \\[1.5em]
      &\approx& \mbox{radius} \cdot \frac{\mbox{diameter}}{\mbox{length}} \\[1.5em]
      &=& 5.5 \cdot 11 / 240 \\[1.5em]
      &=& 0.25"
\end{array}\end{math}
}{\img{fig12.eps}}

\picb{fig-13}{Accuracy --- 10 foot raise}{}{%
\begin{math}\begin{array}[t]{lll}
    a &=& \mbox{area to hit} \\[1.5em]
%      &=& \mbox{radius} \cdot \mbox{diameter} / \mbox{length} \\[1.5em]
      &\approx& \mbox{radius} \cdot \frac{\mbox{diameter}}{\mbox{length}} \\[1.5em]
      &=& 5.5 \cdot 11 / 120 \\[1.5em]
      &=& 0.50"
\end{array}\end{math}
}{\img{fig13.eps}}

\picb{fig-14}{Accuracy --- 5 foot raise}{}{%
\begin{math}\begin{array}[t]{lll}
    a &=& \mbox{area to hit} \\[1.5em]
%      &=& \mbox{radius} \cdot \mbox{diameter} / \mbox{length} \\[1.5em]
      &\approx& \mbox{radius} \cdot \frac{\mbox{diameter}}{\mbox{length}} \\[1.5em]
      &=& 5.5 \cdot 11 / 60 \\[1.5em]
      &=& 1"
\end{array}\end{math}
}{\img{fig14.eps}}

\picfile{fig-15}{Accuracy --- in relation to diameter}{}{fig15.eps}

\subsection*{Propabilities}

The odds of making the raises above depend on
\begin{enumerate}
\item The distance apart
\item The skill level of your team
\begin{itemize}
    \item reading ice
    \item sweeping
    \item throwing
    \item calling line
\end{itemize}
\item Ice conditions
\end{enumerate}

For example, on the 20" raise you have to be accurate to within \myfrac{1}{4}".
If your team is only accurate to with 11" (ie.\ a half hit on either side), then
the odds of making this shot are about 1 in 44 since there are 44 --
\myfrac{1}{4} inches in 11 inches. A highly skilled team may be accurate within
2" and thus have a 1 in 8 chance of making a 20" raise.

\prtfloats

\section{Loss of momentum}

When a thrown rock strikes another rock there is a slight loss of momentum when
struck full on. This is of no consequences in take outs, but is important in
tap back raises. After many test trails, the pattern in figure \ref{fig-16}\ emerged. The
general rule is that a diameter of one rock is lost for each rock raised.
Remember our assumptions on broken in ice with no debri under any of the rocks
(watch out for straw).

\picb{fig-16}{Loss of momentum}{}{%
\begin{itemize}
\item same weight on three draws
\item A, B, C are the outcome
\item A, no raise
\item B, raise one rock
\item C, raise two rocks
\end{itemize}
}{\img{fig16.eps}}


Figure \ref{fig-17}\ shows the other way momentum is lost. When raising rocks at an angle
you need more weight. The smaller the area of overlap, the more weight you need.
Path C shows the weight required to raise rock A onto the button. Distance C is
about the sum of distances A and B.

\picfile{fig-17}{Angle draw raises. Distance $ C \approx A + B $.}{Angle draw raises}{fig17.eps}

This principle can be useful on "free" draws that are heavy but are going to hit
one of your rocks in the house. An alert sweeping call would be to get a
\myfrac{1}{2} hit to save both rocks.

\prtfloats

\section{The \emph{"Drag"} Effect}

An interesting phenomenon occurs when playing raise take outs on rocks that are
touching (frozen together). Figure \ref{fig-18}\ shows this effect. Rocks B and C are
touching. Rock A is the thrown rock and is a half hit on B. C should go along
the normal path but in this case C is \emph{"dragged"} by B slightly towards B's
path. \textbf{This drag effect disappears only after B and C are more than
\myfrac{1}{2}" apart}.

\picfile{fig-18}{The \emph{"Drag"} Effect}{}{fig18.eps}

You can use this principle to your advantage but ist should be used only if you
have practiced it and no other shots are available. Figures \ref{fig-19}\ and
\ref{fig-20}\ show two examples.

\picb{fig-19}{Drag Effect}{}{%
\begin{itemize}
\item B and C are lined up to just miss D
\item A half hit on the right side of B will cause C to hit D
\item The drag effect disappears after B and C are one half inch apart
\end{itemize}
}{\img{fig19.eps}}

\picb{fig-20}{Drag Effect}{}{%
\begin{itemize}
\item A \myfrac{3}{4} hit on the right side of B will cause C to hit D
\item Paths 1, 2 or 3 or any in-between are possible depending on where it is hit
\end{itemize}
}{\img{fig20.eps}}


Figure \ref{fig-21}\ shows a set up which many curlers believe is an \emph{"automatic"} in
that you can hit B anywhere. You must hit B in spot $\times$ to avoid
the \emph{"drag"} effect.

\picfile{fig-21}{Avoid the Drag Effect. Hit B at the position $\times$,
the same as if C wasn't there.}{Avoid the Drag Effect}{fig21.eps}

\prtfloats

\section{Ports}

Figure \ref{fig-22}\ to \ref{fig-24}\ show the path the thrown rock takes when comingg through ports
and wicking on one or both of the rocks. In figure \ref{fig-22}\ there is a small port
between rocks B and C (1 or 2 inches to spare to each side). In figure \ref{fig-23}\ the
port is slightly narrower than one rock. Figure \ref{fig-24}\ demonstrates the
follow-through effect if simultaneously striking B and C when the port is
narrowe than one rock. With $10\myfrac{3}{4}$" diameter rocks a port size down to
8" can be navigated. This is exactly the width of the plastic handle covers used
in many clubs.

\picb{fig-22}{Splitting rocks}{}{%
\begin{itemize}
\item B and C are 13 to 15 inches apart. Fine hit C first, glancing off B
\item A very fine hit on C causes A to take path 1
\item Slightly fuller hits on C progress A's position to 5
\end{itemize}
}{\img{fig22.eps}}

\picb{fig-23}{Splitting rocks}{}{%
\begin{itemize}
\item B and C are about 10 inches apart
\item Fine hitting C first results in path 1
\item Fine hitting B first results in path 2
\end{itemize}
}{\img{fig23.eps}}

\picb{fig-24}{Splitting through a port}{}{%
\begin{itemize}
\item Port sizes of less than 8 inches cannot be navigated. (Width of
      plastic handle).
\item B, C must be hit simultaneously.
\end{itemize}
}{\img{fig24.eps}}


Figure \ref{fig-25}\ shows two different double situations. D, E is a \emph{"natural"}
double, in that a half hit on D is required. B, C is a very difficult double
unless played properly. A half hit on the centre line side of B makes it as easy
as a \emph{"natural"} double.


\picb{fig-25}{Natural doubles}{}{%
\begin{itemize}
\item Defined as two rocks situated so that a half hit on the first one hits
      the second one directly.
\item D and E are a natural double.
\item Characteristics: You cannot roll across the face of E.
\item B, C are an extremely tough double if you attempt to split them. A
      half hit on the centre line side of B makes it as easy as a natural
      double.
\end{itemize}
}{\img{fig25.eps}}

\prtfloats

\section{Practice routines}

All of the principles shown here should be set up in practices. Rocks can be
delivered (pushed) from the nearest hog line to save time.

A very good off season routine would be to have the team practice various shots
on a pool table. Specially:
\begin{itemize}
\item learning about proper area of overlap (full, three quarter, half, one
      quarter hits)
\item learning to visualize the "T" rule for doubles
\item learning about the drag effect
\end{itemize}

A more realistic environment can be set up on a pool table by making a scale
drawing of the house area out to hog line on a sheet of paper. The scale is 5.5
to 1 since a billiard ball is 2 inches in diameter vs 11 inches for a rock.

\end{document}
