\iffalse
History:
    %%%% mr1998/07/13: - Einige inhaltliche Fehler beseitigt,
                         Formeln minimal klarer formuliert.
    %%%% mr1998/04/03: - Einige inhaltliche Fehler beseitigt,
                         und fragliche Stellen mit ^? markiert
                       - einige Gleichungen hinzugef"ugt (mit \NONUMBER)
                       - kleine "Anderungen an den Zeichnungen
                       - Umbenennungen:
                             Befehl: label/ref -> eqlab/eqref
                             v   -> v_R    Geschwindigkeit auf dem Laufring
                             v_0 -> v      Geschwindigkeit des Steins
                             v_min -> v_b  Parameter in der Reibungsfunktion
                             v_0   -> v_a  Parameter in der Reibungsfunktion

\fi
\documentclass[a4paper]{report}
\usepackage{latexsym}
\usepackage{german}
\usepackage[T1]{fontenc}
% \usepackage[cp850]{inputenc}
\usepackage{pictex}
\input{tumlogo}
%Inputfile Frame3cm
\pagestyle{empty}              % keine Kopf oder Fu"szeile
\columnsep      =  12.5mm      % Spaltenabstand 1.25 cm
\columnseprule  =   0.0mm      % keine Trennlinie zwischen den Spalten
\topmargin      =  -5.4mm      % 1 Inch wird automatisch addiert
\headheight     =   7.0mm      % H"ohe der Kopfzeile
\headsep        =   3.0mm      % Abstand Kopzeile - Text
\topskip        =   1.0ex      % Abstand Unterkante erste Zeile - Oberkante Textbereich
\textheight     = 230.0mm      % H"ohe Textbereich
\textwidth      = 150.0mm      % Breite Textbereich
%\footheight    =   7.0mm      % H"ohe Fu"szeile
\footskip       =  10.0mm      % Abstand Unterkante Fu"s - Unterkante Textbereich
\evensidemargin =   4.6mm      % 1 Inch wird automatisch addiert
\oddsidemargin  =   4.6mm      % 1 Inch wird automatisch addiert
\parindent      =   0.0mm      % Absatzeinr"uckung
\parskip        = 1ex plus .4ex minus .2ex       % Absatzabstand
\flushbottom                   % letzte Zeile im der Seite immer ganz unten


\pagestyle{plain}

\parindent=0mm
\parskip=3mm
\jot=1ex          % zus"atzlicher Zeilenabstand innerhalb eqnarray
\setcounter{secnumdepth}{2}
\setcounter{tocdepth}{3}

%%%%%%%%%%%%%%%%%%%%%%%%%%%%%%%%%%%%%%%%%%%%%%%%%%%%%%%%%%%
%%  Spezielle Befehle f"ur Labels und Referenzen
%%%%%%%%%%%%%%%%%%%%%%%%%%%%%%%%%%%%%%%%%%%%%%%%%%%%%%%%%%%
\newcommand{\eqlab}[1]{\label{eq:#1}}
\newcommand{\eqref}[1]{Gl.~(\ref{eq:#1})}

\newcommand{\figlab}[1]{\label{fig:#1}}
\newcommand{\figref}[1]{Fig.~(\ref{fig:#1})}

\newcommand{\tablab}[1]{\label{tab:#1}}
\newcommand{\tabref}[1]{Tab.~(\ref{tab:#1})}

\iftrue
% keep original numbering => don't number new equations
\newcommand{\NONUMBER}{\nonumber}
\else
% don't keep original numbering => number new equations
\newcommand{\NONUMBER}{}
\fi
%%%%%%%%%%%%%%%%%%%%%%%%%%%%%%%%%%%%%%%%%%%%%%%%%%%%%%%%%%%
%%  Spezielle Befehle f"ur gro"se Ausdr"ucke:
%%%%%%%%%%%%%%%%%%%%%%%%%%%%%%%%%%%%%%%%%%%%%%%%%%%%%%%%%%%

\newcommand{\vRo}{ \vec{v}   }   % the rock's center's speed
\newcommand{\vRi}{ \vec{v}_R }   % the rock's running-ring's speed
\newcommand{\vRiAbsDivvRoCOS}[1]{\sqrt{\displaystyle
    1 + 2\frac{\omega #1}{|\vRo|}\cos\varphi +
    \left( \frac{\omega #1}{|\vRo|} \right)^2 } }
\newcommand{\vRiAbsDivvRoXI}[1]{\sqrt{\displaystyle
    1 + 2\frac{\omega #1}{|\vRo|}\xi +
    \left( \frac{\omega #1}{|\vRo|} \right)^2 } }
\newcommand{\SubstCos}{ \sqrt{1-\xi^2} }
\newcommand{\fpolar}[2]{ f( |\vRi| , #2 \xi, #1 #2 \SubstCos ,\omega) }
\newcommand{\Int}{\int\limits}

\unitlength=5mm
\setcoordinatesystem units <\unitlength,\unitlength>
\title{Curling \\ Diplomarbeit an der \oTUM{3ex} \\ Auszug}
\author{Uli Sutor}
\date{1988}

\begin{document}
\maketitle
\tableofcontents
\listoffigures

\setcounter{chapter}{8}
\chapter{Theorie}

\begin{figure}[htb]
\begin{center}\mbox{%
  %
  % PicTeX-Bild:
  %
  % Bild 80 aus Uli Sutors Diplomarbeit:
  % Der Stein auf dem Rink, Koordinatensysteme
  %
  \beginpicture
  \setplotarea x from -5 to 5, y from -5 to -10
  % The rink-system:
%  \iffinalplot
  \circulararc 360 degrees from 0.25 0 center at 0 0	  % Button
  \circulararc 360 degrees from 1.0  0 center at 0 0	  % 4foot
  \circulararc 360 degrees from 2.0  0 center at 0 0	  % 6foot
  \circulararc 360 degrees from 3.0  0 center at 0 0	  % 8foot
%  \fi
  \put {{\huge S'}}         [tl] at -5   10               % coordsys: title
%  \put {O'}                 [tl] at  0.2 -0.2             % Zero
  \put {\vector(1,0){10}}   [Bl] at -5	  0		  % x axis = Tee-line
  \put {p}		    [l]  at  5	  0		  % x axis label
  \put {\vector(0,1){15}}   [Bl] at  0	 -5		  % y axis = Centerline
  \put {q}		    [b]  at  0	 10		  % y axis label
  %
  % the rock-system:
  %
%  \iffinalplot
  \circulararc 360 degrees from 2.5 5 center at 2 5	  % Rock
%  \fi
  \put {{\huge S}}	    [tl] at  4	  9		  % coordsys: title
%   \put {O}		    [tl] at  2.4  4.6             % Zero
  \put {\vector(4,-1){1.6}} [Bl] at  2	  5		  % x axis
  \put {\line(-4,1){1}}     [Bl] at  2	  5		  % neg. x axis
  \put {x}		    [l]  at  3.6  4.6		  % x axis label
  \put {\vector(1,4){0.8}}  [Bl] at  2	  5		  % y axis
  \put {\line(-1,-4){0.25}} [Bl] at  2	  5		  % neg. y axis
  \put {y}		    [b]  at  2.8  8.4		  % y axis label
  \put {\vector(1,4){0.4}}  [Bl] at  2	  5		  % speedvector
  \put {$\vRo$}	            [tr] at  1.9  6.5		  % speed label
  \endpicture
}\end{center}
\caption{Der Stein auf dem Eis-Rink\figlab{pictureA}}
\end{figure}

\begin{figure}[htb]
\begin{center}\mbox{%
  %
  % PicTeX-Bild:
  %
  % Bild 81 aus Uli Sutors Diplomarbeit:
  % Der Stein mit eigenem Koordinatensystem
  %
  \beginpicture
  \setplotarea x from -9 to 9, y from -7 to 9
  % The rink-system:
%  \iffinalplot
  \circulararc 360 degrees from 3.0 0 center at 0 0
  \circulararc 360 degrees from 3.5 0 center at 0 0
  \circulararc 360 degrees from 5.5 0 center at 0 0
%  \fi
  \put {{\huge S}}	    [tl] at -7	  8
%  \put {O}		    [tr] at -0.1 -0.1
  \put {\vector(1,0){18}}   [Bl] at -9	  0		  % x axis
  \put {x}		    [l]  at  9	  0
  \put {\vector(0,1){16}}   [Bl] at  0	 -7		  % y axis
  \put {y}		    [b] at   0    9
  %
  \put {\vector(0,1){7}}   [Bl] at  0	 0		  % v
  \put {$\vRo$}	           [cl] at  0.1  7
  %
  % Nun der ganze Kleinkram:
  %
  % omega:
%  \iffinalplot
  \circulararc -32 degrees from -2 0.5 center at 0 0
%  \fi
  \put {\vector(1,1){0.1}}  [Bl] at -1.35 1.55
  \put {$\vec{\omega}$}     [Br] at -1.8  1.1
  %
  % Delta R
  \put {\vector(1,-2){0.5}} [Bl] at -2.06 4.13
  \put {\vector(-1,2){0.5}} [Bl] at -0.84 1.68
  \put {$\Delta R$}	    [Br] at -2	  3
  %
  % \vec{r}
  \put {\vector(1,1){2.3}}  [Bl] at  0	  0
  \put {$\vec{r}$}	    [Br] at  1	  1
  %
  % \phi
  \circulararc 45 degrees from 2 0 center at 0 0
  \put {$\varphi$}	    [Bl] at  2.1  0.7
  %
  % R
  \put {\vector(3,-1){2.85}}[Bl] at  0	  0
  \put {R}		    [tl] at  1.5 -0.75
  \endpicture
}\end{center}
\caption{Der Laufring des Steins}
\end{figure}

\section{Das Curlen}

\begin{figure}[htb]
\begin{center}\mbox{%
  %
  % PicTeX-Bild:
  %
  % Die Bahnkurve eines Steins
  %
  \beginpicture
  %\unitlength=5mm
  %\setcoordinatesystem units <\unitlength,\unitlength>
  \setplotarea x from -5 to 5, y from -13 to 5
%  \iffinalplot
  \circulararc 360 degrees from 0 0.5 center at 0 0
  \circulararc 360 degrees from 0 1   center at 0 0
  \circulararc 360 degrees from 0 2   center at 0 0
  \circulararc 360 degrees from 0 3   center at 0 0
%  \fi
  \setquadratic
  \put {\line(1,0){9}}	  [Bl] at -4.5	 3    % Back - line
  \put {\line(1,0){9}}	  [Bl] at -4.5	 0    % Tee - line
  \put {\line(1,0){9}}	  [Bl] at -4.5 -11.5  % Hog - line
  \put {\line(0,1){15.5}} [Bl] at  0   -12.5  % Centre - line
  %
  \plot 0.9  -11   1.5	-3   3	 3.5 /
  \put {\vector(1,2){0.1}} [Bl] at 3.1 3.6
%  \iffinalplot
  \circulararc 360 degrees from 3.3 3.8 center at 3.3 4.3
%  \fi
  %
  \setdashes <1mm>
  \plot 0.45 -11   0.75 -3   1.5 4 /
  \put {\vector(1,3){0.1}} [Bl] at 1.6 4.3
  \endpicture
}\end{center}
\caption{Typische Bahnkurve eines Curlingsteins}
\end{figure}

\subsection{Herleitung der Gleichungen\label{herleitung}}

Geschwindigkeit eines Punktes des Laufrings

\begin{tabular}{@{}ll}
$ \vRi $	  & Geschwindigkeit eines Punktes auf dem Laufring \\
$ \vRo $	  & Geschwindigkeit des Steins \\
$ \vec{\omega} $  & Winkelgeschwindigkeit des Steins
\end{tabular}

\begin{eqnarray}
\vRi &=& \vRo + \vec{\omega} \times \vec{r} =
{-\omega r \sin\varphi \choose |\vRo| + \omega r \cos\varphi}
    \eqlab{speed-vektor}\\
|\vRi | &=& |\vRo| \vRiAbsDivvRoCOS{r}
    \eqlab{speed-skalar}
\end{eqnarray}

F"ur die geschwindigkeitsabh"angige Kraft gilt

\begin{tabular}{@{}ll}
$ \vec{f} $     & Reibungskraft pro Fl"acheneinheit des Laufrings
\end{tabular}

\begin{equation}
\vec{f} = - \frac{\vRi}{|\vRi|} f \eqlab{reibung-vektor}
\end{equation}

dabei wird f"ur $ f $ die allgemeinste Form angenommen
%
\begin{equation}
f = f(|\vRi|,x,y,\omega)    \eqlab{reibung-skalar}
\end{equation}
%
F"ur die Umwandlung in Polarkoordinaten gilt
%
\begin{eqnarray}
x &=& r \cos\varphi \\
y &=& r \sin\varphi
\end{eqnarray}
%
damit ergibt sich f"ur \eqref{reibung-skalar}
%
\begin{equation}
f = f(|\vRi|,r\cos\varphi,r\sin\varphi,\omega) \eqlab{reibung-polar}
\end{equation}

\subsubsection{Kraft senkrecht auf Laufrichtung (bewirkt curlen)}

Wenn man in \eqref{reibung-vektor} nur die $ x $-Komponente von
\eqref{speed-vektor} einsetzt und sowohl "uber den ganzen Laufring, als auch
"uber die Breite des Rings integriert, erh"alt man

\begin{tabular}{@{}ll}
$ \Delta R $	  & Breite des Laufrings \\
$ R $		  & Radius des Laufrings
\end{tabular}

\begin{equation}
F_x = - \Int_{R}^{R+\Delta R} dr \Int_{0}^{2\pi} d\varphi\; r
\frac{-( \omega r \sin\varphi ) f}{|\vRo| \vRiAbsDivvRoCOS{r} }
\end{equation}
%
mit der Substitution
%
\begin{equation}
\xi = \cos\varphi, \quad \textrm{dann}\quad \frac{d\xi}{d\varphi} = -\sin\varphi =
\left\{ {- \SubstCos , \; 0	\leq\varphi < \pi \atop
	 + \SubstCos , \; \pi\leq\varphi < 2\pi }\right.
	  \eqlab{substitution}
\end{equation}
%
folgt mit \eqref{reibung-polar}
%
\begin{equation}
F_x = -\frac{\omega}{|\vRo|} \Int_{R}^{R+\Delta R} dr \; r^2 \left[
\Int_{1}^{-1} d\xi\;\frac{\fpolar{+}{r}}{ \vRiAbsDivvRoXI{r} } +
\Int_{-1}^{1} d\xi\;\frac{\fpolar{-}{r}}{ \vRiAbsDivvRoXI{r} } \right]
\end{equation}
%
und damit
%
\begin{equation}
F_x = -\frac{\omega}{|\vRo|} \Int_{R}^{R+\Delta R} dr\;r^2
\Int_{-1}^{1} d\xi\;\frac{\fpolar{-}{r}-\fpolar{+}{r}}{ \vRiAbsDivvRoXI{r} }
\eqlab{kraft-x}
\end{equation}

Da die Summe im Nenner von \eqref{kraft-x} nur einen Unterschied in der
$ y $-Komponente der Kraft aufweist, ergibt sich nur dann ein von Null
verschiedener Wert, wenn die Kraft eine $ y $-Abh"angigkeit hat.

\subsubsection{Kraft parallel Laufrichtung (bremst Stein)}

Wenn man in \eqref{reibung-vektor} nur die $ y $-Komponente von
\eqref{speed-vektor} einsetzt, ergibt sich
%
\begin{eqnarray}
F_y &=& -\Int_{R}^{R+\Delta R} dr \; \Int_{0}^{2\pi} d\varphi\; % r %%%% mr1998/04/03: r removed!
\frac{ (|\vRo|+\omega r\cos\varphi) f }{ |\vRo| \vRiAbsDivvRoCOS{r}  } =		      \\
&=& -\Int_{R}^{R+\Delta R}dr\;\left[
    \Int_{1}^{-1}d\xi\; (-  \SubstCos^{-1^?})
        \frac{\left(1+\frac{\omega r}{|\vRo|}\xi\right)
             }{ \vRiAbsDivvRoXI{r} } \fpolar{+}{r} +
    \right.
\nonumber\\&&
    +\left.\Int_{-1}^{1}d\xi\; (+  \SubstCos^{-1^?})
        \frac{ \left( 1+\frac{\omega r}{|\vRo|}\xi \right)
             }{ \vRiAbsDivvRoXI{r}  } \fpolar{-}{r}
        \right]
\end{eqnarray}

Die beiden Integranden kann man wieder zusammenfassen
%
\begin{equation}
F_y =  -\Int_{R}^{R+\Delta R} dr \; \Int_{-1}^{1} d\xi \;
%%%% mr1998/04/03: Changed: \frac{ \left( 1 + \frac{\omega r}{|\vRo|}\xi \right)  \SubstCos
    \frac{ \left( 1 + \frac{\omega r}{|\vRo|}\xi \right)   \SubstCos^{-1^?}
         }{ \vRiAbsDivvRoXI{r}  }
    \left[ \fpolar{+}{r} + \fpolar{-}{r} \right]
    \eqlab{kraft-y}
\end{equation}

\subsubsection{Drehmoment (bremst Drehung des Steins)}

F"ur das Drehmoment gilt mit \eqref{reibung-vektor}

\begin{tabular}{@{}ll}
$ \mu $   & Drehmoment auf Fl"achenelement des Laufrings
\end{tabular}

\begin{equation}
\vec{\mu} = \vec{r}\times \vec{f} = -\frac{f}{|\vRi|}\vec{r}\times\vec{v}_R
	\eqlab{dreh-vektor}
\end{equation}

Umgerechnet in Polarkoordinaten folgt in Vektorschreibweise f"ur das
Vektorprodukt
%
\begin{eqnarray}
\vec{r}\times\vec{v}_R &=& \left(\begin{array}{@{}c@{}}
	r\cos\varphi \\ r\sin\varphi \\ 0
\end{array}\right) \times
\left(\begin{array}{@{}c@{}}
	-\omega r\sin\varphi \\ |\vRo| + \omega r\cos\varphi \\ 0
\end{array}\right) =
\left(\begin{array}{@{}c@{}}
	0 \\ 0 \\ r\cos\varphi(|\vRo|+\omega r\cos\varphi) +
                  \omega r^2\sin^2\varphi
\end{array}\right) \\
%%%% mr1998/07/13: new:
&=& \left( |\vRo| r \cos\varphi + \omega r^2 \right) \cdot
    \left(\begin{array}{@{}c@{}} 0 \\ 0 \\ 1 \end{array}\right) \nonumber
\end{eqnarray}

F"ur den Betrag des Drehmoments also
%
\begin{equation}
%%%% mr1998/07/13: _3 added
\mu_3 = -\frac{f}{|\vRi|}\cdot \left(\vec{r}\times\vec{v}_R\right)_3
      = -\frac{f}{|\vRi|}\cdot \left(|\vRo| r \cos\varphi + \omega r^2 \right)
\eqlab{dreh-skalar}\end{equation}
%
setzt man \eqref{dreh-skalar} in \eqref{dreh-vektor} ein, mit
\eqref{speed-skalar}
%
\begin{equation}
%%%% mr1998/07/13: _3 added
\mu_3 = -(|\vRo| r \cos\varphi + \omega r^2) \frac{f}{|\vRi|} =
- \frac{r(\cos\varphi + \frac{\omega r}{|\vRo|}) f }{ \vRiAbsDivvRoCOS{r} } \eqlab{mu-skalar}
\end{equation}

Damit ergibt sich f"ur das Gesamtdrehmoment $ M $
%
\begin{equation}
M = \Int_{R}^{R+\Delta R}dr\;\Int_{0}^{2\pi}d\varphi\;r \mu_3
\end{equation}
%
und nach Einsetzen von \eqref{mu-skalar} mit der Substitution
\eqref{substitution}
%
\begin{eqnarray}
= -\Int_{R}^{R+\Delta R} dr & r^2 &
	\left[
	\Int_{1}^{-1}d\xi\; \frac{-\left(\frac{\omega r}{|\vRo|}+\xi\right)
				  \SubstCos^{-1^?}\; \fpolar{+}{r} }{ \vRiAbsDivvRoXI{r} }  +
	\right.\\ &&
	+\left.
	\Int_{-1}^{1}d\xi\; \frac{+\left(\frac{\omega r}{|\vRo|}+\xi\right)
				  \SubstCos^{-1^?}\; \fpolar{-}{r} }{ \vRiAbsDivvRoXI{r} }
	\right]
	\nonumber
\end{eqnarray}

Die Integranden kann man wieder zusammenfassen
%
\begin{equation}
M = -\Int_{R}^{R+\Delta R} dr\; r^2
    \Int_{1}^{-1}d\xi\;
	\frac{\left(\frac{\omega r}{|\vRo|}+\xi\right)   \SubstCos^{-1^?} }{ \vRiAbsDivvRoXI{r} }
	\left[ \fpolar{+}{r} + \fpolar{-}{r} \right]
    \eqlab{drehmoment}
\end{equation}

\subsection{Analytische Auswertung}

\subsubsection{Allgemeine Bemerkungen}

Aus Gleichung \eqref{kraft-x} folgt, da"z der Stein nur curlt, wenn eine
$ y $-Abh"angigkeit der Kraft vorhanden ist, d.h.\ die Reibungskraft mu"z auf der
Vorderseite des Steins (in Laufrichtung gesehen) verschieden von der auf der
R"uckseite sein. Dabei mu"z die Kraft auf der Vorderseite betragsm"a"zig kleiner
sein.
%%%% mr1998/07/13:
\footnote{Anm.: Meiner Meinung nach kommt \emph{nur} eine
Geschwindigkeitsabh"angigkeit in Frage, da alle Effekte lokal und
isoliert an den einzelnen Pebble wirken. Aus der Kreisbewegung +
L"angsbewegung ergibt sich rein geometrisch die geforderte
Reibungsverteilung.}
Diese Abh"angigkeit hat nichts mit der Geschwindigkeitsabh"angigkeit zu
tun. Aus einer reinen Geschwindigkeitsabh"angigkeit ergibt sich kein Curlen des
Steins, egal, welche genaue Form diese Geschwindigkeitsabh"angigkeit hat.

Die physikalische Ursache f"ur diese Ortsabh"angigkeit ist die Verdr"angung des
Wasserfilms durch den laufenden Stein, d.h.\ die Vorderseite gleitet auf einem
etwas dickeren Wasserfilm als die R"uckseite und hat daher eine niedrigere
Reibung (gleicher Effekt wie bei der Reibungsverminderung durch das
Wischen).%
%%%% mr1998/07/13:
\footnote{Anm.: Auch dies trifft nicht zu, da der Stein keinen
Wasserfilm ben"otigt um darauf zu gleiten. Das Eis selbst ist schon
'glatt'. Man beachte die Forschungsergebnisse von Gabor Somorjai.}

Beim Shuffleboard, bei dem die Scheiben ja in der "`falschen"' Richtung curlen,
wird die $ y $-Abh"angigkeit vermutlich durch kleine Schmutzpartikel auf der Bahn
verursacht, daher hat die R"uckseite, bei der weniger Schmutzpartikel vorhanden
sind, \emph{mehr}%
%%%% mr1998/07/13:
\footnote{Anm.: \emph{weniger}}
Reibung. Damit ergibt sich das Curlen in die andere Richtung. Dies best"atigen
auch Versuche mit einem Curlingstein auf einem Linoleumbelag. 

Des weiteren mu"z die $ y $-Abh"angigkeit der Reibungskraft $ \omega $-abh"angig
sein, da bei einem sehr schnell drehenden Stein der Unterschied zwischen Vorder-
und R"uckseite nicht mehr so ausgepr"agt sein sollte.

F"ur die Tatsache, da"z der Stein curlt ist die Geschwindigkeitsabh"angigkeit
ohne Bedeutung, sie hat nur einen Einflu"z auf den Betrag des Curlens.

\subsubsection{N"aherungsweise L"osung}

\begin{enumerate}
\item Schmaler Laufring: $ \Delta R \ll R $
%
\begin{equation}
r = (R + d R) = R(1+\epsilon), \quad \epsilon\ll 1
\end{equation}

Man n"ahere innerhalb der Integrale "uber $ \xi $ alle $ r $ durch $ R $ an, und
ber"ucksichtige beim Integral "uber $ r $ nur das Glied nullter % erster %%%% mr1998/04/03:
Ordnung in $ \epsilon $.

dann folgt aus \eqref{kraft-x}
%
\begin{equation}
F_x = -\frac{\omega R}{|\vRo|}
\Int_{-1}^{1} d\xi\;\frac{R\Delta R\left(\fpolar{-}{R}-\fpolar{+}{R}\right)}%
			 { \vRiAbsDivvRoXI{R}  }
\eqlab{kraft-x-R}
\end{equation}
%
und aus \eqref{kraft-y}
%
\begin{equation}
F_y =  -\Int_{-1}^{1}d\xi\;
    \frac{ \left( 1 + \frac{\omega R}{|\vRo|}\xi \right)
             \SubstCos^{-1^?}
         }{ \vRiAbsDivvRoXI{R}  }
    R \Delta R
    \left( \fpolar{+}{R} + \fpolar{-}{R} \right)
    \eqlab{kraft-y-R}
\end{equation}
%
das Gesamtdrehmoment ergibt sich
%
\begin{equation}
\eqlab{GesamtDrehmoment}
M = -\Int_{1}^{-1}d\xi\;
    \frac{ R \left( \frac{\omega R}{|\vRo|} + \xi \right)
             \SubstCos^{-1^?}
         }{ \vRiAbsDivvRoXI{R}  }
	R \Delta R
	\left( \fpolar{+}{R} + \fpolar{-}{R} \right)
    \eqlab{drehmoment-R}
\end{equation}

\item Einfachster Ansatz f"ur f:

Da nur eine $ y $-Abh"angigkeit der Kraft interessiert, wird f"ur diese als
einfachster Ansatz eine lineare Beziehung angesetzt.
%
\begin{equation}
f(|\vRi|,x,y,\omega) = f_0\left(1-\alpha\frac{y}{R}\right),
\quad \textrm{mit}\quad \alpha > 0.
\eqlab{kraft-linear}
\end{equation}

Bei diesem Ansatz ist die $ \omega $-Abh"angigkeit von $ f $ vernachl"assigt, ebenso
die Geschwindigkeitsabh"angigkeit. Die $ y $-Abh"angigkeit wird durch eine lineare
Beziehung gen"ahert.

Als willk"urliche Kraftkonstante wird nun $ K $ eingef"uhrt mit
%
\begin{displaymath}
K = \pi R \Delta R f_0, \quad\textrm{dann gilt:}
\end{displaymath}

%%%% mr1998/04/03: F"ur den Nenner von \eqref{kraft-x-R} mit \eqref{kraft-linear}
F"ur den Z"ahler von \eqref{kraft-x-R} mit \eqref{GesamtDrehmoment}
%
\begin{eqnarray}
R \Delta R \left(\fpolar{-}{R} - \fpolar{+}{R} \right) &=&
\frac{2}{\pi} \alpha K  \SubstCos 		   \\
R \Delta R \left(\fpolar{+}{R} + \fpolar{-}{R} \right) &=&
\frac{2}{\pi} K
\end{eqnarray}
%
und damit wegen
%
\begin{equation}
\Int_{-1}^{1} d\xi\; \SubstCos  = \frac{\pi}{2}
\end{equation}
%
\begin{eqnarray}
F_x &=& - \frac{\omega R}{|\vRo|} \alpha K \eqlab{kraft-x-N}\\
F_y &=& - K				\eqlab{kraft-y-N}\\
M   &=& - \frac{\omega R}{|\vRo|} R K	\eqlab{drehmoment-N}
\end{eqnarray}

\end{enumerate}

In dieser N"aherung wurde noch verwendet, da"z $ \frac{\omega R}{|\vRi|}\ll 1 $, eine
Annahme, die bei einer der Realit"at entsprechenden Winkelgeschwindigkeit von
ca.\ $ 1\frac{1}{ \textrm{s} } $ richtig ist, bis kurz vor dem Stillstand des Steins.

\subsubsection{Auswertung der N"aherungsl"osung}

Die Gleichungen \eqref{kraft-x-N} bis \eqref{drehmoment-N} beschreiben das grobe
Verhalten des Steins in dem Bereich, in dem die Bahngeschwindigkeit gro"z ist
gegen die Geschwindigkeit eines Punktes auf dem Laufring, solage der Stein sich
nicht zu schnell dreht.

In dieser N"aherung ist die bestimmende Gr"o"ze die den Stein abbremsende
Reibungskraft $ K $. Die senkrecht auf die Bewegungsrichtung wirkende Kraft, die
das Curlen verursacht, ist demgegen"uber verringert. Dabei gibt der Parameter
$ \alpha $ die $ y $-Abh"angigkeit der Kraft an.

Auch das Drehmoment ist proportional zur Kraft $ K $.

\subsection{Numerische L"osung}

Verwendet wird das Koordinatensystem $ S' $ zu $ S $ dazu. In diesem System gibt es
folgende Einheitsvektoren in b.z.w.\ senkrecht zur Laufrichtung des Steins:
\figref{pictureA} %%%% mr1998/07/13: new
%
\iftrue
\begin{equation}
\begin{array}{rcl}
\vec{s}_\parallel &=& \frac{1}{\sqrt{ \dot{p}^2 + \dot{q}^2 }}
    \left( \begin{array}{@{}c@{}}\dot{p} \\[\jot] \dot{q} \end{array} \right) \\[3\jot]
%
\vec{s}_\perp &=& \frac{1}{\sqrt{ \dot{p}^2 + \dot{q}^2 }}
    \left( \begin{array}{@{}r@{}} \dot{q} \\ -\dot{p} \end{array}\right)
\end{array}
\end{equation}
\else
\begin{eqnarray}
\vec{s}_\parallel &=& \frac{ 1 }{ \sqrt{ \dot{p}^2 + \dot{q}^2 } }
    \left(\begin{array}{@{}r@{}} \dot{p} \\[\jot] \dot{q} \end{array} \right) \\
%
\vec{s}_\perp &=& \frac{ 1 }{ \sqrt{ \dot{p}^2 + \dot{q}^2 } }
    \left(\begin{array}{@{}r@{}} \dot{q} \\[\jot] -\dot{p} \end{array} \right)
\end{eqnarray}
\fi

Das Vorzeichen von $ \vec{s}_\perp $ ergibt sich aus der Forderung, da"z $ S $ ein
Rechtssystem sein mu"z.

\subsubsection{Differentialgleichungen}

Mit $ F_x $, $ F_y $ und $ M $ aus Kapitel (\ref{herleitung}) ergeben sich
damit folgende Differentialgleichungen f"ur die Bahnkurve
%
\begin{eqnarray}
\textrm{mit}\quad K &=& \frac{F}{m}\vec{e}	\\
%%%% mr1998/04/03:
\ddot{\vec{x}} &=& \frac{ 1 }{ m }
    \left( \begin{array}{@{}rr@{}}
         F_y & F_x \\[\jot]
        -F_x & F_y
    \end{array} \right)
    \frac{ \dot{\vec{x}} }{ |\dot{\vec{x}}| } \NONUMBER\\
%
\ddot{p} &=& \frac{F_y}{m \sqrt{\dot{p}^2+\dot{q}^2}} \dot{p} +
	     \frac{F_x}{m \sqrt{\dot{p}^2+\dot{q}^2}} \dot{q}
	\eqlab{diff-p}\\
\ddot{q} &=& \frac{F_y}{m \sqrt{\dot{p}^2+\dot{q}^2}} \dot{q} -
	     \frac{F_x}{m \sqrt{\dot{p}^2+\dot{q}^2}} \dot{p}
	\eqlab{diff-q}\\
\dot{\omega} &=& \frac{M}{\Theta}
	\eqlab{diff-w}
\end{eqnarray}

Dabei ist $ m $ die Masse des Steins, $ \Theta $ sein Tr"agheitsmoment.

\subsubsection{Numerik}

Mit dem Eulerschen Verfahren ergeben sich folgende Gleichungen f"ur die
Integration von \eqref{diff-p} bis \eqref{diff-w} mit dem einfachen Ansatz
%
\begin{eqnarray}
%%%% mr1998/04/03:
\ddot{\vec{x}}^{(n)}  &=& \frac{ \dot{\vec{x}}^{(n+1)} - \dot{\vec{x}}^{(n)}
                               }{ \Delta t } \NONUMBER\\
\dot{\vec{x}}^{(n+1)} &=& \Delta t \cdot \ddot{\vec{x}}^{(n)} + \dot{\vec{x}}^{(n)} \NONUMBER\\
     \vec{x} ^{(n+1)} &=& \Delta t \cdot  \dot{\vec{x}}^{(n)} +      \vec{x} ^{(n)} \NONUMBER
%
\end{eqnarray}
oder komponentenweise ausgeschrieben
\begin{eqnarray}
\ddot{p}^{(n)}	&=& \frac{\dot{p}^{(n+1)} - \dot{p}^{(n)}}{\Delta t}
	\eqlab{euler-1}\\
\dot{p}^{(n+1)} &=& \frac{\Delta t}%
		    {m \sqrt{ \dot{p}^{(n)^2} +  \dot{q}^{(n)^2} }}
		    \left(F_y \dot{p}^{(n)} + F_x \dot{q}^{(n)} \right) +
		    \dot{p}^{(n)}
	\eqlab{euler-2}\\
\dot{q}^{(n+1)} &=& \frac{\Delta t}%
		    {m \sqrt{ \dot{p}^{(n)^2} +  \dot{q}^{(n)^2} }}
		    \left(F_y \dot{q}^{(n)} - F_x \dot{p}^{(n)} \right) +
		    \dot{q}^{(n)}
	\eqlab{euler-3}\\
p^{(n+1)} &=& \Delta t\; \dot{p}^{(n)} + p^{(n)}
	\eqlab{euler-4}\\
q^{(n+1)} &=& \Delta t\; \dot{q}^{(n)} + q^{(n)}
	\eqlab{euler-5}\\
\omega^{(n+1)} &=& \frac{\Delta t}{\Theta} M + \omega^{(n)}
	\eqlab{euler-6}
\end{eqnarray}

Dabei bezeichnen der Index $ {}^{(n+1)} $ den Zeitpunkt $ (n+1) \cdot \Delta t $,
der Index $ {}^{(n)} $ den Zeitpunkt $ n \cdot \Delta t $.

Die Anfangswerte f"ur $ n=0 $ m"ussen gegeben sein. $ F_x $, $ F_y $, $ M $
m"ussen zu jedem Zeitpunkt als Funktionen der Geschwindigkeit und der
Winkelgeschwindigkeit berechnet werden.

\subsubsection{Modell f"ur die Reibungskraft}

Die in Kapitel (6.1) gemessene geschwindigkeitsabh"angige Reibung wird mit der
folgenden Funktion angen"ahert.

\begin{figure}[htb]
\begin{center}%
\iffalse
History:
    %%%% mr1998/07/13: - Einige inhaltliche Fehler beseitigt,
                         Formeln minimal klarer formuliert.
    %%%% mr1998/04/03: - Einige inhaltliche Fehler beseitigt,
                         und fragliche Stellen mit ^? markiert
                       - einige Gleichungen hinzugef"ugt (mit \NONUMBER)
                       - kleine "Anderungen an den Zeichnungen
                       - Umbenennungen:
                             Befehl: label/ref -> eqlab/eqref
                             v   -> v_R    Geschwindigkeit auf dem Laufring
                             v_0 -> v      Geschwindigkeit des Steins
                             v_min -> v_b  Parameter in der Reibungsfunktion
                             v_0   -> v_a  Parameter in der Reibungsfunktion

\fi
\documentclass[a4paper]{report}
\usepackage{latexsym}
\usepackage{german}
\usepackage[T1]{fontenc}
% \usepackage[cp850]{inputenc}
\usepackage{pictex}
\input{tumlogo}
%Inputfile Frame3cm
\pagestyle{empty}              % keine Kopf oder Fu"szeile
\columnsep      =  12.5mm      % Spaltenabstand 1.25 cm
\columnseprule  =   0.0mm      % keine Trennlinie zwischen den Spalten
\topmargin      =  -5.4mm      % 1 Inch wird automatisch addiert
\headheight     =   7.0mm      % H"ohe der Kopfzeile
\headsep        =   3.0mm      % Abstand Kopzeile - Text
\topskip        =   1.0ex      % Abstand Unterkante erste Zeile - Oberkante Textbereich
\textheight     = 230.0mm      % H"ohe Textbereich
\textwidth      = 150.0mm      % Breite Textbereich
%\footheight    =   7.0mm      % H"ohe Fu"szeile
\footskip       =  10.0mm      % Abstand Unterkante Fu"s - Unterkante Textbereich
\evensidemargin =   4.6mm      % 1 Inch wird automatisch addiert
\oddsidemargin  =   4.6mm      % 1 Inch wird automatisch addiert
\parindent      =   0.0mm      % Absatzeinr"uckung
\parskip        = 1ex plus .4ex minus .2ex       % Absatzabstand
\flushbottom                   % letzte Zeile im der Seite immer ganz unten


\pagestyle{plain}

\parindent=0mm
\parskip=3mm
\jot=1ex          % zus"atzlicher Zeilenabstand innerhalb eqnarray
\setcounter{secnumdepth}{2}
\setcounter{tocdepth}{3}

%%%%%%%%%%%%%%%%%%%%%%%%%%%%%%%%%%%%%%%%%%%%%%%%%%%%%%%%%%%
%%  Spezielle Befehle f"ur Labels und Referenzen
%%%%%%%%%%%%%%%%%%%%%%%%%%%%%%%%%%%%%%%%%%%%%%%%%%%%%%%%%%%
\newcommand{\eqlab}[1]{\label{eq:#1}}
\newcommand{\eqref}[1]{Gl.~(\ref{eq:#1})}

\newcommand{\figlab}[1]{\label{fig:#1}}
\newcommand{\figref}[1]{Fig.~(\ref{fig:#1})}

\newcommand{\tablab}[1]{\label{tab:#1}}
\newcommand{\tabref}[1]{Tab.~(\ref{tab:#1})}

\iftrue
% keep original numbering => don't number new equations
\newcommand{\NONUMBER}{\nonumber}
\else
% don't keep original numbering => number new equations
\newcommand{\NONUMBER}{}
\fi
%%%%%%%%%%%%%%%%%%%%%%%%%%%%%%%%%%%%%%%%%%%%%%%%%%%%%%%%%%%
%%  Spezielle Befehle f"ur gro"se Ausdr"ucke:
%%%%%%%%%%%%%%%%%%%%%%%%%%%%%%%%%%%%%%%%%%%%%%%%%%%%%%%%%%%

\newcommand{\vRo}{ \vec{v}   }   % the rock's center's speed
\newcommand{\vRi}{ \vec{v}_R }   % the rock's running-ring's speed
\newcommand{\vRiAbsDivvRoCOS}[1]{\sqrt{\displaystyle
    1 + 2\frac{\omega #1}{|\vRo|}\cos\varphi +
    \left( \frac{\omega #1}{|\vRo|} \right)^2 } }
\newcommand{\vRiAbsDivvRoXI}[1]{\sqrt{\displaystyle
    1 + 2\frac{\omega #1}{|\vRo|}\xi +
    \left( \frac{\omega #1}{|\vRo|} \right)^2 } }
\newcommand{\SubstCos}{ \sqrt{1-\xi^2} }
\newcommand{\fpolar}[2]{ f( |\vRi| , #2 \xi, #1 #2 \SubstCos ,\omega) }
\newcommand{\Int}{\int\limits}

\unitlength=5mm
\setcoordinatesystem units <\unitlength,\unitlength>
\title{Curling \\ Diplomarbeit an der \oTUM{3ex} \\ Auszug}
\author{Uli Sutor}
\date{1988}

\begin{document}
\maketitle
\tableofcontents
\listoffigures

\setcounter{chapter}{8}
\chapter{Theorie}

\begin{figure}[htb]
\begin{center}\mbox{%
  %
  % PicTeX-Bild:
  %
  % Bild 80 aus Uli Sutors Diplomarbeit:
  % Der Stein auf dem Rink, Koordinatensysteme
  %
  \beginpicture
  \setplotarea x from -5 to 5, y from -5 to -10
  % The rink-system:
%  \iffinalplot
  \circulararc 360 degrees from 0.25 0 center at 0 0	  % Button
  \circulararc 360 degrees from 1.0  0 center at 0 0	  % 4foot
  \circulararc 360 degrees from 2.0  0 center at 0 0	  % 6foot
  \circulararc 360 degrees from 3.0  0 center at 0 0	  % 8foot
%  \fi
  \put {{\huge S'}}         [tl] at -5   10               % coordsys: title
%  \put {O'}                 [tl] at  0.2 -0.2             % Zero
  \put {\vector(1,0){10}}   [Bl] at -5	  0		  % x axis = Tee-line
  \put {p}		    [l]  at  5	  0		  % x axis label
  \put {\vector(0,1){15}}   [Bl] at  0	 -5		  % y axis = Centerline
  \put {q}		    [b]  at  0	 10		  % y axis label
  %
  % the rock-system:
  %
%  \iffinalplot
  \circulararc 360 degrees from 2.5 5 center at 2 5	  % Rock
%  \fi
  \put {{\huge S}}	    [tl] at  4	  9		  % coordsys: title
%   \put {O}		    [tl] at  2.4  4.6             % Zero
  \put {\vector(4,-1){1.6}} [Bl] at  2	  5		  % x axis
  \put {\line(-4,1){1}}     [Bl] at  2	  5		  % neg. x axis
  \put {x}		    [l]  at  3.6  4.6		  % x axis label
  \put {\vector(1,4){0.8}}  [Bl] at  2	  5		  % y axis
  \put {\line(-1,-4){0.25}} [Bl] at  2	  5		  % neg. y axis
  \put {y}		    [b]  at  2.8  8.4		  % y axis label
  \put {\vector(1,4){0.4}}  [Bl] at  2	  5		  % speedvector
  \put {$\vRo$}	            [tr] at  1.9  6.5		  % speed label
  \endpicture
}\end{center}
\caption{Der Stein auf dem Eis-Rink\figlab{pictureA}}
\end{figure}

\begin{figure}[htb]
\begin{center}\mbox{%
  %
  % PicTeX-Bild:
  %
  % Bild 81 aus Uli Sutors Diplomarbeit:
  % Der Stein mit eigenem Koordinatensystem
  %
  \beginpicture
  \setplotarea x from -9 to 9, y from -7 to 9
  % The rink-system:
%  \iffinalplot
  \circulararc 360 degrees from 3.0 0 center at 0 0
  \circulararc 360 degrees from 3.5 0 center at 0 0
  \circulararc 360 degrees from 5.5 0 center at 0 0
%  \fi
  \put {{\huge S}}	    [tl] at -7	  8
%  \put {O}		    [tr] at -0.1 -0.1
  \put {\vector(1,0){18}}   [Bl] at -9	  0		  % x axis
  \put {x}		    [l]  at  9	  0
  \put {\vector(0,1){16}}   [Bl] at  0	 -7		  % y axis
  \put {y}		    [b] at   0    9
  %
  \put {\vector(0,1){7}}   [Bl] at  0	 0		  % v
  \put {$\vRo$}	           [cl] at  0.1  7
  %
  % Nun der ganze Kleinkram:
  %
  % omega:
%  \iffinalplot
  \circulararc -32 degrees from -2 0.5 center at 0 0
%  \fi
  \put {\vector(1,1){0.1}}  [Bl] at -1.35 1.55
  \put {$\vec{\omega}$}     [Br] at -1.8  1.1
  %
  % Delta R
  \put {\vector(1,-2){0.5}} [Bl] at -2.06 4.13
  \put {\vector(-1,2){0.5}} [Bl] at -0.84 1.68
  \put {$\Delta R$}	    [Br] at -2	  3
  %
  % \vec{r}
  \put {\vector(1,1){2.3}}  [Bl] at  0	  0
  \put {$\vec{r}$}	    [Br] at  1	  1
  %
  % \phi
  \circulararc 45 degrees from 2 0 center at 0 0
  \put {$\varphi$}	    [Bl] at  2.1  0.7
  %
  % R
  \put {\vector(3,-1){2.85}}[Bl] at  0	  0
  \put {R}		    [tl] at  1.5 -0.75
  \endpicture
}\end{center}
\caption{Der Laufring des Steins}
\end{figure}

\section{Das Curlen}

\begin{figure}[htb]
\begin{center}\mbox{%
  %
  % PicTeX-Bild:
  %
  % Die Bahnkurve eines Steins
  %
  \beginpicture
  %\unitlength=5mm
  %\setcoordinatesystem units <\unitlength,\unitlength>
  \setplotarea x from -5 to 5, y from -13 to 5
%  \iffinalplot
  \circulararc 360 degrees from 0 0.5 center at 0 0
  \circulararc 360 degrees from 0 1   center at 0 0
  \circulararc 360 degrees from 0 2   center at 0 0
  \circulararc 360 degrees from 0 3   center at 0 0
%  \fi
  \setquadratic
  \put {\line(1,0){9}}	  [Bl] at -4.5	 3    % Back - line
  \put {\line(1,0){9}}	  [Bl] at -4.5	 0    % Tee - line
  \put {\line(1,0){9}}	  [Bl] at -4.5 -11.5  % Hog - line
  \put {\line(0,1){15.5}} [Bl] at  0   -12.5  % Centre - line
  %
  \plot 0.9  -11   1.5	-3   3	 3.5 /
  \put {\vector(1,2){0.1}} [Bl] at 3.1 3.6
%  \iffinalplot
  \circulararc 360 degrees from 3.3 3.8 center at 3.3 4.3
%  \fi
  %
  \setdashes <1mm>
  \plot 0.45 -11   0.75 -3   1.5 4 /
  \put {\vector(1,3){0.1}} [Bl] at 1.6 4.3
  \endpicture
}\end{center}
\caption{Typische Bahnkurve eines Curlingsteins}
\end{figure}

\subsection{Herleitung der Gleichungen\label{herleitung}}

Geschwindigkeit eines Punktes des Laufrings

\begin{tabular}{@{}ll}
$ \vRi $	  & Geschwindigkeit eines Punktes auf dem Laufring \\
$ \vRo $	  & Geschwindigkeit des Steins \\
$ \vec{\omega} $  & Winkelgeschwindigkeit des Steins
\end{tabular}

\begin{eqnarray}
\vRi &=& \vRo + \vec{\omega} \times \vec{r} =
{-\omega r \sin\varphi \choose |\vRo| + \omega r \cos\varphi}
    \eqlab{speed-vektor}\\
|\vRi | &=& |\vRo| \vRiAbsDivvRoCOS{r}
    \eqlab{speed-skalar}
\end{eqnarray}

F"ur die geschwindigkeitsabh"angige Kraft gilt

\begin{tabular}{@{}ll}
$ \vec{f} $     & Reibungskraft pro Fl"acheneinheit des Laufrings
\end{tabular}

\begin{equation}
\vec{f} = - \frac{\vRi}{|\vRi|} f \eqlab{reibung-vektor}
\end{equation}

dabei wird f"ur $ f $ die allgemeinste Form angenommen
%
\begin{equation}
f = f(|\vRi|,x,y,\omega)    \eqlab{reibung-skalar}
\end{equation}
%
F"ur die Umwandlung in Polarkoordinaten gilt
%
\begin{eqnarray}
x &=& r \cos\varphi \\
y &=& r \sin\varphi
\end{eqnarray}
%
damit ergibt sich f"ur \eqref{reibung-skalar}
%
\begin{equation}
f = f(|\vRi|,r\cos\varphi,r\sin\varphi,\omega) \eqlab{reibung-polar}
\end{equation}

\subsubsection{Kraft senkrecht auf Laufrichtung (bewirkt curlen)}

Wenn man in \eqref{reibung-vektor} nur die $ x $-Komponente von
\eqref{speed-vektor} einsetzt und sowohl "uber den ganzen Laufring, als auch
"uber die Breite des Rings integriert, erh"alt man

\begin{tabular}{@{}ll}
$ \Delta R $	  & Breite des Laufrings \\
$ R $		  & Radius des Laufrings
\end{tabular}

\begin{equation}
F_x = - \Int_{R}^{R+\Delta R} dr \Int_{0}^{2\pi} d\varphi\; r
\frac{-( \omega r \sin\varphi ) f}{|\vRo| \vRiAbsDivvRoCOS{r} }
\end{equation}
%
mit der Substitution
%
\begin{equation}
\xi = \cos\varphi, \quad \textrm{dann}\quad \frac{d\xi}{d\varphi} = -\sin\varphi =
\left\{ {- \SubstCos , \; 0	\leq\varphi < \pi \atop
	 + \SubstCos , \; \pi\leq\varphi < 2\pi }\right.
	  \eqlab{substitution}
\end{equation}
%
folgt mit \eqref{reibung-polar}
%
\begin{equation}
F_x = -\frac{\omega}{|\vRo|} \Int_{R}^{R+\Delta R} dr \; r^2 \left[
\Int_{1}^{-1} d\xi\;\frac{\fpolar{+}{r}}{ \vRiAbsDivvRoXI{r} } +
\Int_{-1}^{1} d\xi\;\frac{\fpolar{-}{r}}{ \vRiAbsDivvRoXI{r} } \right]
\end{equation}
%
und damit
%
\begin{equation}
F_x = -\frac{\omega}{|\vRo|} \Int_{R}^{R+\Delta R} dr\;r^2
\Int_{-1}^{1} d\xi\;\frac{\fpolar{-}{r}-\fpolar{+}{r}}{ \vRiAbsDivvRoXI{r} }
\eqlab{kraft-x}
\end{equation}

Da die Summe im Nenner von \eqref{kraft-x} nur einen Unterschied in der
$ y $-Komponente der Kraft aufweist, ergibt sich nur dann ein von Null
verschiedener Wert, wenn die Kraft eine $ y $-Abh"angigkeit hat.

\subsubsection{Kraft parallel Laufrichtung (bremst Stein)}

Wenn man in \eqref{reibung-vektor} nur die $ y $-Komponente von
\eqref{speed-vektor} einsetzt, ergibt sich
%
\begin{eqnarray}
F_y &=& -\Int_{R}^{R+\Delta R} dr \; \Int_{0}^{2\pi} d\varphi\; % r %%%% mr1998/04/03: r removed!
\frac{ (|\vRo|+\omega r\cos\varphi) f }{ |\vRo| \vRiAbsDivvRoCOS{r}  } =		      \\
&=& -\Int_{R}^{R+\Delta R}dr\;\left[
    \Int_{1}^{-1}d\xi\; (-  \SubstCos^{-1^?})
        \frac{\left(1+\frac{\omega r}{|\vRo|}\xi\right)
             }{ \vRiAbsDivvRoXI{r} } \fpolar{+}{r} +
    \right.
\nonumber\\&&
    +\left.\Int_{-1}^{1}d\xi\; (+  \SubstCos^{-1^?})
        \frac{ \left( 1+\frac{\omega r}{|\vRo|}\xi \right)
             }{ \vRiAbsDivvRoXI{r}  } \fpolar{-}{r}
        \right]
\end{eqnarray}

Die beiden Integranden kann man wieder zusammenfassen
%
\begin{equation}
F_y =  -\Int_{R}^{R+\Delta R} dr \; \Int_{-1}^{1} d\xi \;
%%%% mr1998/04/03: Changed: \frac{ \left( 1 + \frac{\omega r}{|\vRo|}\xi \right)  \SubstCos
    \frac{ \left( 1 + \frac{\omega r}{|\vRo|}\xi \right)   \SubstCos^{-1^?}
         }{ \vRiAbsDivvRoXI{r}  }
    \left[ \fpolar{+}{r} + \fpolar{-}{r} \right]
    \eqlab{kraft-y}
\end{equation}

\subsubsection{Drehmoment (bremst Drehung des Steins)}

F"ur das Drehmoment gilt mit \eqref{reibung-vektor}

\begin{tabular}{@{}ll}
$ \mu $   & Drehmoment auf Fl"achenelement des Laufrings
\end{tabular}

\begin{equation}
\vec{\mu} = \vec{r}\times \vec{f} = -\frac{f}{|\vRi|}\vec{r}\times\vec{v}_R
	\eqlab{dreh-vektor}
\end{equation}

Umgerechnet in Polarkoordinaten folgt in Vektorschreibweise f"ur das
Vektorprodukt
%
\begin{eqnarray}
\vec{r}\times\vec{v}_R &=& \left(\begin{array}{@{}c@{}}
	r\cos\varphi \\ r\sin\varphi \\ 0
\end{array}\right) \times
\left(\begin{array}{@{}c@{}}
	-\omega r\sin\varphi \\ |\vRo| + \omega r\cos\varphi \\ 0
\end{array}\right) =
\left(\begin{array}{@{}c@{}}
	0 \\ 0 \\ r\cos\varphi(|\vRo|+\omega r\cos\varphi) +
                  \omega r^2\sin^2\varphi
\end{array}\right) \\
%%%% mr1998/07/13: new:
&=& \left( |\vRo| r \cos\varphi + \omega r^2 \right) \cdot
    \left(\begin{array}{@{}c@{}} 0 \\ 0 \\ 1 \end{array}\right) \nonumber
\end{eqnarray}

F"ur den Betrag des Drehmoments also
%
\begin{equation}
%%%% mr1998/07/13: _3 added
\mu_3 = -\frac{f}{|\vRi|}\cdot \left(\vec{r}\times\vec{v}_R\right)_3
      = -\frac{f}{|\vRi|}\cdot \left(|\vRo| r \cos\varphi + \omega r^2 \right)
\eqlab{dreh-skalar}\end{equation}
%
setzt man \eqref{dreh-skalar} in \eqref{dreh-vektor} ein, mit
\eqref{speed-skalar}
%
\begin{equation}
%%%% mr1998/07/13: _3 added
\mu_3 = -(|\vRo| r \cos\varphi + \omega r^2) \frac{f}{|\vRi|} =
- \frac{r(\cos\varphi + \frac{\omega r}{|\vRo|}) f }{ \vRiAbsDivvRoCOS{r} } \eqlab{mu-skalar}
\end{equation}

Damit ergibt sich f"ur das Gesamtdrehmoment $ M $
%
\begin{equation}
M = \Int_{R}^{R+\Delta R}dr\;\Int_{0}^{2\pi}d\varphi\;r \mu_3
\end{equation}
%
und nach Einsetzen von \eqref{mu-skalar} mit der Substitution
\eqref{substitution}
%
\begin{eqnarray}
= -\Int_{R}^{R+\Delta R} dr & r^2 &
	\left[
	\Int_{1}^{-1}d\xi\; \frac{-\left(\frac{\omega r}{|\vRo|}+\xi\right)
				  \SubstCos^{-1^?}\; \fpolar{+}{r} }{ \vRiAbsDivvRoXI{r} }  +
	\right.\\ &&
	+\left.
	\Int_{-1}^{1}d\xi\; \frac{+\left(\frac{\omega r}{|\vRo|}+\xi\right)
				  \SubstCos^{-1^?}\; \fpolar{-}{r} }{ \vRiAbsDivvRoXI{r} }
	\right]
	\nonumber
\end{eqnarray}

Die Integranden kann man wieder zusammenfassen
%
\begin{equation}
M = -\Int_{R}^{R+\Delta R} dr\; r^2
    \Int_{1}^{-1}d\xi\;
	\frac{\left(\frac{\omega r}{|\vRo|}+\xi\right)   \SubstCos^{-1^?} }{ \vRiAbsDivvRoXI{r} }
	\left[ \fpolar{+}{r} + \fpolar{-}{r} \right]
    \eqlab{drehmoment}
\end{equation}

\subsection{Analytische Auswertung}

\subsubsection{Allgemeine Bemerkungen}

Aus Gleichung \eqref{kraft-x} folgt, da"z der Stein nur curlt, wenn eine
$ y $-Abh"angigkeit der Kraft vorhanden ist, d.h.\ die Reibungskraft mu"z auf der
Vorderseite des Steins (in Laufrichtung gesehen) verschieden von der auf der
R"uckseite sein. Dabei mu"z die Kraft auf der Vorderseite betragsm"a"zig kleiner
sein.
%%%% mr1998/07/13:
\footnote{Anm.: Meiner Meinung nach kommt \emph{nur} eine
Geschwindigkeitsabh"angigkeit in Frage, da alle Effekte lokal und
isoliert an den einzelnen Pebble wirken. Aus der Kreisbewegung +
L"angsbewegung ergibt sich rein geometrisch die geforderte
Reibungsverteilung.}
Diese Abh"angigkeit hat nichts mit der Geschwindigkeitsabh"angigkeit zu
tun. Aus einer reinen Geschwindigkeitsabh"angigkeit ergibt sich kein Curlen des
Steins, egal, welche genaue Form diese Geschwindigkeitsabh"angigkeit hat.

Die physikalische Ursache f"ur diese Ortsabh"angigkeit ist die Verdr"angung des
Wasserfilms durch den laufenden Stein, d.h.\ die Vorderseite gleitet auf einem
etwas dickeren Wasserfilm als die R"uckseite und hat daher eine niedrigere
Reibung (gleicher Effekt wie bei der Reibungsverminderung durch das
Wischen).%
%%%% mr1998/07/13:
\footnote{Anm.: Auch dies trifft nicht zu, da der Stein keinen
Wasserfilm ben"otigt um darauf zu gleiten. Das Eis selbst ist schon
'glatt'. Man beachte die Forschungsergebnisse von Gabor Somorjai.}

Beim Shuffleboard, bei dem die Scheiben ja in der "`falschen"' Richtung curlen,
wird die $ y $-Abh"angigkeit vermutlich durch kleine Schmutzpartikel auf der Bahn
verursacht, daher hat die R"uckseite, bei der weniger Schmutzpartikel vorhanden
sind, \emph{mehr}%
%%%% mr1998/07/13:
\footnote{Anm.: \emph{weniger}}
Reibung. Damit ergibt sich das Curlen in die andere Richtung. Dies best"atigen
auch Versuche mit einem Curlingstein auf einem Linoleumbelag. 

Des weiteren mu"z die $ y $-Abh"angigkeit der Reibungskraft $ \omega $-abh"angig
sein, da bei einem sehr schnell drehenden Stein der Unterschied zwischen Vorder-
und R"uckseite nicht mehr so ausgepr"agt sein sollte.

F"ur die Tatsache, da"z der Stein curlt ist die Geschwindigkeitsabh"angigkeit
ohne Bedeutung, sie hat nur einen Einflu"z auf den Betrag des Curlens.

\subsubsection{N"aherungsweise L"osung}

\begin{enumerate}
\item Schmaler Laufring: $ \Delta R \ll R $
%
\begin{equation}
r = (R + d R) = R(1+\epsilon), \quad \epsilon\ll 1
\end{equation}

Man n"ahere innerhalb der Integrale "uber $ \xi $ alle $ r $ durch $ R $ an, und
ber"ucksichtige beim Integral "uber $ r $ nur das Glied nullter % erster %%%% mr1998/04/03:
Ordnung in $ \epsilon $.

dann folgt aus \eqref{kraft-x}
%
\begin{equation}
F_x = -\frac{\omega R}{|\vRo|}
\Int_{-1}^{1} d\xi\;\frac{R\Delta R\left(\fpolar{-}{R}-\fpolar{+}{R}\right)}%
			 { \vRiAbsDivvRoXI{R}  }
\eqlab{kraft-x-R}
\end{equation}
%
und aus \eqref{kraft-y}
%
\begin{equation}
F_y =  -\Int_{-1}^{1}d\xi\;
    \frac{ \left( 1 + \frac{\omega R}{|\vRo|}\xi \right)
             \SubstCos^{-1^?}
         }{ \vRiAbsDivvRoXI{R}  }
    R \Delta R
    \left( \fpolar{+}{R} + \fpolar{-}{R} \right)
    \eqlab{kraft-y-R}
\end{equation}
%
das Gesamtdrehmoment ergibt sich
%
\begin{equation}
\eqlab{GesamtDrehmoment}
M = -\Int_{1}^{-1}d\xi\;
    \frac{ R \left( \frac{\omega R}{|\vRo|} + \xi \right)
             \SubstCos^{-1^?}
         }{ \vRiAbsDivvRoXI{R}  }
	R \Delta R
	\left( \fpolar{+}{R} + \fpolar{-}{R} \right)
    \eqlab{drehmoment-R}
\end{equation}

\item Einfachster Ansatz f"ur f:

Da nur eine $ y $-Abh"angigkeit der Kraft interessiert, wird f"ur diese als
einfachster Ansatz eine lineare Beziehung angesetzt.
%
\begin{equation}
f(|\vRi|,x,y,\omega) = f_0\left(1-\alpha\frac{y}{R}\right),
\quad \textrm{mit}\quad \alpha > 0.
\eqlab{kraft-linear}
\end{equation}

Bei diesem Ansatz ist die $ \omega $-Abh"angigkeit von $ f $ vernachl"assigt, ebenso
die Geschwindigkeitsabh"angigkeit. Die $ y $-Abh"angigkeit wird durch eine lineare
Beziehung gen"ahert.

Als willk"urliche Kraftkonstante wird nun $ K $ eingef"uhrt mit
%
\begin{displaymath}
K = \pi R \Delta R f_0, \quad\textrm{dann gilt:}
\end{displaymath}

%%%% mr1998/04/03: F"ur den Nenner von \eqref{kraft-x-R} mit \eqref{kraft-linear}
F"ur den Z"ahler von \eqref{kraft-x-R} mit \eqref{GesamtDrehmoment}
%
\begin{eqnarray}
R \Delta R \left(\fpolar{-}{R} - \fpolar{+}{R} \right) &=&
\frac{2}{\pi} \alpha K  \SubstCos 		   \\
R \Delta R \left(\fpolar{+}{R} + \fpolar{-}{R} \right) &=&
\frac{2}{\pi} K
\end{eqnarray}
%
und damit wegen
%
\begin{equation}
\Int_{-1}^{1} d\xi\; \SubstCos  = \frac{\pi}{2}
\end{equation}
%
\begin{eqnarray}
F_x &=& - \frac{\omega R}{|\vRo|} \alpha K \eqlab{kraft-x-N}\\
F_y &=& - K				\eqlab{kraft-y-N}\\
M   &=& - \frac{\omega R}{|\vRo|} R K	\eqlab{drehmoment-N}
\end{eqnarray}

\end{enumerate}

In dieser N"aherung wurde noch verwendet, da"z $ \frac{\omega R}{|\vRi|}\ll 1 $, eine
Annahme, die bei einer der Realit"at entsprechenden Winkelgeschwindigkeit von
ca.\ $ 1\frac{1}{ \textrm{s} } $ richtig ist, bis kurz vor dem Stillstand des Steins.

\subsubsection{Auswertung der N"aherungsl"osung}

Die Gleichungen \eqref{kraft-x-N} bis \eqref{drehmoment-N} beschreiben das grobe
Verhalten des Steins in dem Bereich, in dem die Bahngeschwindigkeit gro"z ist
gegen die Geschwindigkeit eines Punktes auf dem Laufring, solage der Stein sich
nicht zu schnell dreht.

In dieser N"aherung ist die bestimmende Gr"o"ze die den Stein abbremsende
Reibungskraft $ K $. Die senkrecht auf die Bewegungsrichtung wirkende Kraft, die
das Curlen verursacht, ist demgegen"uber verringert. Dabei gibt der Parameter
$ \alpha $ die $ y $-Abh"angigkeit der Kraft an.

Auch das Drehmoment ist proportional zur Kraft $ K $.

\subsection{Numerische L"osung}

Verwendet wird das Koordinatensystem $ S' $ zu $ S $ dazu. In diesem System gibt es
folgende Einheitsvektoren in b.z.w.\ senkrecht zur Laufrichtung des Steins:
\figref{pictureA} %%%% mr1998/07/13: new
%
\iftrue
\begin{equation}
\begin{array}{rcl}
\vec{s}_\parallel &=& \frac{1}{\sqrt{ \dot{p}^2 + \dot{q}^2 }}
    \left( \begin{array}{@{}c@{}}\dot{p} \\[\jot] \dot{q} \end{array} \right) \\[3\jot]
%
\vec{s}_\perp &=& \frac{1}{\sqrt{ \dot{p}^2 + \dot{q}^2 }}
    \left( \begin{array}{@{}r@{}} \dot{q} \\ -\dot{p} \end{array}\right)
\end{array}
\end{equation}
\else
\begin{eqnarray}
\vec{s}_\parallel &=& \frac{ 1 }{ \sqrt{ \dot{p}^2 + \dot{q}^2 } }
    \left(\begin{array}{@{}r@{}} \dot{p} \\[\jot] \dot{q} \end{array} \right) \\
%
\vec{s}_\perp &=& \frac{ 1 }{ \sqrt{ \dot{p}^2 + \dot{q}^2 } }
    \left(\begin{array}{@{}r@{}} \dot{q} \\[\jot] -\dot{p} \end{array} \right)
\end{eqnarray}
\fi

Das Vorzeichen von $ \vec{s}_\perp $ ergibt sich aus der Forderung, da"z $ S $ ein
Rechtssystem sein mu"z.

\subsubsection{Differentialgleichungen}

Mit $ F_x $, $ F_y $ und $ M $ aus Kapitel (\ref{herleitung}) ergeben sich
damit folgende Differentialgleichungen f"ur die Bahnkurve
%
\begin{eqnarray}
\textrm{mit}\quad K &=& \frac{F}{m}\vec{e}	\\
%%%% mr1998/04/03:
\ddot{\vec{x}} &=& \frac{ 1 }{ m }
    \left( \begin{array}{@{}rr@{}}
         F_y & F_x \\[\jot]
        -F_x & F_y
    \end{array} \right)
    \frac{ \dot{\vec{x}} }{ |\dot{\vec{x}}| } \NONUMBER\\
%
\ddot{p} &=& \frac{F_y}{m \sqrt{\dot{p}^2+\dot{q}^2}} \dot{p} +
	     \frac{F_x}{m \sqrt{\dot{p}^2+\dot{q}^2}} \dot{q}
	\eqlab{diff-p}\\
\ddot{q} &=& \frac{F_y}{m \sqrt{\dot{p}^2+\dot{q}^2}} \dot{q} -
	     \frac{F_x}{m \sqrt{\dot{p}^2+\dot{q}^2}} \dot{p}
	\eqlab{diff-q}\\
\dot{\omega} &=& \frac{M}{\Theta}
	\eqlab{diff-w}
\end{eqnarray}

Dabei ist $ m $ die Masse des Steins, $ \Theta $ sein Tr"agheitsmoment.

\subsubsection{Numerik}

Mit dem Eulerschen Verfahren ergeben sich folgende Gleichungen f"ur die
Integration von \eqref{diff-p} bis \eqref{diff-w} mit dem einfachen Ansatz
%
\begin{eqnarray}
%%%% mr1998/04/03:
\ddot{\vec{x}}^{(n)}  &=& \frac{ \dot{\vec{x}}^{(n+1)} - \dot{\vec{x}}^{(n)}
                               }{ \Delta t } \NONUMBER\\
\dot{\vec{x}}^{(n+1)} &=& \Delta t \cdot \ddot{\vec{x}}^{(n)} + \dot{\vec{x}}^{(n)} \NONUMBER\\
     \vec{x} ^{(n+1)} &=& \Delta t \cdot  \dot{\vec{x}}^{(n)} +      \vec{x} ^{(n)} \NONUMBER
%
\end{eqnarray}
oder komponentenweise ausgeschrieben
\begin{eqnarray}
\ddot{p}^{(n)}	&=& \frac{\dot{p}^{(n+1)} - \dot{p}^{(n)}}{\Delta t}
	\eqlab{euler-1}\\
\dot{p}^{(n+1)} &=& \frac{\Delta t}%
		    {m \sqrt{ \dot{p}^{(n)^2} +  \dot{q}^{(n)^2} }}
		    \left(F_y \dot{p}^{(n)} + F_x \dot{q}^{(n)} \right) +
		    \dot{p}^{(n)}
	\eqlab{euler-2}\\
\dot{q}^{(n+1)} &=& \frac{\Delta t}%
		    {m \sqrt{ \dot{p}^{(n)^2} +  \dot{q}^{(n)^2} }}
		    \left(F_y \dot{q}^{(n)} - F_x \dot{p}^{(n)} \right) +
		    \dot{q}^{(n)}
	\eqlab{euler-3}\\
p^{(n+1)} &=& \Delta t\; \dot{p}^{(n)} + p^{(n)}
	\eqlab{euler-4}\\
q^{(n+1)} &=& \Delta t\; \dot{q}^{(n)} + q^{(n)}
	\eqlab{euler-5}\\
\omega^{(n+1)} &=& \frac{\Delta t}{\Theta} M + \omega^{(n)}
	\eqlab{euler-6}
\end{eqnarray}

Dabei bezeichnen der Index $ {}^{(n+1)} $ den Zeitpunkt $ (n+1) \cdot \Delta t $,
der Index $ {}^{(n)} $ den Zeitpunkt $ n \cdot \Delta t $.

Die Anfangswerte f"ur $ n=0 $ m"ussen gegeben sein. $ F_x $, $ F_y $, $ M $
m"ussen zu jedem Zeitpunkt als Funktionen der Geschwindigkeit und der
Winkelgeschwindigkeit berechnet werden.

\subsubsection{Modell f"ur die Reibungskraft}

Die in Kapitel (6.1) gemessene geschwindigkeitsabh"angige Reibung wird mit der
folgenden Funktion angen"ahert.

\begin{figure}[htb]
\begin{center}%
\iffalse
History:
    %%%% mr1998/07/13: - Einige inhaltliche Fehler beseitigt,
                         Formeln minimal klarer formuliert.
    %%%% mr1998/04/03: - Einige inhaltliche Fehler beseitigt,
                         und fragliche Stellen mit ^? markiert
                       - einige Gleichungen hinzugef"ugt (mit \NONUMBER)
                       - kleine "Anderungen an den Zeichnungen
                       - Umbenennungen:
                             Befehl: label/ref -> eqlab/eqref
                             v   -> v_R    Geschwindigkeit auf dem Laufring
                             v_0 -> v      Geschwindigkeit des Steins
                             v_min -> v_b  Parameter in der Reibungsfunktion
                             v_0   -> v_a  Parameter in der Reibungsfunktion

\fi
\documentclass[a4paper]{report}
\usepackage{latexsym}
\usepackage{german}
\usepackage[T1]{fontenc}
% \usepackage[cp850]{inputenc}
\usepackage{pictex}
\input{tumlogo}
%Inputfile Frame3cm
\pagestyle{empty}              % keine Kopf oder Fu"szeile
\columnsep      =  12.5mm      % Spaltenabstand 1.25 cm
\columnseprule  =   0.0mm      % keine Trennlinie zwischen den Spalten
\topmargin      =  -5.4mm      % 1 Inch wird automatisch addiert
\headheight     =   7.0mm      % H"ohe der Kopfzeile
\headsep        =   3.0mm      % Abstand Kopzeile - Text
\topskip        =   1.0ex      % Abstand Unterkante erste Zeile - Oberkante Textbereich
\textheight     = 230.0mm      % H"ohe Textbereich
\textwidth      = 150.0mm      % Breite Textbereich
%\footheight    =   7.0mm      % H"ohe Fu"szeile
\footskip       =  10.0mm      % Abstand Unterkante Fu"s - Unterkante Textbereich
\evensidemargin =   4.6mm      % 1 Inch wird automatisch addiert
\oddsidemargin  =   4.6mm      % 1 Inch wird automatisch addiert
\parindent      =   0.0mm      % Absatzeinr"uckung
\parskip        = 1ex plus .4ex minus .2ex       % Absatzabstand
\flushbottom                   % letzte Zeile im der Seite immer ganz unten


\pagestyle{plain}

\parindent=0mm
\parskip=3mm
\jot=1ex          % zus"atzlicher Zeilenabstand innerhalb eqnarray
\setcounter{secnumdepth}{2}
\setcounter{tocdepth}{3}

%%%%%%%%%%%%%%%%%%%%%%%%%%%%%%%%%%%%%%%%%%%%%%%%%%%%%%%%%%%
%%  Spezielle Befehle f"ur Labels und Referenzen
%%%%%%%%%%%%%%%%%%%%%%%%%%%%%%%%%%%%%%%%%%%%%%%%%%%%%%%%%%%
\newcommand{\eqlab}[1]{\label{eq:#1}}
\newcommand{\eqref}[1]{Gl.~(\ref{eq:#1})}

\newcommand{\figlab}[1]{\label{fig:#1}}
\newcommand{\figref}[1]{Fig.~(\ref{fig:#1})}

\newcommand{\tablab}[1]{\label{tab:#1}}
\newcommand{\tabref}[1]{Tab.~(\ref{tab:#1})}

\iftrue
% keep original numbering => don't number new equations
\newcommand{\NONUMBER}{\nonumber}
\else
% don't keep original numbering => number new equations
\newcommand{\NONUMBER}{}
\fi
%%%%%%%%%%%%%%%%%%%%%%%%%%%%%%%%%%%%%%%%%%%%%%%%%%%%%%%%%%%
%%  Spezielle Befehle f"ur gro"se Ausdr"ucke:
%%%%%%%%%%%%%%%%%%%%%%%%%%%%%%%%%%%%%%%%%%%%%%%%%%%%%%%%%%%

\newcommand{\vRo}{ \vec{v}   }   % the rock's center's speed
\newcommand{\vRi}{ \vec{v}_R }   % the rock's running-ring's speed
\newcommand{\vRiAbsDivvRoCOS}[1]{\sqrt{\displaystyle
    1 + 2\frac{\omega #1}{|\vRo|}\cos\varphi +
    \left( \frac{\omega #1}{|\vRo|} \right)^2 } }
\newcommand{\vRiAbsDivvRoXI}[1]{\sqrt{\displaystyle
    1 + 2\frac{\omega #1}{|\vRo|}\xi +
    \left( \frac{\omega #1}{|\vRo|} \right)^2 } }
\newcommand{\SubstCos}{ \sqrt{1-\xi^2} }
\newcommand{\fpolar}[2]{ f( |\vRi| , #2 \xi, #1 #2 \SubstCos ,\omega) }
\newcommand{\Int}{\int\limits}

\unitlength=5mm
\setcoordinatesystem units <\unitlength,\unitlength>
\title{Curling \\ Diplomarbeit an der \oTUM{3ex} \\ Auszug}
\author{Uli Sutor}
\date{1988}

\begin{document}
\maketitle
\tableofcontents
\listoffigures

\setcounter{chapter}{8}
\chapter{Theorie}

\begin{figure}[htb]
\begin{center}\mbox{%
  %
  % PicTeX-Bild:
  %
  % Bild 80 aus Uli Sutors Diplomarbeit:
  % Der Stein auf dem Rink, Koordinatensysteme
  %
  \beginpicture
  \setplotarea x from -5 to 5, y from -5 to -10
  % The rink-system:
%  \iffinalplot
  \circulararc 360 degrees from 0.25 0 center at 0 0	  % Button
  \circulararc 360 degrees from 1.0  0 center at 0 0	  % 4foot
  \circulararc 360 degrees from 2.0  0 center at 0 0	  % 6foot
  \circulararc 360 degrees from 3.0  0 center at 0 0	  % 8foot
%  \fi
  \put {{\huge S'}}         [tl] at -5   10               % coordsys: title
%  \put {O'}                 [tl] at  0.2 -0.2             % Zero
  \put {\vector(1,0){10}}   [Bl] at -5	  0		  % x axis = Tee-line
  \put {p}		    [l]  at  5	  0		  % x axis label
  \put {\vector(0,1){15}}   [Bl] at  0	 -5		  % y axis = Centerline
  \put {q}		    [b]  at  0	 10		  % y axis label
  %
  % the rock-system:
  %
%  \iffinalplot
  \circulararc 360 degrees from 2.5 5 center at 2 5	  % Rock
%  \fi
  \put {{\huge S}}	    [tl] at  4	  9		  % coordsys: title
%   \put {O}		    [tl] at  2.4  4.6             % Zero
  \put {\vector(4,-1){1.6}} [Bl] at  2	  5		  % x axis
  \put {\line(-4,1){1}}     [Bl] at  2	  5		  % neg. x axis
  \put {x}		    [l]  at  3.6  4.6		  % x axis label
  \put {\vector(1,4){0.8}}  [Bl] at  2	  5		  % y axis
  \put {\line(-1,-4){0.25}} [Bl] at  2	  5		  % neg. y axis
  \put {y}		    [b]  at  2.8  8.4		  % y axis label
  \put {\vector(1,4){0.4}}  [Bl] at  2	  5		  % speedvector
  \put {$\vRo$}	            [tr] at  1.9  6.5		  % speed label
  \endpicture
}\end{center}
\caption{Der Stein auf dem Eis-Rink\figlab{pictureA}}
\end{figure}

\begin{figure}[htb]
\begin{center}\mbox{%
  %
  % PicTeX-Bild:
  %
  % Bild 81 aus Uli Sutors Diplomarbeit:
  % Der Stein mit eigenem Koordinatensystem
  %
  \beginpicture
  \setplotarea x from -9 to 9, y from -7 to 9
  % The rink-system:
%  \iffinalplot
  \circulararc 360 degrees from 3.0 0 center at 0 0
  \circulararc 360 degrees from 3.5 0 center at 0 0
  \circulararc 360 degrees from 5.5 0 center at 0 0
%  \fi
  \put {{\huge S}}	    [tl] at -7	  8
%  \put {O}		    [tr] at -0.1 -0.1
  \put {\vector(1,0){18}}   [Bl] at -9	  0		  % x axis
  \put {x}		    [l]  at  9	  0
  \put {\vector(0,1){16}}   [Bl] at  0	 -7		  % y axis
  \put {y}		    [b] at   0    9
  %
  \put {\vector(0,1){7}}   [Bl] at  0	 0		  % v
  \put {$\vRo$}	           [cl] at  0.1  7
  %
  % Nun der ganze Kleinkram:
  %
  % omega:
%  \iffinalplot
  \circulararc -32 degrees from -2 0.5 center at 0 0
%  \fi
  \put {\vector(1,1){0.1}}  [Bl] at -1.35 1.55
  \put {$\vec{\omega}$}     [Br] at -1.8  1.1
  %
  % Delta R
  \put {\vector(1,-2){0.5}} [Bl] at -2.06 4.13
  \put {\vector(-1,2){0.5}} [Bl] at -0.84 1.68
  \put {$\Delta R$}	    [Br] at -2	  3
  %
  % \vec{r}
  \put {\vector(1,1){2.3}}  [Bl] at  0	  0
  \put {$\vec{r}$}	    [Br] at  1	  1
  %
  % \phi
  \circulararc 45 degrees from 2 0 center at 0 0
  \put {$\varphi$}	    [Bl] at  2.1  0.7
  %
  % R
  \put {\vector(3,-1){2.85}}[Bl] at  0	  0
  \put {R}		    [tl] at  1.5 -0.75
  \endpicture
}\end{center}
\caption{Der Laufring des Steins}
\end{figure}

\section{Das Curlen}

\begin{figure}[htb]
\begin{center}\mbox{%
  %
  % PicTeX-Bild:
  %
  % Die Bahnkurve eines Steins
  %
  \beginpicture
  %\unitlength=5mm
  %\setcoordinatesystem units <\unitlength,\unitlength>
  \setplotarea x from -5 to 5, y from -13 to 5
%  \iffinalplot
  \circulararc 360 degrees from 0 0.5 center at 0 0
  \circulararc 360 degrees from 0 1   center at 0 0
  \circulararc 360 degrees from 0 2   center at 0 0
  \circulararc 360 degrees from 0 3   center at 0 0
%  \fi
  \setquadratic
  \put {\line(1,0){9}}	  [Bl] at -4.5	 3    % Back - line
  \put {\line(1,0){9}}	  [Bl] at -4.5	 0    % Tee - line
  \put {\line(1,0){9}}	  [Bl] at -4.5 -11.5  % Hog - line
  \put {\line(0,1){15.5}} [Bl] at  0   -12.5  % Centre - line
  %
  \plot 0.9  -11   1.5	-3   3	 3.5 /
  \put {\vector(1,2){0.1}} [Bl] at 3.1 3.6
%  \iffinalplot
  \circulararc 360 degrees from 3.3 3.8 center at 3.3 4.3
%  \fi
  %
  \setdashes <1mm>
  \plot 0.45 -11   0.75 -3   1.5 4 /
  \put {\vector(1,3){0.1}} [Bl] at 1.6 4.3
  \endpicture
}\end{center}
\caption{Typische Bahnkurve eines Curlingsteins}
\end{figure}

\subsection{Herleitung der Gleichungen\label{herleitung}}

Geschwindigkeit eines Punktes des Laufrings

\begin{tabular}{@{}ll}
$ \vRi $	  & Geschwindigkeit eines Punktes auf dem Laufring \\
$ \vRo $	  & Geschwindigkeit des Steins \\
$ \vec{\omega} $  & Winkelgeschwindigkeit des Steins
\end{tabular}

\begin{eqnarray}
\vRi &=& \vRo + \vec{\omega} \times \vec{r} =
{-\omega r \sin\varphi \choose |\vRo| + \omega r \cos\varphi}
    \eqlab{speed-vektor}\\
|\vRi | &=& |\vRo| \vRiAbsDivvRoCOS{r}
    \eqlab{speed-skalar}
\end{eqnarray}

F"ur die geschwindigkeitsabh"angige Kraft gilt

\begin{tabular}{@{}ll}
$ \vec{f} $     & Reibungskraft pro Fl"acheneinheit des Laufrings
\end{tabular}

\begin{equation}
\vec{f} = - \frac{\vRi}{|\vRi|} f \eqlab{reibung-vektor}
\end{equation}

dabei wird f"ur $ f $ die allgemeinste Form angenommen
%
\begin{equation}
f = f(|\vRi|,x,y,\omega)    \eqlab{reibung-skalar}
\end{equation}
%
F"ur die Umwandlung in Polarkoordinaten gilt
%
\begin{eqnarray}
x &=& r \cos\varphi \\
y &=& r \sin\varphi
\end{eqnarray}
%
damit ergibt sich f"ur \eqref{reibung-skalar}
%
\begin{equation}
f = f(|\vRi|,r\cos\varphi,r\sin\varphi,\omega) \eqlab{reibung-polar}
\end{equation}

\subsubsection{Kraft senkrecht auf Laufrichtung (bewirkt curlen)}

Wenn man in \eqref{reibung-vektor} nur die $ x $-Komponente von
\eqref{speed-vektor} einsetzt und sowohl "uber den ganzen Laufring, als auch
"uber die Breite des Rings integriert, erh"alt man

\begin{tabular}{@{}ll}
$ \Delta R $	  & Breite des Laufrings \\
$ R $		  & Radius des Laufrings
\end{tabular}

\begin{equation}
F_x = - \Int_{R}^{R+\Delta R} dr \Int_{0}^{2\pi} d\varphi\; r
\frac{-( \omega r \sin\varphi ) f}{|\vRo| \vRiAbsDivvRoCOS{r} }
\end{equation}
%
mit der Substitution
%
\begin{equation}
\xi = \cos\varphi, \quad \textrm{dann}\quad \frac{d\xi}{d\varphi} = -\sin\varphi =
\left\{ {- \SubstCos , \; 0	\leq\varphi < \pi \atop
	 + \SubstCos , \; \pi\leq\varphi < 2\pi }\right.
	  \eqlab{substitution}
\end{equation}
%
folgt mit \eqref{reibung-polar}
%
\begin{equation}
F_x = -\frac{\omega}{|\vRo|} \Int_{R}^{R+\Delta R} dr \; r^2 \left[
\Int_{1}^{-1} d\xi\;\frac{\fpolar{+}{r}}{ \vRiAbsDivvRoXI{r} } +
\Int_{-1}^{1} d\xi\;\frac{\fpolar{-}{r}}{ \vRiAbsDivvRoXI{r} } \right]
\end{equation}
%
und damit
%
\begin{equation}
F_x = -\frac{\omega}{|\vRo|} \Int_{R}^{R+\Delta R} dr\;r^2
\Int_{-1}^{1} d\xi\;\frac{\fpolar{-}{r}-\fpolar{+}{r}}{ \vRiAbsDivvRoXI{r} }
\eqlab{kraft-x}
\end{equation}

Da die Summe im Nenner von \eqref{kraft-x} nur einen Unterschied in der
$ y $-Komponente der Kraft aufweist, ergibt sich nur dann ein von Null
verschiedener Wert, wenn die Kraft eine $ y $-Abh"angigkeit hat.

\subsubsection{Kraft parallel Laufrichtung (bremst Stein)}

Wenn man in \eqref{reibung-vektor} nur die $ y $-Komponente von
\eqref{speed-vektor} einsetzt, ergibt sich
%
\begin{eqnarray}
F_y &=& -\Int_{R}^{R+\Delta R} dr \; \Int_{0}^{2\pi} d\varphi\; % r %%%% mr1998/04/03: r removed!
\frac{ (|\vRo|+\omega r\cos\varphi) f }{ |\vRo| \vRiAbsDivvRoCOS{r}  } =		      \\
&=& -\Int_{R}^{R+\Delta R}dr\;\left[
    \Int_{1}^{-1}d\xi\; (-  \SubstCos^{-1^?})
        \frac{\left(1+\frac{\omega r}{|\vRo|}\xi\right)
             }{ \vRiAbsDivvRoXI{r} } \fpolar{+}{r} +
    \right.
\nonumber\\&&
    +\left.\Int_{-1}^{1}d\xi\; (+  \SubstCos^{-1^?})
        \frac{ \left( 1+\frac{\omega r}{|\vRo|}\xi \right)
             }{ \vRiAbsDivvRoXI{r}  } \fpolar{-}{r}
        \right]
\end{eqnarray}

Die beiden Integranden kann man wieder zusammenfassen
%
\begin{equation}
F_y =  -\Int_{R}^{R+\Delta R} dr \; \Int_{-1}^{1} d\xi \;
%%%% mr1998/04/03: Changed: \frac{ \left( 1 + \frac{\omega r}{|\vRo|}\xi \right)  \SubstCos
    \frac{ \left( 1 + \frac{\omega r}{|\vRo|}\xi \right)   \SubstCos^{-1^?}
         }{ \vRiAbsDivvRoXI{r}  }
    \left[ \fpolar{+}{r} + \fpolar{-}{r} \right]
    \eqlab{kraft-y}
\end{equation}

\subsubsection{Drehmoment (bremst Drehung des Steins)}

F"ur das Drehmoment gilt mit \eqref{reibung-vektor}

\begin{tabular}{@{}ll}
$ \mu $   & Drehmoment auf Fl"achenelement des Laufrings
\end{tabular}

\begin{equation}
\vec{\mu} = \vec{r}\times \vec{f} = -\frac{f}{|\vRi|}\vec{r}\times\vec{v}_R
	\eqlab{dreh-vektor}
\end{equation}

Umgerechnet in Polarkoordinaten folgt in Vektorschreibweise f"ur das
Vektorprodukt
%
\begin{eqnarray}
\vec{r}\times\vec{v}_R &=& \left(\begin{array}{@{}c@{}}
	r\cos\varphi \\ r\sin\varphi \\ 0
\end{array}\right) \times
\left(\begin{array}{@{}c@{}}
	-\omega r\sin\varphi \\ |\vRo| + \omega r\cos\varphi \\ 0
\end{array}\right) =
\left(\begin{array}{@{}c@{}}
	0 \\ 0 \\ r\cos\varphi(|\vRo|+\omega r\cos\varphi) +
                  \omega r^2\sin^2\varphi
\end{array}\right) \\
%%%% mr1998/07/13: new:
&=& \left( |\vRo| r \cos\varphi + \omega r^2 \right) \cdot
    \left(\begin{array}{@{}c@{}} 0 \\ 0 \\ 1 \end{array}\right) \nonumber
\end{eqnarray}

F"ur den Betrag des Drehmoments also
%
\begin{equation}
%%%% mr1998/07/13: _3 added
\mu_3 = -\frac{f}{|\vRi|}\cdot \left(\vec{r}\times\vec{v}_R\right)_3
      = -\frac{f}{|\vRi|}\cdot \left(|\vRo| r \cos\varphi + \omega r^2 \right)
\eqlab{dreh-skalar}\end{equation}
%
setzt man \eqref{dreh-skalar} in \eqref{dreh-vektor} ein, mit
\eqref{speed-skalar}
%
\begin{equation}
%%%% mr1998/07/13: _3 added
\mu_3 = -(|\vRo| r \cos\varphi + \omega r^2) \frac{f}{|\vRi|} =
- \frac{r(\cos\varphi + \frac{\omega r}{|\vRo|}) f }{ \vRiAbsDivvRoCOS{r} } \eqlab{mu-skalar}
\end{equation}

Damit ergibt sich f"ur das Gesamtdrehmoment $ M $
%
\begin{equation}
M = \Int_{R}^{R+\Delta R}dr\;\Int_{0}^{2\pi}d\varphi\;r \mu_3
\end{equation}
%
und nach Einsetzen von \eqref{mu-skalar} mit der Substitution
\eqref{substitution}
%
\begin{eqnarray}
= -\Int_{R}^{R+\Delta R} dr & r^2 &
	\left[
	\Int_{1}^{-1}d\xi\; \frac{-\left(\frac{\omega r}{|\vRo|}+\xi\right)
				  \SubstCos^{-1^?}\; \fpolar{+}{r} }{ \vRiAbsDivvRoXI{r} }  +
	\right.\\ &&
	+\left.
	\Int_{-1}^{1}d\xi\; \frac{+\left(\frac{\omega r}{|\vRo|}+\xi\right)
				  \SubstCos^{-1^?}\; \fpolar{-}{r} }{ \vRiAbsDivvRoXI{r} }
	\right]
	\nonumber
\end{eqnarray}

Die Integranden kann man wieder zusammenfassen
%
\begin{equation}
M = -\Int_{R}^{R+\Delta R} dr\; r^2
    \Int_{1}^{-1}d\xi\;
	\frac{\left(\frac{\omega r}{|\vRo|}+\xi\right)   \SubstCos^{-1^?} }{ \vRiAbsDivvRoXI{r} }
	\left[ \fpolar{+}{r} + \fpolar{-}{r} \right]
    \eqlab{drehmoment}
\end{equation}

\subsection{Analytische Auswertung}

\subsubsection{Allgemeine Bemerkungen}

Aus Gleichung \eqref{kraft-x} folgt, da"z der Stein nur curlt, wenn eine
$ y $-Abh"angigkeit der Kraft vorhanden ist, d.h.\ die Reibungskraft mu"z auf der
Vorderseite des Steins (in Laufrichtung gesehen) verschieden von der auf der
R"uckseite sein. Dabei mu"z die Kraft auf der Vorderseite betragsm"a"zig kleiner
sein.
%%%% mr1998/07/13:
\footnote{Anm.: Meiner Meinung nach kommt \emph{nur} eine
Geschwindigkeitsabh"angigkeit in Frage, da alle Effekte lokal und
isoliert an den einzelnen Pebble wirken. Aus der Kreisbewegung +
L"angsbewegung ergibt sich rein geometrisch die geforderte
Reibungsverteilung.}
Diese Abh"angigkeit hat nichts mit der Geschwindigkeitsabh"angigkeit zu
tun. Aus einer reinen Geschwindigkeitsabh"angigkeit ergibt sich kein Curlen des
Steins, egal, welche genaue Form diese Geschwindigkeitsabh"angigkeit hat.

Die physikalische Ursache f"ur diese Ortsabh"angigkeit ist die Verdr"angung des
Wasserfilms durch den laufenden Stein, d.h.\ die Vorderseite gleitet auf einem
etwas dickeren Wasserfilm als die R"uckseite und hat daher eine niedrigere
Reibung (gleicher Effekt wie bei der Reibungsverminderung durch das
Wischen).%
%%%% mr1998/07/13:
\footnote{Anm.: Auch dies trifft nicht zu, da der Stein keinen
Wasserfilm ben"otigt um darauf zu gleiten. Das Eis selbst ist schon
'glatt'. Man beachte die Forschungsergebnisse von Gabor Somorjai.}

Beim Shuffleboard, bei dem die Scheiben ja in der "`falschen"' Richtung curlen,
wird die $ y $-Abh"angigkeit vermutlich durch kleine Schmutzpartikel auf der Bahn
verursacht, daher hat die R"uckseite, bei der weniger Schmutzpartikel vorhanden
sind, \emph{mehr}%
%%%% mr1998/07/13:
\footnote{Anm.: \emph{weniger}}
Reibung. Damit ergibt sich das Curlen in die andere Richtung. Dies best"atigen
auch Versuche mit einem Curlingstein auf einem Linoleumbelag. 

Des weiteren mu"z die $ y $-Abh"angigkeit der Reibungskraft $ \omega $-abh"angig
sein, da bei einem sehr schnell drehenden Stein der Unterschied zwischen Vorder-
und R"uckseite nicht mehr so ausgepr"agt sein sollte.

F"ur die Tatsache, da"z der Stein curlt ist die Geschwindigkeitsabh"angigkeit
ohne Bedeutung, sie hat nur einen Einflu"z auf den Betrag des Curlens.

\subsubsection{N"aherungsweise L"osung}

\begin{enumerate}
\item Schmaler Laufring: $ \Delta R \ll R $
%
\begin{equation}
r = (R + d R) = R(1+\epsilon), \quad \epsilon\ll 1
\end{equation}

Man n"ahere innerhalb der Integrale "uber $ \xi $ alle $ r $ durch $ R $ an, und
ber"ucksichtige beim Integral "uber $ r $ nur das Glied nullter % erster %%%% mr1998/04/03:
Ordnung in $ \epsilon $.

dann folgt aus \eqref{kraft-x}
%
\begin{equation}
F_x = -\frac{\omega R}{|\vRo|}
\Int_{-1}^{1} d\xi\;\frac{R\Delta R\left(\fpolar{-}{R}-\fpolar{+}{R}\right)}%
			 { \vRiAbsDivvRoXI{R}  }
\eqlab{kraft-x-R}
\end{equation}
%
und aus \eqref{kraft-y}
%
\begin{equation}
F_y =  -\Int_{-1}^{1}d\xi\;
    \frac{ \left( 1 + \frac{\omega R}{|\vRo|}\xi \right)
             \SubstCos^{-1^?}
         }{ \vRiAbsDivvRoXI{R}  }
    R \Delta R
    \left( \fpolar{+}{R} + \fpolar{-}{R} \right)
    \eqlab{kraft-y-R}
\end{equation}
%
das Gesamtdrehmoment ergibt sich
%
\begin{equation}
\eqlab{GesamtDrehmoment}
M = -\Int_{1}^{-1}d\xi\;
    \frac{ R \left( \frac{\omega R}{|\vRo|} + \xi \right)
             \SubstCos^{-1^?}
         }{ \vRiAbsDivvRoXI{R}  }
	R \Delta R
	\left( \fpolar{+}{R} + \fpolar{-}{R} \right)
    \eqlab{drehmoment-R}
\end{equation}

\item Einfachster Ansatz f"ur f:

Da nur eine $ y $-Abh"angigkeit der Kraft interessiert, wird f"ur diese als
einfachster Ansatz eine lineare Beziehung angesetzt.
%
\begin{equation}
f(|\vRi|,x,y,\omega) = f_0\left(1-\alpha\frac{y}{R}\right),
\quad \textrm{mit}\quad \alpha > 0.
\eqlab{kraft-linear}
\end{equation}

Bei diesem Ansatz ist die $ \omega $-Abh"angigkeit von $ f $ vernachl"assigt, ebenso
die Geschwindigkeitsabh"angigkeit. Die $ y $-Abh"angigkeit wird durch eine lineare
Beziehung gen"ahert.

Als willk"urliche Kraftkonstante wird nun $ K $ eingef"uhrt mit
%
\begin{displaymath}
K = \pi R \Delta R f_0, \quad\textrm{dann gilt:}
\end{displaymath}

%%%% mr1998/04/03: F"ur den Nenner von \eqref{kraft-x-R} mit \eqref{kraft-linear}
F"ur den Z"ahler von \eqref{kraft-x-R} mit \eqref{GesamtDrehmoment}
%
\begin{eqnarray}
R \Delta R \left(\fpolar{-}{R} - \fpolar{+}{R} \right) &=&
\frac{2}{\pi} \alpha K  \SubstCos 		   \\
R \Delta R \left(\fpolar{+}{R} + \fpolar{-}{R} \right) &=&
\frac{2}{\pi} K
\end{eqnarray}
%
und damit wegen
%
\begin{equation}
\Int_{-1}^{1} d\xi\; \SubstCos  = \frac{\pi}{2}
\end{equation}
%
\begin{eqnarray}
F_x &=& - \frac{\omega R}{|\vRo|} \alpha K \eqlab{kraft-x-N}\\
F_y &=& - K				\eqlab{kraft-y-N}\\
M   &=& - \frac{\omega R}{|\vRo|} R K	\eqlab{drehmoment-N}
\end{eqnarray}

\end{enumerate}

In dieser N"aherung wurde noch verwendet, da"z $ \frac{\omega R}{|\vRi|}\ll 1 $, eine
Annahme, die bei einer der Realit"at entsprechenden Winkelgeschwindigkeit von
ca.\ $ 1\frac{1}{ \textrm{s} } $ richtig ist, bis kurz vor dem Stillstand des Steins.

\subsubsection{Auswertung der N"aherungsl"osung}

Die Gleichungen \eqref{kraft-x-N} bis \eqref{drehmoment-N} beschreiben das grobe
Verhalten des Steins in dem Bereich, in dem die Bahngeschwindigkeit gro"z ist
gegen die Geschwindigkeit eines Punktes auf dem Laufring, solage der Stein sich
nicht zu schnell dreht.

In dieser N"aherung ist die bestimmende Gr"o"ze die den Stein abbremsende
Reibungskraft $ K $. Die senkrecht auf die Bewegungsrichtung wirkende Kraft, die
das Curlen verursacht, ist demgegen"uber verringert. Dabei gibt der Parameter
$ \alpha $ die $ y $-Abh"angigkeit der Kraft an.

Auch das Drehmoment ist proportional zur Kraft $ K $.

\subsection{Numerische L"osung}

Verwendet wird das Koordinatensystem $ S' $ zu $ S $ dazu. In diesem System gibt es
folgende Einheitsvektoren in b.z.w.\ senkrecht zur Laufrichtung des Steins:
\figref{pictureA} %%%% mr1998/07/13: new
%
\iftrue
\begin{equation}
\begin{array}{rcl}
\vec{s}_\parallel &=& \frac{1}{\sqrt{ \dot{p}^2 + \dot{q}^2 }}
    \left( \begin{array}{@{}c@{}}\dot{p} \\[\jot] \dot{q} \end{array} \right) \\[3\jot]
%
\vec{s}_\perp &=& \frac{1}{\sqrt{ \dot{p}^2 + \dot{q}^2 }}
    \left( \begin{array}{@{}r@{}} \dot{q} \\ -\dot{p} \end{array}\right)
\end{array}
\end{equation}
\else
\begin{eqnarray}
\vec{s}_\parallel &=& \frac{ 1 }{ \sqrt{ \dot{p}^2 + \dot{q}^2 } }
    \left(\begin{array}{@{}r@{}} \dot{p} \\[\jot] \dot{q} \end{array} \right) \\
%
\vec{s}_\perp &=& \frac{ 1 }{ \sqrt{ \dot{p}^2 + \dot{q}^2 } }
    \left(\begin{array}{@{}r@{}} \dot{q} \\[\jot] -\dot{p} \end{array} \right)
\end{eqnarray}
\fi

Das Vorzeichen von $ \vec{s}_\perp $ ergibt sich aus der Forderung, da"z $ S $ ein
Rechtssystem sein mu"z.

\subsubsection{Differentialgleichungen}

Mit $ F_x $, $ F_y $ und $ M $ aus Kapitel (\ref{herleitung}) ergeben sich
damit folgende Differentialgleichungen f"ur die Bahnkurve
%
\begin{eqnarray}
\textrm{mit}\quad K &=& \frac{F}{m}\vec{e}	\\
%%%% mr1998/04/03:
\ddot{\vec{x}} &=& \frac{ 1 }{ m }
    \left( \begin{array}{@{}rr@{}}
         F_y & F_x \\[\jot]
        -F_x & F_y
    \end{array} \right)
    \frac{ \dot{\vec{x}} }{ |\dot{\vec{x}}| } \NONUMBER\\
%
\ddot{p} &=& \frac{F_y}{m \sqrt{\dot{p}^2+\dot{q}^2}} \dot{p} +
	     \frac{F_x}{m \sqrt{\dot{p}^2+\dot{q}^2}} \dot{q}
	\eqlab{diff-p}\\
\ddot{q} &=& \frac{F_y}{m \sqrt{\dot{p}^2+\dot{q}^2}} \dot{q} -
	     \frac{F_x}{m \sqrt{\dot{p}^2+\dot{q}^2}} \dot{p}
	\eqlab{diff-q}\\
\dot{\omega} &=& \frac{M}{\Theta}
	\eqlab{diff-w}
\end{eqnarray}

Dabei ist $ m $ die Masse des Steins, $ \Theta $ sein Tr"agheitsmoment.

\subsubsection{Numerik}

Mit dem Eulerschen Verfahren ergeben sich folgende Gleichungen f"ur die
Integration von \eqref{diff-p} bis \eqref{diff-w} mit dem einfachen Ansatz
%
\begin{eqnarray}
%%%% mr1998/04/03:
\ddot{\vec{x}}^{(n)}  &=& \frac{ \dot{\vec{x}}^{(n+1)} - \dot{\vec{x}}^{(n)}
                               }{ \Delta t } \NONUMBER\\
\dot{\vec{x}}^{(n+1)} &=& \Delta t \cdot \ddot{\vec{x}}^{(n)} + \dot{\vec{x}}^{(n)} \NONUMBER\\
     \vec{x} ^{(n+1)} &=& \Delta t \cdot  \dot{\vec{x}}^{(n)} +      \vec{x} ^{(n)} \NONUMBER
%
\end{eqnarray}
oder komponentenweise ausgeschrieben
\begin{eqnarray}
\ddot{p}^{(n)}	&=& \frac{\dot{p}^{(n+1)} - \dot{p}^{(n)}}{\Delta t}
	\eqlab{euler-1}\\
\dot{p}^{(n+1)} &=& \frac{\Delta t}%
		    {m \sqrt{ \dot{p}^{(n)^2} +  \dot{q}^{(n)^2} }}
		    \left(F_y \dot{p}^{(n)} + F_x \dot{q}^{(n)} \right) +
		    \dot{p}^{(n)}
	\eqlab{euler-2}\\
\dot{q}^{(n+1)} &=& \frac{\Delta t}%
		    {m \sqrt{ \dot{p}^{(n)^2} +  \dot{q}^{(n)^2} }}
		    \left(F_y \dot{q}^{(n)} - F_x \dot{p}^{(n)} \right) +
		    \dot{q}^{(n)}
	\eqlab{euler-3}\\
p^{(n+1)} &=& \Delta t\; \dot{p}^{(n)} + p^{(n)}
	\eqlab{euler-4}\\
q^{(n+1)} &=& \Delta t\; \dot{q}^{(n)} + q^{(n)}
	\eqlab{euler-5}\\
\omega^{(n+1)} &=& \frac{\Delta t}{\Theta} M + \omega^{(n)}
	\eqlab{euler-6}
\end{eqnarray}

Dabei bezeichnen der Index $ {}^{(n+1)} $ den Zeitpunkt $ (n+1) \cdot \Delta t $,
der Index $ {}^{(n)} $ den Zeitpunkt $ n \cdot \Delta t $.

Die Anfangswerte f"ur $ n=0 $ m"ussen gegeben sein. $ F_x $, $ F_y $, $ M $
m"ussen zu jedem Zeitpunkt als Funktionen der Geschwindigkeit und der
Winkelgeschwindigkeit berechnet werden.

\subsubsection{Modell f"ur die Reibungskraft}

Die in Kapitel (6.1) gemessene geschwindigkeitsabh"angige Reibung wird mit der
folgenden Funktion angen"ahert.

\begin{figure}[htb]
\begin{center}%
\iffalse
History:
    %%%% mr1998/07/13: - Einige inhaltliche Fehler beseitigt,
                         Formeln minimal klarer formuliert.
    %%%% mr1998/04/03: - Einige inhaltliche Fehler beseitigt,
                         und fragliche Stellen mit ^? markiert
                       - einige Gleichungen hinzugef"ugt (mit \NONUMBER)
                       - kleine "Anderungen an den Zeichnungen
                       - Umbenennungen:
                             Befehl: label/ref -> eqlab/eqref
                             v   -> v_R    Geschwindigkeit auf dem Laufring
                             v_0 -> v      Geschwindigkeit des Steins
                             v_min -> v_b  Parameter in der Reibungsfunktion
                             v_0   -> v_a  Parameter in der Reibungsfunktion

\fi
\documentclass[a4paper]{report}
\usepackage{latexsym}
\usepackage{german}
\usepackage[T1]{fontenc}
% \usepackage[cp850]{inputenc}
\usepackage{pictex}
\input{tumlogo}
\input{frame3cm}

\pagestyle{plain}

\parindent=0mm
\parskip=3mm
\jot=1ex          % zus"atzlicher Zeilenabstand innerhalb eqnarray
\setcounter{secnumdepth}{2}
\setcounter{tocdepth}{3}

%%%%%%%%%%%%%%%%%%%%%%%%%%%%%%%%%%%%%%%%%%%%%%%%%%%%%%%%%%%
%%  Spezielle Befehle f"ur Labels und Referenzen
%%%%%%%%%%%%%%%%%%%%%%%%%%%%%%%%%%%%%%%%%%%%%%%%%%%%%%%%%%%
\newcommand{\eqlab}[1]{\label{eq:#1}}
\newcommand{\eqref}[1]{Gl.~(\ref{eq:#1})}

\newcommand{\figlab}[1]{\label{fig:#1}}
\newcommand{\figref}[1]{Fig.~(\ref{fig:#1})}

\newcommand{\tablab}[1]{\label{tab:#1}}
\newcommand{\tabref}[1]{Tab.~(\ref{tab:#1})}

\iftrue
% keep original numbering => don't number new equations
\newcommand{\NONUMBER}{\nonumber}
\else
% don't keep original numbering => number new equations
\newcommand{\NONUMBER}{}
\fi
%%%%%%%%%%%%%%%%%%%%%%%%%%%%%%%%%%%%%%%%%%%%%%%%%%%%%%%%%%%
%%  Spezielle Befehle f"ur gro"se Ausdr"ucke:
%%%%%%%%%%%%%%%%%%%%%%%%%%%%%%%%%%%%%%%%%%%%%%%%%%%%%%%%%%%

\newcommand{\vRo}{ \vec{v}   }   % the rock's center's speed
\newcommand{\vRi}{ \vec{v}_R }   % the rock's running-ring's speed
\newcommand{\vRiAbsDivvRoCOS}[1]{\sqrt{\displaystyle
    1 + 2\frac{\omega #1}{|\vRo|}\cos\varphi +
    \left( \frac{\omega #1}{|\vRo|} \right)^2 } }
\newcommand{\vRiAbsDivvRoXI}[1]{\sqrt{\displaystyle
    1 + 2\frac{\omega #1}{|\vRo|}\xi +
    \left( \frac{\omega #1}{|\vRo|} \right)^2 } }
\newcommand{\SubstCos}{ \sqrt{1-\xi^2} }
\newcommand{\fpolar}[2]{ f( |\vRi| , #2 \xi, #1 #2 \SubstCos ,\omega) }
\newcommand{\Int}{\int\limits}

\unitlength=5mm
\setcoordinatesystem units <\unitlength,\unitlength>
\title{Curling \\ Diplomarbeit an der \oTUM{3ex} \\ Auszug}
\author{Uli Sutor}
\date{1988}

\begin{document}
\maketitle
\tableofcontents
\listoffigures

\setcounter{chapter}{8}
\chapter{Theorie}

\begin{figure}[htb]
\begin{center}\mbox{%
  %
  % PicTeX-Bild:
  %
  % Bild 80 aus Uli Sutors Diplomarbeit:
  % Der Stein auf dem Rink, Koordinatensysteme
  %
  \beginpicture
  \setplotarea x from -5 to 5, y from -5 to -10
  % The rink-system:
%  \iffinalplot
  \circulararc 360 degrees from 0.25 0 center at 0 0	  % Button
  \circulararc 360 degrees from 1.0  0 center at 0 0	  % 4foot
  \circulararc 360 degrees from 2.0  0 center at 0 0	  % 6foot
  \circulararc 360 degrees from 3.0  0 center at 0 0	  % 8foot
%  \fi
  \put {{\huge S'}}         [tl] at -5   10               % coordsys: title
%  \put {O'}                 [tl] at  0.2 -0.2             % Zero
  \put {\vector(1,0){10}}   [Bl] at -5	  0		  % x axis = Tee-line
  \put {p}		    [l]  at  5	  0		  % x axis label
  \put {\vector(0,1){15}}   [Bl] at  0	 -5		  % y axis = Centerline
  \put {q}		    [b]  at  0	 10		  % y axis label
  %
  % the rock-system:
  %
%  \iffinalplot
  \circulararc 360 degrees from 2.5 5 center at 2 5	  % Rock
%  \fi
  \put {{\huge S}}	    [tl] at  4	  9		  % coordsys: title
%   \put {O}		    [tl] at  2.4  4.6             % Zero
  \put {\vector(4,-1){1.6}} [Bl] at  2	  5		  % x axis
  \put {\line(-4,1){1}}     [Bl] at  2	  5		  % neg. x axis
  \put {x}		    [l]  at  3.6  4.6		  % x axis label
  \put {\vector(1,4){0.8}}  [Bl] at  2	  5		  % y axis
  \put {\line(-1,-4){0.25}} [Bl] at  2	  5		  % neg. y axis
  \put {y}		    [b]  at  2.8  8.4		  % y axis label
  \put {\vector(1,4){0.4}}  [Bl] at  2	  5		  % speedvector
  \put {$\vRo$}	            [tr] at  1.9  6.5		  % speed label
  \endpicture
}\end{center}
\caption{Der Stein auf dem Eis-Rink\figlab{pictureA}}
\end{figure}

\begin{figure}[htb]
\begin{center}\mbox{%
  %
  % PicTeX-Bild:
  %
  % Bild 81 aus Uli Sutors Diplomarbeit:
  % Der Stein mit eigenem Koordinatensystem
  %
  \beginpicture
  \setplotarea x from -9 to 9, y from -7 to 9
  % The rink-system:
%  \iffinalplot
  \circulararc 360 degrees from 3.0 0 center at 0 0
  \circulararc 360 degrees from 3.5 0 center at 0 0
  \circulararc 360 degrees from 5.5 0 center at 0 0
%  \fi
  \put {{\huge S}}	    [tl] at -7	  8
%  \put {O}		    [tr] at -0.1 -0.1
  \put {\vector(1,0){18}}   [Bl] at -9	  0		  % x axis
  \put {x}		    [l]  at  9	  0
  \put {\vector(0,1){16}}   [Bl] at  0	 -7		  % y axis
  \put {y}		    [b] at   0    9
  %
  \put {\vector(0,1){7}}   [Bl] at  0	 0		  % v
  \put {$\vRo$}	           [cl] at  0.1  7
  %
  % Nun der ganze Kleinkram:
  %
  % omega:
%  \iffinalplot
  \circulararc -32 degrees from -2 0.5 center at 0 0
%  \fi
  \put {\vector(1,1){0.1}}  [Bl] at -1.35 1.55
  \put {$\vec{\omega}$}     [Br] at -1.8  1.1
  %
  % Delta R
  \put {\vector(1,-2){0.5}} [Bl] at -2.06 4.13
  \put {\vector(-1,2){0.5}} [Bl] at -0.84 1.68
  \put {$\Delta R$}	    [Br] at -2	  3
  %
  % \vec{r}
  \put {\vector(1,1){2.3}}  [Bl] at  0	  0
  \put {$\vec{r}$}	    [Br] at  1	  1
  %
  % \phi
  \circulararc 45 degrees from 2 0 center at 0 0
  \put {$\varphi$}	    [Bl] at  2.1  0.7
  %
  % R
  \put {\vector(3,-1){2.85}}[Bl] at  0	  0
  \put {R}		    [tl] at  1.5 -0.75
  \endpicture
}\end{center}
\caption{Der Laufring des Steins}
\end{figure}

\section{Das Curlen}

\begin{figure}[htb]
\begin{center}\mbox{%
  %
  % PicTeX-Bild:
  %
  % Die Bahnkurve eines Steins
  %
  \beginpicture
  %\unitlength=5mm
  %\setcoordinatesystem units <\unitlength,\unitlength>
  \setplotarea x from -5 to 5, y from -13 to 5
%  \iffinalplot
  \circulararc 360 degrees from 0 0.5 center at 0 0
  \circulararc 360 degrees from 0 1   center at 0 0
  \circulararc 360 degrees from 0 2   center at 0 0
  \circulararc 360 degrees from 0 3   center at 0 0
%  \fi
  \setquadratic
  \put {\line(1,0){9}}	  [Bl] at -4.5	 3    % Back - line
  \put {\line(1,0){9}}	  [Bl] at -4.5	 0    % Tee - line
  \put {\line(1,0){9}}	  [Bl] at -4.5 -11.5  % Hog - line
  \put {\line(0,1){15.5}} [Bl] at  0   -12.5  % Centre - line
  %
  \plot 0.9  -11   1.5	-3   3	 3.5 /
  \put {\vector(1,2){0.1}} [Bl] at 3.1 3.6
%  \iffinalplot
  \circulararc 360 degrees from 3.3 3.8 center at 3.3 4.3
%  \fi
  %
  \setdashes <1mm>
  \plot 0.45 -11   0.75 -3   1.5 4 /
  \put {\vector(1,3){0.1}} [Bl] at 1.6 4.3
  \endpicture
}\end{center}
\caption{Typische Bahnkurve eines Curlingsteins}
\end{figure}

\subsection{Herleitung der Gleichungen\label{herleitung}}

Geschwindigkeit eines Punktes des Laufrings

\begin{tabular}{@{}ll}
$ \vRi $	  & Geschwindigkeit eines Punktes auf dem Laufring \\
$ \vRo $	  & Geschwindigkeit des Steins \\
$ \vec{\omega} $  & Winkelgeschwindigkeit des Steins
\end{tabular}

\begin{eqnarray}
\vRi &=& \vRo + \vec{\omega} \times \vec{r} =
{-\omega r \sin\varphi \choose |\vRo| + \omega r \cos\varphi}
    \eqlab{speed-vektor}\\
|\vRi | &=& |\vRo| \vRiAbsDivvRoCOS{r}
    \eqlab{speed-skalar}
\end{eqnarray}

F"ur die geschwindigkeitsabh"angige Kraft gilt

\begin{tabular}{@{}ll}
$ \vec{f} $     & Reibungskraft pro Fl"acheneinheit des Laufrings
\end{tabular}

\begin{equation}
\vec{f} = - \frac{\vRi}{|\vRi|} f \eqlab{reibung-vektor}
\end{equation}

dabei wird f"ur $ f $ die allgemeinste Form angenommen
%
\begin{equation}
f = f(|\vRi|,x,y,\omega)    \eqlab{reibung-skalar}
\end{equation}
%
F"ur die Umwandlung in Polarkoordinaten gilt
%
\begin{eqnarray}
x &=& r \cos\varphi \\
y &=& r \sin\varphi
\end{eqnarray}
%
damit ergibt sich f"ur \eqref{reibung-skalar}
%
\begin{equation}
f = f(|\vRi|,r\cos\varphi,r\sin\varphi,\omega) \eqlab{reibung-polar}
\end{equation}

\subsubsection{Kraft senkrecht auf Laufrichtung (bewirkt curlen)}

Wenn man in \eqref{reibung-vektor} nur die $ x $-Komponente von
\eqref{speed-vektor} einsetzt und sowohl "uber den ganzen Laufring, als auch
"uber die Breite des Rings integriert, erh"alt man

\begin{tabular}{@{}ll}
$ \Delta R $	  & Breite des Laufrings \\
$ R $		  & Radius des Laufrings
\end{tabular}

\begin{equation}
F_x = - \Int_{R}^{R+\Delta R} dr \Int_{0}^{2\pi} d\varphi\; r
\frac{-( \omega r \sin\varphi ) f}{|\vRo| \vRiAbsDivvRoCOS{r} }
\end{equation}
%
mit der Substitution
%
\begin{equation}
\xi = \cos\varphi, \quad \textrm{dann}\quad \frac{d\xi}{d\varphi} = -\sin\varphi =
\left\{ {- \SubstCos , \; 0	\leq\varphi < \pi \atop
	 + \SubstCos , \; \pi\leq\varphi < 2\pi }\right.
	  \eqlab{substitution}
\end{equation}
%
folgt mit \eqref{reibung-polar}
%
\begin{equation}
F_x = -\frac{\omega}{|\vRo|} \Int_{R}^{R+\Delta R} dr \; r^2 \left[
\Int_{1}^{-1} d\xi\;\frac{\fpolar{+}{r}}{ \vRiAbsDivvRoXI{r} } +
\Int_{-1}^{1} d\xi\;\frac{\fpolar{-}{r}}{ \vRiAbsDivvRoXI{r} } \right]
\end{equation}
%
und damit
%
\begin{equation}
F_x = -\frac{\omega}{|\vRo|} \Int_{R}^{R+\Delta R} dr\;r^2
\Int_{-1}^{1} d\xi\;\frac{\fpolar{-}{r}-\fpolar{+}{r}}{ \vRiAbsDivvRoXI{r} }
\eqlab{kraft-x}
\end{equation}

Da die Summe im Nenner von \eqref{kraft-x} nur einen Unterschied in der
$ y $-Komponente der Kraft aufweist, ergibt sich nur dann ein von Null
verschiedener Wert, wenn die Kraft eine $ y $-Abh"angigkeit hat.

\subsubsection{Kraft parallel Laufrichtung (bremst Stein)}

Wenn man in \eqref{reibung-vektor} nur die $ y $-Komponente von
\eqref{speed-vektor} einsetzt, ergibt sich
%
\begin{eqnarray}
F_y &=& -\Int_{R}^{R+\Delta R} dr \; \Int_{0}^{2\pi} d\varphi\; % r %%%% mr1998/04/03: r removed!
\frac{ (|\vRo|+\omega r\cos\varphi) f }{ |\vRo| \vRiAbsDivvRoCOS{r}  } =		      \\
&=& -\Int_{R}^{R+\Delta R}dr\;\left[
    \Int_{1}^{-1}d\xi\; (-  \SubstCos^{-1^?})
        \frac{\left(1+\frac{\omega r}{|\vRo|}\xi\right)
             }{ \vRiAbsDivvRoXI{r} } \fpolar{+}{r} +
    \right.
\nonumber\\&&
    +\left.\Int_{-1}^{1}d\xi\; (+  \SubstCos^{-1^?})
        \frac{ \left( 1+\frac{\omega r}{|\vRo|}\xi \right)
             }{ \vRiAbsDivvRoXI{r}  } \fpolar{-}{r}
        \right]
\end{eqnarray}

Die beiden Integranden kann man wieder zusammenfassen
%
\begin{equation}
F_y =  -\Int_{R}^{R+\Delta R} dr \; \Int_{-1}^{1} d\xi \;
%%%% mr1998/04/03: Changed: \frac{ \left( 1 + \frac{\omega r}{|\vRo|}\xi \right)  \SubstCos
    \frac{ \left( 1 + \frac{\omega r}{|\vRo|}\xi \right)   \SubstCos^{-1^?}
         }{ \vRiAbsDivvRoXI{r}  }
    \left[ \fpolar{+}{r} + \fpolar{-}{r} \right]
    \eqlab{kraft-y}
\end{equation}

\subsubsection{Drehmoment (bremst Drehung des Steins)}

F"ur das Drehmoment gilt mit \eqref{reibung-vektor}

\begin{tabular}{@{}ll}
$ \mu $   & Drehmoment auf Fl"achenelement des Laufrings
\end{tabular}

\begin{equation}
\vec{\mu} = \vec{r}\times \vec{f} = -\frac{f}{|\vRi|}\vec{r}\times\vec{v}_R
	\eqlab{dreh-vektor}
\end{equation}

Umgerechnet in Polarkoordinaten folgt in Vektorschreibweise f"ur das
Vektorprodukt
%
\begin{eqnarray}
\vec{r}\times\vec{v}_R &=& \left(\begin{array}{@{}c@{}}
	r\cos\varphi \\ r\sin\varphi \\ 0
\end{array}\right) \times
\left(\begin{array}{@{}c@{}}
	-\omega r\sin\varphi \\ |\vRo| + \omega r\cos\varphi \\ 0
\end{array}\right) =
\left(\begin{array}{@{}c@{}}
	0 \\ 0 \\ r\cos\varphi(|\vRo|+\omega r\cos\varphi) +
                  \omega r^2\sin^2\varphi
\end{array}\right) \\
%%%% mr1998/07/13: new:
&=& \left( |\vRo| r \cos\varphi + \omega r^2 \right) \cdot
    \left(\begin{array}{@{}c@{}} 0 \\ 0 \\ 1 \end{array}\right) \nonumber
\end{eqnarray}

F"ur den Betrag des Drehmoments also
%
\begin{equation}
%%%% mr1998/07/13: _3 added
\mu_3 = -\frac{f}{|\vRi|}\cdot \left(\vec{r}\times\vec{v}_R\right)_3
      = -\frac{f}{|\vRi|}\cdot \left(|\vRo| r \cos\varphi + \omega r^2 \right)
\eqlab{dreh-skalar}\end{equation}
%
setzt man \eqref{dreh-skalar} in \eqref{dreh-vektor} ein, mit
\eqref{speed-skalar}
%
\begin{equation}
%%%% mr1998/07/13: _3 added
\mu_3 = -(|\vRo| r \cos\varphi + \omega r^2) \frac{f}{|\vRi|} =
- \frac{r(\cos\varphi + \frac{\omega r}{|\vRo|}) f }{ \vRiAbsDivvRoCOS{r} } \eqlab{mu-skalar}
\end{equation}

Damit ergibt sich f"ur das Gesamtdrehmoment $ M $
%
\begin{equation}
M = \Int_{R}^{R+\Delta R}dr\;\Int_{0}^{2\pi}d\varphi\;r \mu_3
\end{equation}
%
und nach Einsetzen von \eqref{mu-skalar} mit der Substitution
\eqref{substitution}
%
\begin{eqnarray}
= -\Int_{R}^{R+\Delta R} dr & r^2 &
	\left[
	\Int_{1}^{-1}d\xi\; \frac{-\left(\frac{\omega r}{|\vRo|}+\xi\right)
				  \SubstCos^{-1^?}\; \fpolar{+}{r} }{ \vRiAbsDivvRoXI{r} }  +
	\right.\\ &&
	+\left.
	\Int_{-1}^{1}d\xi\; \frac{+\left(\frac{\omega r}{|\vRo|}+\xi\right)
				  \SubstCos^{-1^?}\; \fpolar{-}{r} }{ \vRiAbsDivvRoXI{r} }
	\right]
	\nonumber
\end{eqnarray}

Die Integranden kann man wieder zusammenfassen
%
\begin{equation}
M = -\Int_{R}^{R+\Delta R} dr\; r^2
    \Int_{1}^{-1}d\xi\;
	\frac{\left(\frac{\omega r}{|\vRo|}+\xi\right)   \SubstCos^{-1^?} }{ \vRiAbsDivvRoXI{r} }
	\left[ \fpolar{+}{r} + \fpolar{-}{r} \right]
    \eqlab{drehmoment}
\end{equation}

\subsection{Analytische Auswertung}

\subsubsection{Allgemeine Bemerkungen}

Aus Gleichung \eqref{kraft-x} folgt, da"z der Stein nur curlt, wenn eine
$ y $-Abh"angigkeit der Kraft vorhanden ist, d.h.\ die Reibungskraft mu"z auf der
Vorderseite des Steins (in Laufrichtung gesehen) verschieden von der auf der
R"uckseite sein. Dabei mu"z die Kraft auf der Vorderseite betragsm"a"zig kleiner
sein.
%%%% mr1998/07/13:
\footnote{Anm.: Meiner Meinung nach kommt \emph{nur} eine
Geschwindigkeitsabh"angigkeit in Frage, da alle Effekte lokal und
isoliert an den einzelnen Pebble wirken. Aus der Kreisbewegung +
L"angsbewegung ergibt sich rein geometrisch die geforderte
Reibungsverteilung.}
Diese Abh"angigkeit hat nichts mit der Geschwindigkeitsabh"angigkeit zu
tun. Aus einer reinen Geschwindigkeitsabh"angigkeit ergibt sich kein Curlen des
Steins, egal, welche genaue Form diese Geschwindigkeitsabh"angigkeit hat.

Die physikalische Ursache f"ur diese Ortsabh"angigkeit ist die Verdr"angung des
Wasserfilms durch den laufenden Stein, d.h.\ die Vorderseite gleitet auf einem
etwas dickeren Wasserfilm als die R"uckseite und hat daher eine niedrigere
Reibung (gleicher Effekt wie bei der Reibungsverminderung durch das
Wischen).%
%%%% mr1998/07/13:
\footnote{Anm.: Auch dies trifft nicht zu, da der Stein keinen
Wasserfilm ben"otigt um darauf zu gleiten. Das Eis selbst ist schon
'glatt'. Man beachte die Forschungsergebnisse von Gabor Somorjai.}

Beim Shuffleboard, bei dem die Scheiben ja in der "`falschen"' Richtung curlen,
wird die $ y $-Abh"angigkeit vermutlich durch kleine Schmutzpartikel auf der Bahn
verursacht, daher hat die R"uckseite, bei der weniger Schmutzpartikel vorhanden
sind, \emph{mehr}%
%%%% mr1998/07/13:
\footnote{Anm.: \emph{weniger}}
Reibung. Damit ergibt sich das Curlen in die andere Richtung. Dies best"atigen
auch Versuche mit einem Curlingstein auf einem Linoleumbelag. 

Des weiteren mu"z die $ y $-Abh"angigkeit der Reibungskraft $ \omega $-abh"angig
sein, da bei einem sehr schnell drehenden Stein der Unterschied zwischen Vorder-
und R"uckseite nicht mehr so ausgepr"agt sein sollte.

F"ur die Tatsache, da"z der Stein curlt ist die Geschwindigkeitsabh"angigkeit
ohne Bedeutung, sie hat nur einen Einflu"z auf den Betrag des Curlens.

\subsubsection{N"aherungsweise L"osung}

\begin{enumerate}
\item Schmaler Laufring: $ \Delta R \ll R $
%
\begin{equation}
r = (R + d R) = R(1+\epsilon), \quad \epsilon\ll 1
\end{equation}

Man n"ahere innerhalb der Integrale "uber $ \xi $ alle $ r $ durch $ R $ an, und
ber"ucksichtige beim Integral "uber $ r $ nur das Glied nullter % erster %%%% mr1998/04/03:
Ordnung in $ \epsilon $.

dann folgt aus \eqref{kraft-x}
%
\begin{equation}
F_x = -\frac{\omega R}{|\vRo|}
\Int_{-1}^{1} d\xi\;\frac{R\Delta R\left(\fpolar{-}{R}-\fpolar{+}{R}\right)}%
			 { \vRiAbsDivvRoXI{R}  }
\eqlab{kraft-x-R}
\end{equation}
%
und aus \eqref{kraft-y}
%
\begin{equation}
F_y =  -\Int_{-1}^{1}d\xi\;
    \frac{ \left( 1 + \frac{\omega R}{|\vRo|}\xi \right)
             \SubstCos^{-1^?}
         }{ \vRiAbsDivvRoXI{R}  }
    R \Delta R
    \left( \fpolar{+}{R} + \fpolar{-}{R} \right)
    \eqlab{kraft-y-R}
\end{equation}
%
das Gesamtdrehmoment ergibt sich
%
\begin{equation}
\eqlab{GesamtDrehmoment}
M = -\Int_{1}^{-1}d\xi\;
    \frac{ R \left( \frac{\omega R}{|\vRo|} + \xi \right)
             \SubstCos^{-1^?}
         }{ \vRiAbsDivvRoXI{R}  }
	R \Delta R
	\left( \fpolar{+}{R} + \fpolar{-}{R} \right)
    \eqlab{drehmoment-R}
\end{equation}

\item Einfachster Ansatz f"ur f:

Da nur eine $ y $-Abh"angigkeit der Kraft interessiert, wird f"ur diese als
einfachster Ansatz eine lineare Beziehung angesetzt.
%
\begin{equation}
f(|\vRi|,x,y,\omega) = f_0\left(1-\alpha\frac{y}{R}\right),
\quad \textrm{mit}\quad \alpha > 0.
\eqlab{kraft-linear}
\end{equation}

Bei diesem Ansatz ist die $ \omega $-Abh"angigkeit von $ f $ vernachl"assigt, ebenso
die Geschwindigkeitsabh"angigkeit. Die $ y $-Abh"angigkeit wird durch eine lineare
Beziehung gen"ahert.

Als willk"urliche Kraftkonstante wird nun $ K $ eingef"uhrt mit
%
\begin{displaymath}
K = \pi R \Delta R f_0, \quad\textrm{dann gilt:}
\end{displaymath}

%%%% mr1998/04/03: F"ur den Nenner von \eqref{kraft-x-R} mit \eqref{kraft-linear}
F"ur den Z"ahler von \eqref{kraft-x-R} mit \eqref{GesamtDrehmoment}
%
\begin{eqnarray}
R \Delta R \left(\fpolar{-}{R} - \fpolar{+}{R} \right) &=&
\frac{2}{\pi} \alpha K  \SubstCos 		   \\
R \Delta R \left(\fpolar{+}{R} + \fpolar{-}{R} \right) &=&
\frac{2}{\pi} K
\end{eqnarray}
%
und damit wegen
%
\begin{equation}
\Int_{-1}^{1} d\xi\; \SubstCos  = \frac{\pi}{2}
\end{equation}
%
\begin{eqnarray}
F_x &=& - \frac{\omega R}{|\vRo|} \alpha K \eqlab{kraft-x-N}\\
F_y &=& - K				\eqlab{kraft-y-N}\\
M   &=& - \frac{\omega R}{|\vRo|} R K	\eqlab{drehmoment-N}
\end{eqnarray}

\end{enumerate}

In dieser N"aherung wurde noch verwendet, da"z $ \frac{\omega R}{|\vRi|}\ll 1 $, eine
Annahme, die bei einer der Realit"at entsprechenden Winkelgeschwindigkeit von
ca.\ $ 1\frac{1}{ \textrm{s} } $ richtig ist, bis kurz vor dem Stillstand des Steins.

\subsubsection{Auswertung der N"aherungsl"osung}

Die Gleichungen \eqref{kraft-x-N} bis \eqref{drehmoment-N} beschreiben das grobe
Verhalten des Steins in dem Bereich, in dem die Bahngeschwindigkeit gro"z ist
gegen die Geschwindigkeit eines Punktes auf dem Laufring, solage der Stein sich
nicht zu schnell dreht.

In dieser N"aherung ist die bestimmende Gr"o"ze die den Stein abbremsende
Reibungskraft $ K $. Die senkrecht auf die Bewegungsrichtung wirkende Kraft, die
das Curlen verursacht, ist demgegen"uber verringert. Dabei gibt der Parameter
$ \alpha $ die $ y $-Abh"angigkeit der Kraft an.

Auch das Drehmoment ist proportional zur Kraft $ K $.

\subsection{Numerische L"osung}

Verwendet wird das Koordinatensystem $ S' $ zu $ S $ dazu. In diesem System gibt es
folgende Einheitsvektoren in b.z.w.\ senkrecht zur Laufrichtung des Steins:
\figref{pictureA} %%%% mr1998/07/13: new
%
\iftrue
\begin{equation}
\begin{array}{rcl}
\vec{s}_\parallel &=& \frac{1}{\sqrt{ \dot{p}^2 + \dot{q}^2 }}
    \left( \begin{array}{@{}c@{}}\dot{p} \\[\jot] \dot{q} \end{array} \right) \\[3\jot]
%
\vec{s}_\perp &=& \frac{1}{\sqrt{ \dot{p}^2 + \dot{q}^2 }}
    \left( \begin{array}{@{}r@{}} \dot{q} \\ -\dot{p} \end{array}\right)
\end{array}
\end{equation}
\else
\begin{eqnarray}
\vec{s}_\parallel &=& \frac{ 1 }{ \sqrt{ \dot{p}^2 + \dot{q}^2 } }
    \left(\begin{array}{@{}r@{}} \dot{p} \\[\jot] \dot{q} \end{array} \right) \\
%
\vec{s}_\perp &=& \frac{ 1 }{ \sqrt{ \dot{p}^2 + \dot{q}^2 } }
    \left(\begin{array}{@{}r@{}} \dot{q} \\[\jot] -\dot{p} \end{array} \right)
\end{eqnarray}
\fi

Das Vorzeichen von $ \vec{s}_\perp $ ergibt sich aus der Forderung, da"z $ S $ ein
Rechtssystem sein mu"z.

\subsubsection{Differentialgleichungen}

Mit $ F_x $, $ F_y $ und $ M $ aus Kapitel (\ref{herleitung}) ergeben sich
damit folgende Differentialgleichungen f"ur die Bahnkurve
%
\begin{eqnarray}
\textrm{mit}\quad K &=& \frac{F}{m}\vec{e}	\\
%%%% mr1998/04/03:
\ddot{\vec{x}} &=& \frac{ 1 }{ m }
    \left( \begin{array}{@{}rr@{}}
         F_y & F_x \\[\jot]
        -F_x & F_y
    \end{array} \right)
    \frac{ \dot{\vec{x}} }{ |\dot{\vec{x}}| } \NONUMBER\\
%
\ddot{p} &=& \frac{F_y}{m \sqrt{\dot{p}^2+\dot{q}^2}} \dot{p} +
	     \frac{F_x}{m \sqrt{\dot{p}^2+\dot{q}^2}} \dot{q}
	\eqlab{diff-p}\\
\ddot{q} &=& \frac{F_y}{m \sqrt{\dot{p}^2+\dot{q}^2}} \dot{q} -
	     \frac{F_x}{m \sqrt{\dot{p}^2+\dot{q}^2}} \dot{p}
	\eqlab{diff-q}\\
\dot{\omega} &=& \frac{M}{\Theta}
	\eqlab{diff-w}
\end{eqnarray}

Dabei ist $ m $ die Masse des Steins, $ \Theta $ sein Tr"agheitsmoment.

\subsubsection{Numerik}

Mit dem Eulerschen Verfahren ergeben sich folgende Gleichungen f"ur die
Integration von \eqref{diff-p} bis \eqref{diff-w} mit dem einfachen Ansatz
%
\begin{eqnarray}
%%%% mr1998/04/03:
\ddot{\vec{x}}^{(n)}  &=& \frac{ \dot{\vec{x}}^{(n+1)} - \dot{\vec{x}}^{(n)}
                               }{ \Delta t } \NONUMBER\\
\dot{\vec{x}}^{(n+1)} &=& \Delta t \cdot \ddot{\vec{x}}^{(n)} + \dot{\vec{x}}^{(n)} \NONUMBER\\
     \vec{x} ^{(n+1)} &=& \Delta t \cdot  \dot{\vec{x}}^{(n)} +      \vec{x} ^{(n)} \NONUMBER
%
\end{eqnarray}
oder komponentenweise ausgeschrieben
\begin{eqnarray}
\ddot{p}^{(n)}	&=& \frac{\dot{p}^{(n+1)} - \dot{p}^{(n)}}{\Delta t}
	\eqlab{euler-1}\\
\dot{p}^{(n+1)} &=& \frac{\Delta t}%
		    {m \sqrt{ \dot{p}^{(n)^2} +  \dot{q}^{(n)^2} }}
		    \left(F_y \dot{p}^{(n)} + F_x \dot{q}^{(n)} \right) +
		    \dot{p}^{(n)}
	\eqlab{euler-2}\\
\dot{q}^{(n+1)} &=& \frac{\Delta t}%
		    {m \sqrt{ \dot{p}^{(n)^2} +  \dot{q}^{(n)^2} }}
		    \left(F_y \dot{q}^{(n)} - F_x \dot{p}^{(n)} \right) +
		    \dot{q}^{(n)}
	\eqlab{euler-3}\\
p^{(n+1)} &=& \Delta t\; \dot{p}^{(n)} + p^{(n)}
	\eqlab{euler-4}\\
q^{(n+1)} &=& \Delta t\; \dot{q}^{(n)} + q^{(n)}
	\eqlab{euler-5}\\
\omega^{(n+1)} &=& \frac{\Delta t}{\Theta} M + \omega^{(n)}
	\eqlab{euler-6}
\end{eqnarray}

Dabei bezeichnen der Index $ {}^{(n+1)} $ den Zeitpunkt $ (n+1) \cdot \Delta t $,
der Index $ {}^{(n)} $ den Zeitpunkt $ n \cdot \Delta t $.

Die Anfangswerte f"ur $ n=0 $ m"ussen gegeben sein. $ F_x $, $ F_y $, $ M $
m"ussen zu jedem Zeitpunkt als Funktionen der Geschwindigkeit und der
Winkelgeschwindigkeit berechnet werden.

\subsubsection{Modell f"ur die Reibungskraft}

Die in Kapitel (6.1) gemessene geschwindigkeitsabh"angige Reibung wird mit der
folgenden Funktion angen"ahert.

\begin{figure}[htb]
\begin{center}%
\input{sutor.ltx}
\end{center}
\caption{Reibungsmodell Stein/Eis $ F(v) $ nach Uli Sutor (grobe N"aherung)}
\end{figure}

Ansatz f"ur $ f $:
%
\begin{eqnarray}
f      &=& f_1(v)\;f_2(y,\omega)       \\
f_1(v) &=& \frac{1}{2} \frac{ f_\textrm{min} }{ v_b } \left(
	   v + v_a + \frac{ v_b^2 }{v + v_a} \right)
\end{eqnarray}

Die Funktion hat an der Stelle $ v_b - v_a $ ein absolutes Minimum mit
dem Wert $ f_\textrm{min} $. Ist der Wert von $ f_1 $ f"ur $ v = 0 $ bekannt,
n"amlich $ f_0 $, so l"a"zt sich daraus $ v_a $ bestimmen:
%
\begin{equation}
v_a =  v_b  \left( \frac{ f_0 }{ f_\textrm{min} } -
	\sqrt{\left( \frac{ f_0 }{ f_\textrm{min} } \right)^2 - 1} \right)
\end{equation}

Diese Form der Geschwindigkeitsabh"angigkeit entspricht den experimentellen
Daten. Der Abfall von $ v=0 $ bis $ v_b - v_a $ l"a"zt sich durch das
Wegfallen der Haftreibung, der Anstieg danach durch die normale Zunahme der
Gleitreibung mit der Geschwindigkeit erkl"aren.
%%%% mr1998/04/03: Das $ v_a $ in diesen Gleichungen ist nat"urlich nicht identisch mit dem $ v_a $ aus den Gleichungen der vorhergehenden Abschnitte
%
\begin{equation}
f_2(y,\omega) = 1 - \left( 1 - e^{ -\frac{\alpha y}{r}} \right)
			       e^{1-\frac{\omega}{\omega_0}}
\end{equation}

Exponentieller Abfall mit $ y $ beschreibt die $ y $-Abh"angigkeit, diese wird
durch den zweiten Exponentialfaktor ged"ampft, um die $ \omega $-Abh"angigkeit
richtig zu erhalten. F"ur kleine $ \alpha $ und $ \omega = \omega_0 $ ergibt
sich die lineare Abh"angigkeit als N"aherung, die im analytischen Modell
gemacht wurde.

Die Auswertung der numerischen Formeln der N"aherung $ R \gg \Delta R $ erfolgt
im FORTRAN-Programm CURL.PRG. (siehe Anhang)

\end{document}

\end{center}
\caption{Reibungsmodell Stein/Eis $ F(v) $ nach Uli Sutor (grobe N"aherung)}
\end{figure}

Ansatz f"ur $ f $:
%
\begin{eqnarray}
f      &=& f_1(v)\;f_2(y,\omega)       \\
f_1(v) &=& \frac{1}{2} \frac{ f_\textrm{min} }{ v_b } \left(
	   v + v_a + \frac{ v_b^2 }{v + v_a} \right)
\end{eqnarray}

Die Funktion hat an der Stelle $ v_b - v_a $ ein absolutes Minimum mit
dem Wert $ f_\textrm{min} $. Ist der Wert von $ f_1 $ f"ur $ v = 0 $ bekannt,
n"amlich $ f_0 $, so l"a"zt sich daraus $ v_a $ bestimmen:
%
\begin{equation}
v_a =  v_b  \left( \frac{ f_0 }{ f_\textrm{min} } -
	\sqrt{\left( \frac{ f_0 }{ f_\textrm{min} } \right)^2 - 1} \right)
\end{equation}

Diese Form der Geschwindigkeitsabh"angigkeit entspricht den experimentellen
Daten. Der Abfall von $ v=0 $ bis $ v_b - v_a $ l"a"zt sich durch das
Wegfallen der Haftreibung, der Anstieg danach durch die normale Zunahme der
Gleitreibung mit der Geschwindigkeit erkl"aren.
%%%% mr1998/04/03: Das $ v_a $ in diesen Gleichungen ist nat"urlich nicht identisch mit dem $ v_a $ aus den Gleichungen der vorhergehenden Abschnitte
%
\begin{equation}
f_2(y,\omega) = 1 - \left( 1 - e^{ -\frac{\alpha y}{r}} \right)
			       e^{1-\frac{\omega}{\omega_0}}
\end{equation}

Exponentieller Abfall mit $ y $ beschreibt die $ y $-Abh"angigkeit, diese wird
durch den zweiten Exponentialfaktor ged"ampft, um die $ \omega $-Abh"angigkeit
richtig zu erhalten. F"ur kleine $ \alpha $ und $ \omega = \omega_0 $ ergibt
sich die lineare Abh"angigkeit als N"aherung, die im analytischen Modell
gemacht wurde.

Die Auswertung der numerischen Formeln der N"aherung $ R \gg \Delta R $ erfolgt
im FORTRAN-Programm CURL.PRG. (siehe Anhang)

\end{document}

\end{center}
\caption{Reibungsmodell Stein/Eis $ F(v) $ nach Uli Sutor (grobe N"aherung)}
\end{figure}

Ansatz f"ur $ f $:
%
\begin{eqnarray}
f      &=& f_1(v)\;f_2(y,\omega)       \\
f_1(v) &=& \frac{1}{2} \frac{ f_\textrm{min} }{ v_b } \left(
	   v + v_a + \frac{ v_b^2 }{v + v_a} \right)
\end{eqnarray}

Die Funktion hat an der Stelle $ v_b - v_a $ ein absolutes Minimum mit
dem Wert $ f_\textrm{min} $. Ist der Wert von $ f_1 $ f"ur $ v = 0 $ bekannt,
n"amlich $ f_0 $, so l"a"zt sich daraus $ v_a $ bestimmen:
%
\begin{equation}
v_a =  v_b  \left( \frac{ f_0 }{ f_\textrm{min} } -
	\sqrt{\left( \frac{ f_0 }{ f_\textrm{min} } \right)^2 - 1} \right)
\end{equation}

Diese Form der Geschwindigkeitsabh"angigkeit entspricht den experimentellen
Daten. Der Abfall von $ v=0 $ bis $ v_b - v_a $ l"a"zt sich durch das
Wegfallen der Haftreibung, der Anstieg danach durch die normale Zunahme der
Gleitreibung mit der Geschwindigkeit erkl"aren.
%%%% mr1998/04/03: Das $ v_a $ in diesen Gleichungen ist nat"urlich nicht identisch mit dem $ v_a $ aus den Gleichungen der vorhergehenden Abschnitte
%
\begin{equation}
f_2(y,\omega) = 1 - \left( 1 - e^{ -\frac{\alpha y}{r}} \right)
			       e^{1-\frac{\omega}{\omega_0}}
\end{equation}

Exponentieller Abfall mit $ y $ beschreibt die $ y $-Abh"angigkeit, diese wird
durch den zweiten Exponentialfaktor ged"ampft, um die $ \omega $-Abh"angigkeit
richtig zu erhalten. F"ur kleine $ \alpha $ und $ \omega = \omega_0 $ ergibt
sich die lineare Abh"angigkeit als N"aherung, die im analytischen Modell
gemacht wurde.

Die Auswertung der numerischen Formeln der N"aherung $ R \gg \Delta R $ erfolgt
im FORTRAN-Programm CURL.PRG. (siehe Anhang)

\end{document}

\end{center}
\caption{Reibungsmodell Stein/Eis $ F(v) $ nach Uli Sutor (grobe N"aherung)}
\end{figure}

Ansatz f"ur $ f $:
%
\begin{eqnarray}
f      &=& f_1(v)\;f_2(y,\omega)       \\
f_1(v) &=& \frac{1}{2} \frac{ f_\textrm{min} }{ v_b } \left(
	   v + v_a + \frac{ v_b^2 }{v + v_a} \right)
\end{eqnarray}

Die Funktion hat an der Stelle $ v_b - v_a $ ein absolutes Minimum mit
dem Wert $ f_\textrm{min} $. Ist der Wert von $ f_1 $ f"ur $ v = 0 $ bekannt,
n"amlich $ f_0 $, so l"a"zt sich daraus $ v_a $ bestimmen:
%
\begin{equation}
v_a =  v_b  \left( \frac{ f_0 }{ f_\textrm{min} } -
	\sqrt{\left( \frac{ f_0 }{ f_\textrm{min} } \right)^2 - 1} \right)
\end{equation}

Diese Form der Geschwindigkeitsabh"angigkeit entspricht den experimentellen
Daten. Der Abfall von $ v=0 $ bis $ v_b - v_a $ l"a"zt sich durch das
Wegfallen der Haftreibung, der Anstieg danach durch die normale Zunahme der
Gleitreibung mit der Geschwindigkeit erkl"aren.
%%%% mr1998/04/03: Das $ v_a $ in diesen Gleichungen ist nat"urlich nicht identisch mit dem $ v_a $ aus den Gleichungen der vorhergehenden Abschnitte
%
\begin{equation}
f_2(y,\omega) = 1 - \left( 1 - e^{ -\frac{\alpha y}{r}} \right)
			       e^{1-\frac{\omega}{\omega_0}}
\end{equation}

Exponentieller Abfall mit $ y $ beschreibt die $ y $-Abh"angigkeit, diese wird
durch den zweiten Exponentialfaktor ged"ampft, um die $ \omega $-Abh"angigkeit
richtig zu erhalten. F"ur kleine $ \alpha $ und $ \omega = \omega_0 $ ergibt
sich die lineare Abh"angigkeit als N"aherung, die im analytischen Modell
gemacht wurde.

Die Auswertung der numerischen Formeln der N"aherung $ R \gg \Delta R $ erfolgt
im FORTRAN-Programm CURL.PRG. (siehe Anhang)

\end{document}
