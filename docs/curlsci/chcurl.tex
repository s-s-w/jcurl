\chapter{Trajectory}

\section{The 'Denny' model}

Mark Denny published a model in \cite{denny:98}. It's quite simple and provides
polynomes of fourth degree to describe the rock's motion along the sheet.

\begin{figure}[htb]
\begin{center}
\input{denny.pic}
\end{center}
\caption[Denny setup]{The \emph{running band} of radius $R$ is the surface
    of contact between rock and ice. Here the velocity components, relative to
    the ice, of a point on the running band are shown. $v$ is the CM velocity,
    and $w$ is the angular component at radius $R$. The unit vectors $(u_T, u)$
    coincide with $(u_x,u_y)$ at time $t=0$. In this case the rock is curling
    in a counterclockwise sense (so angular velocity unit vector $=u_z$).
    \figlab{denny}}
\end{figure}

\iftrue
\begin{figure}[tb]
\begin{center}
\input{denny.emt}
\end{center}
\caption[Curling rock trajectories]{\figlab{dennytrj}%
     Approximate curling rock trajectories, calculated from 
     \eqref{denA}, \eqref{denB} and \eqref{denC}
     for three values of initial velocity $v_0$. The other parameters are:
     $\mu           = 0.0127$,
     $b_\mathrm{LR} = 0.003$,
     $\epsilon      = 2.63$,
     $R             = 0.065$.}
\end{figure}
\fi

\begin{eqnarray}
\eqlab{denA} 
x(t) &\approx& - \frac{ b_\mathrm{LR} v_0^2 }{ 4 \epsilon R \tau }
    \left( \frac{ t^3 }{ 3 } - \frac{ t^4 } { 4 \tau } \right) \\
\eqlab{denB} 
y(t) &\approx& v_0 \left( t - \frac{ t^2 }{ 2\tau } \right)
\end{eqnarray}

\bigskip
\begin{tabular}{lll}
$t$		& [s] & time \\
$x$             & [m] & curl \\
$y$             & [m] & distance along the track \\
$b_\mathrm{LR}$ & [1] & parameter for the curl's magnitude \\
$\epsilon$	& [1] & parameter $ \frac{ I }{ mR^2 } $\\
$\tau$		& [s] & total time until $v=0$ \\
$R$		& [m] & radius of the touching area rock/ice $\approx$ 6.3e-2 \\
$\mu$		& [1] & friction coefficient rock/ice \\
$I$		&     & moment of inertia (z-direction) \\
$g$		& [N/kg] & 9.81 gravitation \\
\end{tabular}
\bigskip

The final properties are:
\begin{eqnarray}
\eqlab{denC} 
\tau    & = &     \frac{ v_0 }{ \mu g } \\
x(\tau) &\approx& - \frac{ b_\mathrm{LR} }{ 12\epsilon } \frac{y^2(\tau) }{R} \\
y(\tau) &\approx& \frac{ v_0^2 }{ 2\mu g }
\end{eqnarray}

\subsection{The ice-properties $\mu$ and $b$}

To calculate $\mu$ and $b$ from the time $\tau=T$ a draw-to-tee requires
(hog to tee) we find:

\begin{eqnarray}
y(T) &=& D \quad \mbox{(distance hog to tee)} \\
y(t) &=& v_0 \left(t-\frac{t^2}{2\tau}\right) \nonumber\\
\mbox{and} \quad v_0 &=& \tau \mu g \nonumber\\
\Rightarrow\quad y(T) &=& T \mu g \left(T - \frac{T^2}{2T} \right) =
\frac{1}{2} T^2 \mu g \\
\Longrightarrow\quad \mu &=& \frac{2D}{T^2 g}
\end{eqnarray}


\begin{eqnarray}
x(T) &\approx& - \frac{b}{12\epsilon} \frac{D^2}{R} \quad(\cdot \sgn(\omega_0)) \nonumber\\
\Longrightarrow\quad
b &\approx& \frac{-(curl) \cdot 12\epsilon R}{ D^2 \quad (\cdot \sgn(\omega_0)) }
\end{eqnarray}

\subsection{Some initial speeds}

\textit{How hard do we have to throw a rock, that will take 12 seconds hog to
hog?}

To calculate $v_0$ at the far hog we don't use the time hog-to-tee for not
every rock reaches the tee-line. Here the time hog-to-hog ($T_H$) is better.
\begin{eqnarray}
\mbox{dist.~hog--hog:} \quad H &=& v_0 \left( T_H - \frac{T_H^2}{2\tau}\right) \\
&=& v_0 \left(T_H - \frac{T_H^2}{2 \frac{v_0}{\mu g} }\right) \\
\frac{H}{v_0} &=& T_H - \frac{T_H^2}{v_0} \frac{\mu g}{2} \\
T_H &=& \frac{1}{v_0} \left(H + T_H^2 \frac{\mu g}{2}\right) \\
\Longrightarrow\quad
v_0 &=& \frac{H}{T_H} + \frac{T_H\mu g}{2}
\end{eqnarray}

If we don't want $v_0$ at the far hog but at any given distance, things get a
little bit more complicated.

Here we use \eqref{denB} at three times

\medskip
\begin{tabular}{l@{ : }l}
$t_0          $ & our starting point \\
$t_1 \equiv 0 $ & the rock crosses the far hog \\
$T_H $          & time hog-to-hog
\end{tabular}
\medskip

\noindent
with two distances

\medskip
\begin{tabular}{l@{ : }l}
$y_0$ & distance start-to-far-hog \\
$H  $ & distance hog-to-hog
\end{tabular}
\medskip

\begin{eqnarray}
y(t_0 - t_0) &:& 0 = 0 \\
\eqlab{denVA}
y(t_1 - t_0) &:& v_0 \left(-t_0 - \frac{t_0^2 \mu g}{2 v_0}\right) = -y_0 \\
\eqlab{denVB}
y(T_H - t_0) &:& v_0 \left(T_H-t_0 - \frac{(T_H-t_0)^2\mu g}{2v_0}\right) = H-y_0 
\end{eqnarray}

\eqref{denVA} + \eqref{denVB}:
\begin{eqnarray}
H &=& v_0 T_H - T_H\frac{\mu g}{2} + T_H t_0 \mu g \\
\Longrightarrow\quad
t_0 &=& \frac{H - v_0 T_H + T_H^2\frac{\mu g}{2}}{ T_H \mu g }
\end{eqnarray}

Substituting this into \eqref{denVA} and solving for $v_0$ gives
\begin{eqnarray}
\Longrightarrow v_0 =  \frac{1}{2}\cdot 
    \sqrt{
	\left(\mu g T_H + 2\frac{H}{T_H}\right)^2 +
	4 \mu g H - 
	8 \mu g y_0
    }
\end{eqnarray}

If you prefer using $Y_0$ measured from the tee, use
\begin{eqnarray}
\Longrightarrow v_0 =  \frac{1}{2}\cdot 
    \sqrt{
	\left(\mu g T_H + 2\frac{H}{T_H}\right)^2 + 
	4 \mu g H - 
	8 \mu g (\mbox{far-hog-to-tee} - Y_0)
    }
\end{eqnarray}

\subsection{Coordinate transformation}

Because of Denny assumes the rock to start at
$(0,0)^T$ with $v_0$ pointing along the $\hat{y}$-axis, we need a
rotation and shift to get the general equations.

\begin{eqnarray}
{x \choose y} &:=& \frac{ \vec{v}_\mathrm{real} }{ | \vec{v}_\mathrm{real} | }
\end{eqnarray}

\begin{eqnarray}
\mbox{The required transformation is:}\quad
\left(\begin{array}{cc}
 y &  x \\
 x & -y
\end{array}\right)
\end{eqnarray}

Applying this trafo to e.g.\ ${a \choose b}$ results in
\begin{eqnarray}
\left(\begin{array}{cc}
 y &  x \\
 x & -y
\end{array}\right)
\cdot {a \choose b} &=&
{ay+bx \choose ax-by}
\end{eqnarray}
%
where $a$ and $b$ are polynomes of max.\ fourth degree:
%
\begin{eqnarray}
a &=:& At^4 + Bt^3 + Ct^2 + Dt + E \\
b &=:& \alpha t^4 + \beta t^3 + \gamma t^2 + \delta t + \phi
\end{eqnarray}

This leads to a x-component of
\begin{eqnarray}
&& y ( At^4 + Bt^3 + Ct^2 + Dt + E ) + \nonumber \\
&& x ( \alpha t^4 + \beta t^3 + \gamma t^2 + \delta t + \phi ) \\
&=& (Ay + \alpha x)t^4 + \ldots + (Ey + \phi x)
\end{eqnarray}

and a y-component of
\begin{eqnarray}
&&  x ( At^4 + Bt^3 + Ct^2 + Dt + E ) + \nonumber \\
&& -y ( \alpha t^4 + \beta t^3 + \gamma t^2 + \delta t + \phi ) \\
&=& (Ax - \alpha y)t^4 + \ldots + (Ex - \phi y)
\end{eqnarray}

Now just the shift is missing.

\subsection{Spin stuff}

So much about speed and location, but what about the spin and angle?
Denny says:
\begin{eqnarray}
\omega = \omega_0 \left(1 - \frac{t}{\tau}\right)^{1/\epsilon}
\end{eqnarray}

This equation seems pretty hard to integrate, so better just use a numerical
means to get the actual handle-angles. (For they aren't very important, too.)

\clearpage
\section{The 'Shegelski et~al.' model}

This model is much more complicated. A sketch is published in
\cite{shegelski:96}:

\dots

{%% keep the following 'newcommand's local:
\newcommand{\si}[2]{r_{#1}\omega\sin\Theta #2 v_x}
\newcommand{\co}[2]{v_y #2 r_{#1}\omega\cos\Theta}
\newcommand{\tansico}[3]{ \tan^{-1}\left[\frac{ \si{#1}{#2} }{ \co{#1}{#3} }\right] }
\newcommand{\sqrsico}[3]{ \left[ (\si{#1}{#2})^2 + (\co{#1}{#3})^2 \right]^{\Phi/2} }
\newenvironment{wideeq}[1]{\begin{list}{}{%
	\leftmargin=#1%
	\advance\leftmargin by -\textwidth%
	\leftmargin=-.5\leftmargin\rightmargin=\leftmargin}\item[]%
%	\hrulefill
	\footnotesize
    \begin{eqnarray}%
}{\end{eqnarray}\end{list}}
\newenvironment{flowtext}[1]{%
%    \begin{multicols}{#1}\small%
}{%
%    \end{multicols}%
}

\subsection{Equations of motion of a curling rock}

\begin{flowtext}{2}
We next give the equations that determine the motion of the curling rock down
the pebbled ice surface. This is a straightforward exercise in determining the
net force and torque exerted by the ice on the rock.

We choose the coordinate axes as follows. The $+y$ axis is in the direction of
the \emph{initial} velocity of the rock, and the $+x$ axis is perpendicular to
the initial velocity and in the direction in which the rock is expected to
curl. In obtaining the results given in the next section, we also used a $+y'$
axis, which is in the direction of the instantaneous velocity of the rock, as
well as a $+x'$ axis.

We assume that both wet and dry frictional forces exerted on a small section of
the annulus of the rock are in the direction opposite to the velocity of the
point relative to the ice. The magnitude of the dry friction is
\begin{eqnarray}
\Delta F^d &=& \mu M g \left( \frac{\Delta\Theta}{2\pi}\right)
\end{eqnarray}
where $\mu$ is the coefficient of kinetic friction.\footnote{In the case of a
curling rock, the nonuniformity in the normal force around the rock is
neglegible in our model; the nonuniformity is a consequence of the acceleration
due to friction.} We take the magnitude of the wet friction increase with
velocity, in analogy with increased drag on an object moving in a fluid, and
write
\begin{eqnarray}
\Delta F^w &=& k \left[ u(\Theta) \right]^\Phi
\end{eqnarray}
where $k$ and $\Phi$ are parameters to be discussed below, and $u(\Theta)$ is
the net speed, relative to the ice, of the portion of the contact annulus at
angle $\Theta$, as shown in \figref{shegelski}.

To minimize the numerical analysis, instead of integrating over the entire area
of the annulus, we break each semicircle into an outer and inner semicircle. By
considering each quadrant of the rock seperately, the following equations for
the $x$ and $y$ components of the force on the rock are readily obtained. We
first give the equations, then complete our definitions of symbols used.
\end{flowtext}

\begin{figure}[htb] 
\begin{center} 
\input{sheg.pic}
\end{center} 
\caption[Shegelski et al.\ setup]{The contributions $\vec{v}_y$,
    $\vec{v}_x$ and $\vec{v}_{rot}$ (with $\vec{v}_{rot} = \vec{r}\omega$) to
    the net velocity $\vec{u} \equiv \vec{u}(\Theta)$, relative to the ice, of
    a portion of the contact annulus located at the angle $\Theta$ relative to
    the $x$-axis. The figure shown is for a portion located in the first
    quadrant; similar figures are readily constructed for the other three
    quadrants. The angle $\Theta$ in each case is from the $x$-axis toward the
    $y$-axis. Note that the force $\vec{F}$ exerted on the portion is in the
    opposite direction to $\vec{u}$. The angle $\eta$ appears in the equations
    explicitely as the arctan of the ratio of the $x$ and $y$ components of
    $\vec{u}$. 
    \figlab{shegelski}} 
\end{figure}

The x-component of dry friction is:
\begin{wideeq}{170mm}
F_x^d &=& - f_L^d \frac{\mu Mg}{2\pi} \Int_0^{\pi/2} d\Theta \left\{
\sin\left( \tansico{2}{+}{+} \right)
+
\sin\left( \tansico{2}{+}{-} \right)
\right\} \nonumber \\
&& + f_T^d \frac{\mu Mg}{2\pi} \Int_0^{\pi/2} d\Theta \left\{
\sin\left( \tansico{1}{-}{+} \right)
+
\sin\left( \tansico{1}{-}{-} \right)
\right\}
\end{wideeq}

The x-component of wet friction is:
\begin{wideeq}{170mm}
F_x^w &=& - kf_L^w \Int_0^{\pi/2} d\Theta
 \sqrsico{1}{+}{+} \sin\left( \tansico{1}{+}{+} \right) \nonumber\\
 && - kf_L^w \Int_0^{\pi/2} d\Theta
 \sqrsico{1}{+}{-} \sin\left( \tansico{1}{+}{-} \right) \nonumber\\
 && + kf_T^w \Int_0^{\pi/2} d\Theta
 \sqrsico{2}{-}{+} \sin\left( \tansico{2}{-}{+} \right) \nonumber\\
 && + kf_T^w \Int_0^{\pi/2} d\Theta
 \sqrsico{2}{-}{-} \sin\left( \tansico{2}{-}{-} \right)
\end{wideeq}

Similar equations hold for the y-component of the force.

The following equations give the torque due to dry and wet friction:
\begin{wideeq}{170mm}
\tau^d &=& - r_2 \frac{\mu Mg}{2\pi} \Int_0^{\pi/2} d\Theta f_L^d
    \sin\left(\Theta + \frac{\pi}{2} - \tansico{2}{+}{+} \right) \nonumber\\
&& + r_2 \frac{\mu Mg}{2\pi} \Int_0^{\pi/2} d\Theta f_L^d
    \sin\left(\Theta + \frac{\pi}{2} + \tansico{2}{+}{-} \right) \nonumber\\
&& - r_1 \frac{\mu Mg}{2\pi} \Int_0^{\pi/2} d\Theta f_T^d
    \sin\left(\frac{\pi}{2} - \Theta + \tansico{1}{-}{+} \right) \nonumber\\
&& + r_1 \frac{\mu Mg}{2\pi} \Int_0^{\pi/2} d\Theta f_T^d
    \sin\left(\frac{\pi}{2} - \Theta - \tansico{1}{-}{-} \right)
\end{wideeq}

\begin{wideeq}{170mm}
\tau^w &=& - r_1 k f_L^w \Int_0^{\pi/2} d\Theta
 \sqrsico{1}{+}{+} \sin\left(\Theta+\frac{\pi}{2} - \tansico{1}{+}{+} \right) \nonumber\\
&& + r_1 k f_L^w \Int_0^{\pi/2} d\Theta
 \sqrsico{1}{+}{-} \sin\left(\Theta+\frac{\pi}{2} + \tansico{1}{+}{-} \right) \nonumber\\
&& - r_2 k f_T^w \Int_0^{\pi/2} d\Theta
 \sqrsico{2}{-}{+} \sin\left(\frac{\pi}{2}-\Theta + \tansico{2}{-}{+} \right) \nonumber\\
&& + r_2 k f_T^w \Int_0^{\pi/2} d\Theta
 \sqrsico{2}{-}{-} \sin\left(\frac{\pi}{2}-\Theta - \tansico{2}{-}{-} \right)
\end{wideeq}

\begin{flowtext}{2}

In these equations $f_L^d$ is the effective fraction of the leading semicircle
of the rock which experiences dry friction, and similarly for $f_T^d$, etc.,
$r_1$ and $r_2$ are the effective values of the radii of the inner and outer
portions of the semicircle, respectively; $v_x$ and $v_y$ are the components of
the velocity of the centre of mass of the rock; $\Theta$ in each quadrant is
measured from the $x$-axis toward the $y$-axis, as shown in \figref{shegelski};
and $\omega$ is the instantaneous angular speed of the rock.

\Figref{shegelski} is presented to assist in understanding and appreciating the
significance of the above equations. The figure shows the contributions
$\vec{v}_y$, $\vec{v}_x$ and $\vec{v}_{rot}$ (with $\vec{v}_{rot} =
\vec{r}\omega$) to the net velocity $\vec{u} \equiv \vec{u}(\Theta)$, relative
to the ice, of a portion of the contact annulus located at the angle $\Theta$;
note that $\vec{u} = \vec{v}_y + \vec{v}_x + \vec{v}_{rot}$, and recall that
the velocity of the centre of mass of the rock relative to the ice is $\vec{v}
= \vec{v}_y + \vec{v}_x$. Figure \figref{shegelski} also shows that the force
$\vec{F}$ exerted on that portion the direction opposite to $\vec{u}$. Similar
figures are readily structured for the other three quadrants of the rock. We
p???\ the equations above with each of the four quadrants exp???\ displayed so
that the reader may easily recognize the ???-bution from the four quadrants and
further appreciate asymmetries that arise, which are responsible for the curl
of the rock.

The $x$ and $y$ components of the velocity of the rock at any time $t$ are readily obtained
numerically from the ???\ equations along with $\vec{F} = M d\vec{v}/dt$, and
initial conditions. The location of the rock is then obtained by using
\begin{eqnarray}
x(t) &=& x_0 + \Int_0^t dt' v_x(t')
\end{eqnarray}
and
\begin{eqnarray}
y(t) &=& y_0 + \Int_0^t dt' v_y(t')
\end{eqnarray}

The angular speed $\omega$ is  given by
\begin{eqnarray}
\omega(t) &=& \omega_0 + \Int_0^t dt' \alpha(t')
\end{eqnarray}
where $\tau = MR^2\alpha/2$ determines the time development of the angular
acceleration of the rock.

In the first phase of the rock's motion, we put
$f_L^d = f_T^w = 1$,
$f_T^d = f_L^w = 0$ and
$r_1 = r_2 = r$.

In the middle phase of the rock's motion, we put
$f_L^d = f_T^w = f_T^d = f_L^w = 1/2$ and
$r_1 = r - \Delta r/2$, and
$r_2 = r + \Delta r/2$.

In the final phase, we have
$r_1 = r_2 = r$,
$f^d = 0$, and $f^w = 1$ for $0 \le \Theta' \le \Theta'_0$, and
$f^d = 1$, and $f^w = 0$ for $\Theta'_0 \le \Theta' \le 2\pi$, where $\Theta'$
is measured counterclockwise from the $-x'$ direction, and $\Theta'_0$ gives
the point on the rock where the velocity relative to the ice is radially
outwarded from the centre of the rock.

We do not have a first-priciples derivation of the function that should be used
to describe the transition between the first and middle, and middle and last
phases of motion. Instead we use a smooth function for the transition in
$f_L^d$ that has the following features:
$f_L^d \approx 1  $ for $v \ge 1.5 \meter\second^{-1}$;
$f_L^d \approx 1/2$ for $1.5 \meter\second^{-1} \ge v \ge 1.0 \meter\second^{-1}$; and
$f_L^d \approx 0  $ for $1.0 \meter\second^{-1} \ge v \ge 0 \meter\second^{-1}$.
The transition from one value to the next is smooth and occurs within a full
width of about $0.2 \meter\second^{-1}$. Small variations in the functional
form of $f_L^d$ give only very slight changes in the results presented in the
next section.

\end{flowtext}
}

\dots

\clearpage
\section{My own considerations}

\subsection{Introduction}

The rocks are not running straight in general: Besides slowing down they
\emph{curl} to one side. So the force $ \vec{F} $ caused by friction
rock/ice\index{friction!rock/ice} is not parallel to the speed $ \vec{v}
$. Neihter it is constant. The most general form is:
\begin{eqnarray}
    \vec{ F } &=& \vec{ F }( \vec{v}, \omega, \vec{x}, \alpha )
\end{eqnarray}

If ice and rock are homogenous and free of debri it simplifies a little
bit:
\begin{eqnarray}
    \vec{ F } &=& \vec{ F }( \vec{v}, \omega )
\end{eqnarray}

The components of $ \vec{ F } $ are:
\begin{equation}
\begin{array}{ll}
F_y \parallel \vec{v} & \textrm{force parallel running direction} \\
F_x \perp     \vec{v} & \textrm{force perpendicular to running direction}
\end{array}
\end{equation}

\subsection{Parameters to describe the friction}

Statements:
\begin{equation}\begin{array}{l@{\textrm{ for }}ll}
F_x \approx 0 & \omega \gg \omega_0 (\approx \frac{2.5\pi}{25s}) & \textrm{Straight running spinners} \\
F_x \approx 0 & \omega \approx 0    & \textrm{Straight if without handle} \\
F_x \max      & \omega = \omega_0   & \textrm{max.\ curl for a smooth handle} \\
F_y \ll       & \omega \gg \omega_0 & \textrm{few friction for spinners} \\
F_y \gg       & \omega \approx 0    & \textrm{much friction if without handle} \\
\end{array}\end{equation}

Practicable parameters to describe the ice might be:
\begin{itemize}
\item time hog to tee for a draw-to-tee,
\item curl for a draw-to-tee,
\item curl for a take-out of defined speed (e.g.\ 9s hog-to-tee),
\item do the rocks slow down quick once they started to or don't they.
\end{itemize}

\subsection{Base for $ \vec{ F } $}

The area of contact rock/ice is a fine ring on the rock's bottom. $ \vec{ F } $
is the sum of the forces affecting this ring. For a given spot $ (x,y)^T $
on this ring this force is in general:
\begin{equation}
    \vec{f} = \vec{f}(\vec{v}, \omega, x, y )
\end{equation}

Then $ \vec{ F } $ is:
\begin{equation}
    \vec{F} = \Int_R^{R+\Delta R} \Int_0^{2\pi}
                  \vec{f}(\vec{v}, \omega, x, y )
              \diff{\varphi} \diff{r}
\end{equation}
