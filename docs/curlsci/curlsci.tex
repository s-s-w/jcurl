\documentclass[a4paper,11pt,twoside]{report}
%\usepackage[emtex]{graphics}
\usepackage[T1]{fontenc}
%\usepackage[emtex]{graphics}
%  \usepackage[cp850]{inputenc} % DOS Umlaute
%  \usepackage[latin1]{inputenc} % UNIX Umlaute
\usepackage{bibeng}
% \usepackage{showidx}
\usepackage{makeidx}\makeindex
%\usepackage[final]{pictex}
\usepackage{pictex}
\usepackage{emlines3}
%\usepackage{multicol}

\usepackage{curling}
%\input{frame2cm}

%\includeonly{chtakeou,chcurl,chphyapp}
%\includeonly{chspini}
%\includeonly{chsploss}

\nonfrenchspacing
\renewcommand{\baselinestretch}{1.2}
\jot=1em          % zus"atzlicher Zeilenabstand innerhalb eqnarray
% \renewcommand{\familydefault}{\sfdefault}

\pagestyle{plain}
\pagestyle{headings}
\title{Curling scientific}
\author{Marcus Rohrmoser\\{\small marcus.rohrmoser\symbol{64}gmx.de}}
\date{March 27th 1998}

\begin{document}
\maketitle
\tableofcontents
\listoffigures
%\listoftables

% $Id$
\chapter*{Preface}\addcontentsline{toc}{chapter}{Preface}

During my days as an active Curler I spent quite a time thinking about
what's happening when a rock slides over the ice or hits another one.
And the longer I thought and the more people I asked the mystery did
nothing but grow.

Some situations I could predict by experience rather fair. Some more
skilled fellows could others. But some outcomes were completely
miraculous.

So I started to gather scientific material to cover the topic
from a theoretical base. Because once you got a model working properly
for known situations, you can start to examine unknown ones and compare
the model's predictions with reality. This way you get a fundamental
understanding of what's going on.

At the beginning I considered this to be a kind of brain-jogging, but now I
think the sport has developed so far yet that a team can't seriously
practice competitive curling without at least noticing some theoretical
knowledge e.g.\ about psychology but also physics.

Because to throw an \emph{impossible} matchwinner you first have to
recognise the option, then you need to dare it and surely, you need skill
\&\ luck to succeed.

\chapter*{Introduction}\addcontentsline{toc}{chapter}{Introduction}

This paper wants to examine the theoretical basics of running rocks and hits at
first and later apply these laws to real curling situations and get some useful
hints \&\ clues.
		%% preface, introduction

\chapter{Take-out}

Yet many brains boiled, there's poor secure knowledge still. Which directions
take the involved rocks? How much momentum (or energy) is lost? How is the spin
transferred? How much whiskey to swamp for throwing magic ones? And which one?

Maybe this paper can show some ideas and suggest solutions. For the scientific
stuff. About the whiskey you've got to try yourself. I also won't show you the
way to the next whiskey bar\dots

\vspace{1em}
Some things are going to be pretty much same with most of the following models.
They are defined here and will be just referred in the latter. The variables'
and constants' naming also is defined here.

\begin{table}[htb]
\begin{center}
\begin{tabular}{|ll|}
\hline
\Tk         & Time of contact                                              \\
\m{}        & Mass                                                         \\
\v{}        & Speed                                                        \\
\p{}        & Momentum $ \p{} = \m{} \cdot \v{} $                          \\
\F{}        & Force  $ \F{} = \m{} \cdot \ddt \v{} = \ddt \p{} $           \\
\J{}        & Moment of inertia                                            \\
\w{}        & Angular speed, spin                                          \\
\vL{}       & Angular momentum $ \vL{} = \J{} \cdot \vw{} $                \\
\vM{}       & Torque $ \vM{} = \J{} \cdot \ddt \vw{} = \ddt \vL{} = \vec{R} \times \F{} $ \\
\eY{}       & unit vector in (parallel) direction of hit (see \figref{coordsys}) \\
\eX{}       & unit vector perpendicular to direction of hit (see \figref{coordsys}) \\
\FY{}       & $ \FY{} = \F{} \cdot \eY{} $ and $ \F{} = \FX{} + \FY{} $    \\
\FX{}       & $ \FX{} = \F{} \cdot \eX{} $ and $ \F{} = \FX{} + \FY{} $    \\
\v{a}       & $\v{}$ of rock A                                             \\
\v{b}       & $\v{}$ of rock B                                             \\
\bef{\v{}}  & $\v{}$ before hit                                            \\
\aft{\v{}}  & $\v{}$ after hit                                             \\
$\DelTa{1,2} x $ & $ \aft{x} - \bef{x} $ (difference in time)              \\
$\DELTA{a,b}x$& $ x^\mathrm{b} - x^\mathrm{a} $ (difference in space)      \\
\hline
\end{tabular}
\caption[Naming of Variables and Constants]{%
         \tablab{names}%
         Naming of variables (italic) and constants (roman)}
\end{center}
\end{table}

\section*{The setup}

Rock A hits rock B (\figref{setup}), which equals A in all phys.\ properties.
Both can be in motion, but usually only A is.

According to the geometric setup we change to a local coordinate-system with
the $y$-axis pointing from A's center to B's. The $x$-axis is set perpendicular
to get a right-handed system. See \figref{coordsys}.

\begin{figure}[htb]
\setlength{\unitlength}{15mm} % the rock's radius
\begin{center}\large
\setcoordinatesystem units <\unitlength,\unitlength>
\mbox{\beginpicture
    % \setplotarea x from -1.3 to 2, y from -3 to 3.3
    %%% Stein A:
    \iffinalplot%
    \circulararc 360 degrees from 0.447 -0.894 center at  0    0
    \fi%
    \put{{\huge A}}                     [cc]  at 0.9 0.9
    %%% Stein B:
    \iffinalplot%
    \circulararc 360 degrees from 0.447 -0.894 center at  0.894 -1.788
    \fi%
    \put{{\huge B}}                           at  2      -1
    %% the speeds
    \put {\vector(0,-1){3.3}}       [Bl]      at  0      3.3
    \put {\bef{\v{a}}}              [l]       at  0.1    2.0
    \put {\vector(-2,-1){1.3}}      [Bl]      at  0      0
    \put {\aft{\v{a}}}              [r]       at -1.4   -0.6
    \put {\vector(1,-2){1.3}}       [Bl]      at  0.894 -1.788
    \put {\aft{\v{b}}}              [l]       at  2.0   -3.8

    \iffinalplot
    \circulararc -30 degrees from 0.5 -0.4 center at  0 0
    \fi
    \put {\vector(1,1){0.1}}        [Bl]      at 0.5 -0.4
    \put {\w{a}}                    [B]       at 0.45 -0.3
    \iffinalplot
    \circulararc 30 degrees from 0.394 -1.388 center at  0.894 -1.788
    \fi
    \put {\vector(1,1){0.1}}        [Bl]      at 0.394 -1.388
    \put {$-\w{b}$}                 [Bl]      at 0.394 -1.288
\endpicture}
\end{center}
\caption[The Setup]{The setup: Rock A hits B.
                        Both are \emph{identical}
                        in mass, size etc. After the hit i.g.\
                        both move.
                        \figlab{setup}}
\end{figure}

\begin{figure}[htb]
\setlength{\unitlength}{15mm} % the rock's radius
\begin{center}\large
\setcoordinatesystem units <\unitlength,\unitlength>
\mbox{\beginpicture
    \setplotarea x from -1.6 to 1.9, y from -2.8 to 1.5
    \iffinalplot%
    \circulararc 360 degrees from 0.447 -0.894 center at  0    0
    \fi%
    \iffinalplot%
    \circulararc 360 degrees from 0.447 -0.894 center at  0.894 -1.788
    \fi%
    \put{{\huge A}}                 [cc]   at  0.9    0.9
    \put{{\huge B}}                        at  2.0   -1.0
%    \put {\vector(0,-1){1.5}}       [Bl]   at  0      1.5
    \put {\vector(-2,-1){1.4}}      [Bl]   at  0.0    0.0
    \put {$\eX{}$}                  [r]    at -1.5   -0.6
    \put {\vector(1,-2){0.7}}       [Bl]   at  0.0    0.0
    \put {$\eY{}$}                  [Tl]   at  0.8   -1.4
\endpicture}
\end{center}
\caption[The Coordinate-System]{\figlab{coordsys}%
    The coordinate-system. Positive turn is conterclockwise.}
\end{figure}


\section{Without spin, without loss of energy}

This first model is the most primitive and simple one. But it's a proper
approximation in most cases. Also it's a good basis for later refinement.


\subsection{General thoughts}

Energy must be conserved\index{energy!convervation}:
%
\begin{eqnarray}
\bef{E} &=& \aft{E} \\
E &:=& \sum_\mathrm{all~rocks} E^i_{\mathrm{Kin}} +
                               E^i_{\mathrm{Rot}} +
                               E^i_{\mathrm{Fric}} \\
\mbox{here:} \quad 0 &=& E_{\mathrm{Rot}} = E_{\mathrm{Fric}}
\end{eqnarray}

\subsection{Parallel component $\pY{}$\label{para}}

If neglecting friction rock/ice\index{friction!rock/ice} and rock/rock only two
forces appear:
%
\begin{eqnarray}
\FY{a}(t) &=& - \FY{b}(t) \\
\Int_T\FY{a}(t)\dt &=& - \Int_T\FY{b}(t)\dt \\
\DelTa{1,2}\pY{a}  &=& - \DelTa{1,2}\pY{b}
\end{eqnarray}

With conservation of energy and assumption of equal masses we get:
\begin{eqnarray}
\aft{\vY{a}} &=& \bef{\vY{b}} \\
\aft{\vY{b}} &=& \bef{\vY{a}}
\end{eqnarray}

I.e.\ the speed-components along the hitting direction just exchange.

		%% generics concerning take-outs
% $Id$
\section{Without spin, with loss of energy\label{lossI}}

If we look at the rocks hitting each other, what can we see? I mean which
equations do occur? First of all only the components along the
hitting-direction interact at all, the rest remains unchanged. For these
parallel components we state:

\begin{eqnarray}
E_{Kin_1} &=& E_{Kin_2} + U \quad\mbox{with $U$ = lost energy} \\
\vec{F}_a + \vec{F}_b + \vec{\xi} &=& 0
\end{eqnarray}

If we assume the lost momentum to be small, e.g.\ because of a very short time
of contact ($\ll 1$) and a \emph{not} very huge friction rock/ice ($\not\gg
1$), we can assume $ \xi \approx 0 $ and solve this set of equations.

\iffalse
We get
\begin{eqnarray}
\aft{\pY{a}} &=& \bef{\pY{a}} + \bef{\pY{b}} - \aft{\pY{b}} \\
\aft{\pY{b}} &=&
    \frac{
	\sgn(\bef{\pY{a}}-\bef{\pY{b}})
    }{
	\m{a} + \m{b}
    }
    \sqrt{
    	\bef{\pY{a}}{}^2 \m{b}^2
	- 2 \bef{\pY{a}} \bef{\pY{b}} \m{a} \m{b}
	+ \m{a} ( \m{a} \bef{\pY{b}}{}^2 - 2 \m{a} \m{b} U - 2 \m{b}^2 U)
    } \nonumber\\
&+&
    \frac{
	\m{b} (\bef{\pY{a}} + \bef{\pY{b}})
    }{
	\m{a} + \m{b}
    }
\end{eqnarray}
\else
If we assume equal masses, things end up pretty simple --- quite as we like it!
We get
\begin{eqnarray}
\aft{\pY{a}} &=& \bef{\pY{a}} + \bef{\pY{b}} - \aft{\pY{b}} \\
\aft{\pY{b}} &=&
    \frac{
	\sgn(\bef{\pY{a}}-\bef{\pY{b}})
    }{ 2 }
    \sqrt{
	\left( \bef{\pY{a}} - \bef{\pY{b}} \right)^2 - 4 U
    }
    + \frac{
	\bef{\pY{a}} + \bef{\pY{b}}
    }{ 2 } \\
\aft{\pY{a}} &=&
    \frac{
	\sgn(\bef{\pY{b}}-\bef{\pY{a}})
    }{ 2 }
    \sqrt{
	\left( \bef{\pY{a}} - \bef{\pY{b}} \right)^2 - 4 U
    }
    + \frac{
	\bef{\pY{a}} + \bef{\pY{b}}
    }{ 2 }
\end{eqnarray}
\fi

\subsection{The loss' amount}

How big has the loss of energy $ U $ to be, to reduce the remaining path for a
given distance $ \Delta s $? It's got to be the friction's work along $ \Delta
s $. If assuming constant coulomb friction $ |F| = \mu m g $ we get
%
\begin{eqnarray}
|F| &=& \mu m g \\
W   &=& F \cdot s \\
\Delta E &=& W \\
\Longrightarrow
U &=& \Delta s \cdot \mu m g
\end{eqnarray}

\subsection{Resumee\label{lossproblem}}

This model workes fine for rather full hits. But as the hits become extreme
thin --- the speed-component in hitting direction very small --- we get weird
results, e.g.: The hitter changes it's direction, but the hitting rock remains
unmoved! This tells me to include all $ \pY{} $, $ \pX{} $ and $ \L{} $ into
the model, but how?

		%% model with energy-loss

\section{With spin, without loss (Try I)\label{spini}}

The parallel component works like stated in \ref{para}. Much more interesting
turns out to be the

\subsection{Perpendicular component $\pX{}$ and spin $\w{}$}

According to the friction rock/rock\index{friction!rock/rock} in the touching
spot a tangential force appears.

\subsubsection{Getting the equations}

In the touching spot appear forces affecting A and B, speed and spin.
The relations are (with \figref{kraftzerleg}):

\begin{eqnarray}
\FX{b} &=& -\FX{a} \quad\mbox{(actio=reactio)} \\
\M{a}  &=&  \Rad{a}\cdot\eY{} \times \F{a} =
          - \FX{a} \cdot \Rad{a} \\
\M{b}  &=& -\Rad{b}\cdot\eY{} \times \F{b} = 
	    \Rad{b}\cdot\eY{} \times \F{a} = 
          - \FX{a} \cdot \Rad{b}
\end{eqnarray}

Integration of the left sides:
\begin{eqnarray}
\FX{a}(t) &\rightarrow& \Int_\Tk \FX{a}(t)\dt = \DelTa{1,2}\pX{a} = \m{a}\DelTa{1,2}\vX{a} \\
\FX{b}(t) &\rightarrow& \Int_\Tk \FX{b}(t)\dt = \DelTa{1,2}\pX{b} = \m{b}\DelTa{1,2}\vX{b} \\
\M{a}(t)  &\rightarrow& \Int_\Tk \M{a}(t)\dt  = \DelTa{1,2}\L{a}  = \J{a}\DelTa{1,2}\w{a}  \\
\M{b}(t)  &\rightarrow& \Int_\Tk \M{b}(t)\dt  = \DelTa{1,2}\L{b}  = \J{a}\DelTa{1,2}\w{b}
\end{eqnarray}

Giving:
\begin{eqnarray}
\eqlab{AA} \DelTa{1,2}\vX{a} &=& + \frac{1}{\m{a}}           \Int_\Tk\FX{a}(t) \dt \\
\eqlab{BB} \DelTa{1,2}\vX{b} &=& - \frac{1}{\m{b}}           \Int_\Tk\FX{a}(t) \dt \\
\eqlab{CC} \DelTa{1,2}\w{a}  &=& - \frac{ \Rad{a} }{ \J{a} } \Int_\Tk\FX{a}(t) \dt \\
\eqlab{DD} \DelTa{1,2}\w{b}  &=& - \frac{ \Rad{b} }{ \J{b} } \Int_\Tk\FX{a}(t) \dt
\end{eqnarray}

So much about how the transferred forces interact. Now let's say something
about it's size. Because it's a friction, it exists only as long, as there
is a difference of the rocks' surface-speeds. A point on a rotating circle's
edge has the speed\footnote{speed (along the path) + rotation}:
%
\begin{eqnarray}
    \v{\eff} &=& \v{} + \vec{\omega} \times \vec{R}(\alpha)
\end{eqnarray}
%
Or in our case:
%
\begin{eqnarray}
  {\vX{a}}^\eff &=& \vX{a} + \w{a} \times \Rad{a}\eY{} =
                    \vX{a} - \Rad{a}\w{a} \eqlab{EEa} \\
  {\vX{b}}^\eff &=& \vX{b} + \w{b} \times -\Rad{b}\eY{} =
                    \vX{a} + \Rad{b}\w{b} \eqlab{FFa}
\end{eqnarray}
and
\begin{eqnarray}
\eqlab{GG}
  \DELTA{a,b}\vX{\eff} &=& {\vX{b}}^\eff - {\vX{a}}^\eff
\end{eqnarray}

The friction stops when\index{friction!rock/rock!max}:
\begin{eqnarray}
  \aft{\DELTA{a,b}\vX{\eff}} &=& 0 \\
  \aft{\DELTA{a,b}\vX{\eff}} - \bef{\DELTA{a,b}\vX{\eff}} &=&
  - \bef{\DELTA{a,b}\vX{\eff}} \\
\eqlab{HH}
 \DELTA{a,b}\DelTa{1,2}\vX{\eff} &=& \underbrace{%
                                      - \bef{\DELTA{a,b}\vX{\eff}}
                                      }_{ \displaystyle =: \Veff } \\
\eqlab{EE}
\Veff =
 \DELTA{a,b}\DelTa{1,2}\vX{\eff} &=&
      ( \DelTa{1,2}\vX{b} + \Rad{a}\DelTa{1,2}\w{b} ) -
        \DelTa{1,2}\vX{a} + \Rad{a}\DelTa{1,2}\w{a}
\end{eqnarray}

\eqref{AA} -- \eqref{DD} and \eqref{EE} build a solvable set of equations.

But the extremal $ \DELTA{a,b}\DelTa{1,2}\vX{\eff} $ isn't reached always.
Both rocks might already separate while still rubbing. The maximum is limited
by the pressure-force or $ \DelTa{1,2}\pY{a} $:\footnote{%
%
This linear relation between \DelTa{1,2}\pY{} and the left side is only valid for
pure Coulomb-friction\index{friction!Coulomb-}:
%
\begin{eqnarray}
\FX{}(t) &=& -\sgn( v ) \cdot \mu \cdot |\FY{}(t)| \\
\Int_\Tk \FX{}(t) \dt &=& -\sgn( v ) \cdot \mu \cdot \Int_\Tk |\FY{}(t)| \dt
\end{eqnarray}
%
no altering sign of $ \FY{}(t) $:
%
\begin{eqnarray}
\Int_\Tk \FX{}(t) \dt &=& -\sgn( v ) \cdot \mu \cdot \left| \Int_\Tk \FY{}(t) \dt \right| \\
\Int_\Tk \FX{}(t) \dt &=& -\sgn( v ) \cdot \mu \cdot | \DelTa{1,2}\pY{}|
\end{eqnarray}
}

\begin{eqnarray}
\eqlab{FF}
   \left|\Int_\Tk\FX{a}(t)\dt \right| &\le& \mu \cdot | \DelTa{1,2}\pY{a} | \\
\textrm{otherwise}\quad
   \Int_\Tk\FX{a}(t)\dt &=& - \sgn (\Veff) \cdot \mu \cdot | \DelTa{1,2}\pY{a} |
\end{eqnarray}

\subsubsection{Solution}

\eqref{EEa}, \eqref{FFa} \& \eqref{GG}:
\begin{eqnarray}
\Veff &=& - \bef{\vX{b}} - \Rad{b}\bef{\w{b}} +
          ( \bef{\vX{a}} - \Rad{a}\bef{\w{a}} )
\end{eqnarray}
%
\eqref{EE}:
\begin{eqnarray}
\Veff &=& - \DelTa{1,2}\vX{a} + \Rad{a}\DelTa{1,2}\w{a}
          + \DelTa{1,2}\vX{b} + \Rad{a}\DelTa{1,2}\w{b}
\end{eqnarray}
%
Substituting \eqref{AA} through \eqref{DD}:
%
\begin{eqnarray}
\Veff &=& - \Int_\Tk \FX{a}(t) \cdot
          \left(
            \frac{1}{\m{a}} + \frac{ \Rad{a}^2 }{ \J{a} } +
            \frac{1}{\m{b}} + \frac{ \Rad{b}^2 }{ \J{b} }
          \right)
          \dt \\
\eqlab{Xdef}
\underbrace{\Int_\Tk \FX{a}(t) \dt}_{\displaystyle =: X} &=& \frac{ -\Veff }{
            \frac{1}{\m{a}} + \frac{ \Rad{a}^2 }{ \J{a} } +
            \frac{1}{\m{b}} + \frac{ \Rad{b}^2 }{ \J{b} }
          } \\
  \textrm{If}\quad | X | &>& \mu | \DelTa{1,2}\pY{a} | \\
  \textrm{then}\quad  X &=& - \sgn( \Veff ) \cdot \mu \cdot | \DelTa{1,2}\pY{a} |\\
%
  \DelTa{1,2}\vX{a} &=& + \frac{1}{\m{a}}           \cdot X \\
  \DelTa{1,2}\vX{b} &=& - \frac{1}{\m{b}}           \cdot X \\
  \DelTa{1,2}\w{a}  &=& - \frac{ \Rad{a} }{ \J{a} } \cdot X \\
  \DelTa{1,2}\w{b}  &=& - \frac{ \Rad{b} }{ \J{b} } \cdot X
\end{eqnarray}

\subsection{Check it out}

O.k.\ so far, but what about the conservation of energy? This still should be
ensured, so let's test it.

The energy consists of two parts, movement and spin and the motion part can
be split into it's two components.

Beyond that, by using the form $z_2 = z_1 + \DelTa{1,2}z $ for all speeds and
taking their squares we can collect the mixed parts into a remainder $\xi$
%
\begin{eqnarray}
E_{Kin_1} + E_{Rot_1} & \stackrel{!}{=} & E_{Kin_2} + E_{Rot_2} \\
&=& E_{Kin_1} + E_{Rot_1} + \xi(x) 
\end{eqnarray}
%
This remainder $ \xi(x) $ must, for energy might be lost, but not gained,
be less or equal 0.

To test this, we discuss the function $ \xi(x) $.
%
\begin{eqnarray}
\eqlab{xa}
\xi(x) &=& 2 x (\vX{a} - \vX{b} - \Rad{}\w{a} - \Rad{}\w{b} ) +
           2 x^2 \frac{\J{}+\m{}\Rad{}^2}{\m{}\J{}} \\
\eqlab{xb}
\xi''(x) &=& 4 \frac{\J{}+\m{}\Rad{}^2}{\m{}\J{}} \\
\eqlab{xc}
\xi(0) &=& 0
\end{eqnarray}

The symbols $ \vX{a} $ through $ \w{b} $, strictly spoken, should be 
$ \bef{\vX{a}} \dots \bef{\w{b}} $. But for readability reasons in this 
section we skip the index.

Now let's check $ \xi(X) $ with $X$ from \eqref{Xdef}:
\begin{eqnarray}
X &=& \frac{ -\Veff }{
            \frac{1}{\m{}} + \frac{ \Rad{}^2 }{ \J{} } +
            \frac{1}{\m{}} + \frac{ \Rad{}^2 }{ \J{} }
          } \nonumber\\
&=& - \frac{ \vX{a} - \vX{b} - \Rad{}(\w{a} + \w{b}) }{ 2\frac{ \J{} + \m{}\Rad{}^2 }{ \m{}\J{} } } \\
\xi(X) 
&=& 2 X (\vX{a} - \vX{b} - \Rad{}(\w{a} + \w{b}) ) +
    2 X^2 \frac{\J{}+\m{}\Rad{}^2}{\m{}\J{}} \\
&=& - \frac{ 
    (\vX{a} - \vX{b} - \Rad{}(\w{a} + \w{b}) )^2 
}{
    \frac{ \J{}+\m{}\Rad{}^2 }{ \m{}\J{} }
} \nonumber\\
&+& \frac{ 1 }{ 2 }\cdot \frac{ 
    (\vX{a} - \vX{b} - \Rad{}(\w{a} + \w{b}) )^2 
}{
    \frac{ \J{}+\m{}\Rad{}^2 }{ \m{}\J{} } 
} \\
&=& - \frac{ 1 }{ 2 } \cdot \frac{
    (\vX{a} - \vX{b} - \Rad{}(\w{a} + \w{b}) )^2 
}{
    \frac{ \J{}+\m{}\Rad{}^2 }{ \m{}\J{} }
} < 0 \\
\Longrightarrow \xi(2\cdot X) &=& 0 \\
\Longrightarrow \min_{\forall x}\xi(x) &=& \xi(X)
\end{eqnarray}

That means, as long as our values for $x$ lie between 0 and $X$, we've got $\xi
< 0$ and we're on the sunny side of life: We don't gain energy. Worst case we
loose. I don't exactly understand where in the above equations this energy gets
lost, but for the first we seem to have a plausible, more or less simple and
handleable model for what happens with take-outs. The physical effect causing
this loss surely is the friction rock/rock and $ \xi $ equals the energy lost
here.


\subsection{Resumee}

The model seems to work rather fair and tells that both rocks don't separate
under $ 90^0 $ in general. Spin\index{spin!influence on momentum}
has influence on direction of movement and vice versa.

But still there's a friction rock/ice that isn't included in this model, so 
it cannot be perfect in all situations.

\textbf{Improved models are to be developed!}

		%% model with spin, try I

\section{With spin and loss (Try I)\label{lossII}}

The approach \ref{lossI} doesn't give proper results, so we try a new method.
We assume the rock B not to experience acceleration until the initial friction
$ F_H $ is overcome. This causes both, a loss of energy as well as as loss of
momentum and spin. Besides we get a natural splitting of the loss of energy to
the components $ \vY{}, \vX{} $ and $ \w{} $. 

For this purpose we need to know the forces \emph{during} the time of contact $
T $. To achieve this we need to make some assumptions about the rocks'
elasticity mechanism. Here we assume the rocks to behave like springs and
fulfilling Hook's law of elasticity. Also we assume the friction rock/rock to
be constant $ |\FX{}| = \mu \cdot |\FY{} | $. This way we get to know the
forces and everything's just fine.

In this section all letters (if not stated differently) mean rock A before the
hit. E.g. $ x $ really is $ \bef{x^\mathrm{a}} $.

\subsection{Contact behaviour}

Hook says:
{%\samepage
\begin{eqnarray}
F &=& H \cdot x \\
E_\mathrm{Pot} &=& \frac{1}{2} H \cdot x^2
\end{eqnarray}}

{%\samepage
\begin{eqnarray}
H \cdot x &=& m \cdot a = m \frac{ \diff{}^2 x }{ \dt } \\
\Longrightarrow\quad
x(t) &=& A \cdot \sin \bar\omega t \\
\mbox{with}\quad
    \bar\omega &=& \sqrt{ \frac{H}{m} } = \frac{ \pi }{ T }  = \const !\\
\mbox{and}\quad
    A &=& \sqrt{ \frac{m}{H} \vY{}{}^2 } = \frac{ -\vY{} }{ \bar\omega } =
    \frac{ -\vY{} T }{ \pi } 
\end{eqnarray}}
%
{%\samepage
\begin{eqnarray}
 x(t) &=& - \frac{ \vY{} }{ \bar\omega } \cdot \sin \bar\omega t \\
\ddt x(t) &=& -\vY{} \cdot \cos \bar\omega t \\
\frac{ \diff{}^2 }{ \dt^2 } x(t) &=& \vY{} \bar\omega \sin \bar\omega t 
\end{eqnarray}}
%
{%\samepage
\begin{eqnarray}
\DelTa{1,2} \vY{} &=& -\vY{} \left[ \cos\bar\omega t' \right]_0^t = 
   -\vY{} \left( \cos\bar\omega t - 1 \right) 
\end{eqnarray}}


O.k.\ so far. But what happens during such a hit? We split the whole contact
$ 0 \le t \le T $ into three periods:
\begin{enumerate}
\item Overcome the initial friction $ t < t_1 $
\item Level the surface speeds $ t < t_0 $
\item Exchange the rest of the parallel momentum $ t \le T $
\end{enumerate}

\begin{figure}[htb]
{\footnotesize \hfill\input{contacta.emt}\hfill\input{contactb.emt}\hfill }
\caption[$ \FY{}(t) $ \emph{during} the hit]{$ \FY{}(t) $. The parallel forces
    during the hit. As you see, B ain't accelerated until $ t \ge t_1
    $.\figlab{forceABhit}} 
\end{figure}

\subsection{Find $ t_0 $}

$ t_0 $ is the time, when A's effective surface speed becomes 0 (=B's).
{%\samepage
\begin{eqnarray}
    \aft{\v{\eff}} & \stackrel{!}{=} & 0 \\
    \aft{\vX{}} - \Rad{}\aft{\w{}} &=& 0 \\
    \vX{} + \DelTa{1,2}\vX{} - \Rad{}( \w{} + \DelTa{1,2}\w{} ) &=& 0 \\
\end{eqnarray}}
with \eqref{AA} \& \eqref{CC}:
{%\samepage
\begin{eqnarray}
    \DelTa{1,2}\w{} &=& \DelTa{1,2}\vX{} \frac{ - \m{}\Rad{} }{ \J{} } \\
    \DelTa{1,2}\vX{} &=& \frac{ - \v{\eff} }{ \frac{ \J{} + \m{}\Rad{}^2 }{\J{}}} 
\end{eqnarray}}
%
{%\samepage
\begin{eqnarray}
\FX{} &=& -\sgn(\v{\eff}) \cdot \mu \cdot | \FY{} | \\
\Longrightarrow\quad
\DelTa{1,2}\vX{} &=& -\sgn(\v{\eff}) \cdot \mu \cdot \left| -\vY{} \cdot(\cos\bar\omega t_0 - 1) \right| 
\end{eqnarray}}
%
{%\samepage
\begin{eqnarray}
\Longrightarrow\quad
\frac{ - \v{\eff} \frac{ 1 }{ \frac{ \J{} + \m{}\Rad{}^2 }{ \J{} } } }{
- \sgn(\v{\eff}) \mu | \vY{} | } &=&
| \underbrace{\cos\bar\omega t_0 - 1}_{ \le 0 \forall t_0 } | \\
- \left| \frac{ \frac{ \v{\eff} }{ \frac{ \J{} + \m{}\Rad{}^2 }{ \J{} } } }{
\vY{} \mu } \right| + 1 &=& \cos\bar\omega t_0
\end{eqnarray}}

\subsection{Find $ t_1 $}

$ t_1 $ is the time, when the Hook-force equals the friction.
{%\samepage
\begin{eqnarray}
\FY{}{}^2 + \FX{}{}^2 &=& F_H^2 \\
\FY{}{}^2 + \mu^2\FY{}{}^2 &=& F_H^2 \\
\FY{} &=& \frac{ F_H }{ \sqrt{ 1+\mu^2 } } \\
\m{} \vY{} \bar\omega \sin\bar\omega t_1 &=& \frac{ F_H }{ \sqrt{1+\mu^2 } } \\
\sin\bar\omega t_1 &=& \frac{ F_H }{ \m{} \vY{} \bar\omega \sqrt{1+\mu^2} }
\end{eqnarray}}

\subsection*{Result}

The lost momentum is:
{%\samepage
\begin{eqnarray}
\Delta\v{} &=& -\vY{} \left( \sqrt{1-\sin^2\bar\omega t_1} - 1 \right)
{-\sgn(\v{\eff})\cdot\mu \choose -1}
\end{eqnarray}}

If $ \v{\eff} $ becomes zero \emph{before} the friction ends we need to find 
the friction parallel to $ \eY{} $ only, but include the lost momentum
in $ \eX{} $ direction as well.

{%\samepage
\begin{eqnarray}
\mbox{if}\quad  t_0 < t_1 & \Longleftrightarrow & \sin\bar\omega t_0 < \sin\bar\omega t_1 \\
\sin\bar\omega t_1 &=& \frac{ F_H }{ \m{} \vY{} \bar\omega \cdot 1 } 
\end{eqnarray}}
%
Lost momentum:
{%\samepage
\begin{eqnarray}
\perp &:& - \sgn(\v{\eff}) \mu \cdot \left| -\vY{} \cdot (\cos\bar\omega t_0 - 1) \right| \\
\parallel &:& \vY{} \cdot (\sqrt{1-\sin^2\bar\omega t_1} - 1)
\end{eqnarray}}

After applying the above loss of momentum to the \emph{hitting} rock and a 
loss of spin of
{%\samepage
\begin{eqnarray}
\DelTa{1,2}\w{} &=& - \DelTa{1,2}\vX{} \frac{ \m{}\Rad{} }{ \J{} }
\end{eqnarray}}
we can continue with a normal hit, e.g.\ like described in \ref{spini}.

\subsection{Loss of energy \& $ F_H $}

The energy stored in a spring is
{%\samepage
\begin{eqnarray}
E &=& \frac{1}{2} H x^2 \\
&=& \frac{1}{2} \bar\omega^2 \cdot \m{} \cdot x^2 \\
\mbox{with}\quad x &=& - \vY{}/\bar\omega \cdot \sin\bar\omega t_H \\
\mbox{and}\quad F_H &=& \m{}\vY{}\bar\omega \cdot \sin\bar\omega t_H \\
E &=& \frac{1}{2} \bar\omega^2 \cdot \m{} \cdot (- \vY{}/\bar\omega\cdot\sin\bar\omega t_H )^2 \\
E &=& \frac{1}{2} \frac{F_H^2}{\m{}\bar\omega^2} \\
\Longrightarrow\quad
F_H &=& \bar\omega\sqrt{2\m{}E} = \const !
\end{eqnarray}}


That tells us for a given loss of energy we'll get a constant initial friction.
(Which, indeed, is a very nice result.)

\subsection{Resumee}

This model depends on Coulomb-friction rock/rock and a constant friction
rock/ice for non-moving rocks. This friction can be very big, about $ 10^4 $ N!

The results presented by this model seem quite sensible. The only problem
remaining is what happens if hitting a freeze? Do we have to overcome the
friction two times? If yes this model works fine.

		%% model with spin and loss, try I

\section{With spin, with loss (Try II)}

After the recent try (I) which took benefit from superposition but fails in the
loss of energy's inclusion (see \ref{lossproblem}), we try an unsplit approach.
Returning to the basics, we have some laws at hand:

\begin{eqnarray}
\F{a} + \F{b} &=& 0 \\
\Rad{a}\cdot\eY{} \times \F{a} &=& \vM{a} \\
\vM{b} = -\Rad{b}\cdot\eY{} \times \F{b} = -\Rad{b}\cdot\eY{} \times -\F{a} &=& \vM{a}
\end{eqnarray}
\begin{eqnarray}
\eqlab{EconstA}\DelTa{1,2}( E_{Kin} + E_{Rot} ) &=& 0 \\
\DelTa{1,2}\p{a} + \DelTa{1,2}\p{b} &=& 0 \\
\DelTa{1,2}\vL{a} - \DelTa{1,2}\vL{b} &=& 0 \\
\DelTa{1,2}\vL{a} - \Rad{a}\cdot\eY{} \times \DelTa{1,2}\p{a} &=& 0 \\
\eX{} \cdot \left(
    \frac{\p{a}}{\m{a}} + \frac{\vL{a}}{\J{a}} \times  \Rad{a}\cdot\eY{} -
    \left(
    \frac{\p{b}}{\m{b}} + \frac{\vL{b}}{\J{b}} \times -\Rad{b}\cdot\eY{} 
    \right)
\right) &=& 0 \\
\mu|\DelTa{1,2}\p{a} \cdot\eY{} | &\ge& | \DelTa{1,2}\p{a} \cdot\eX{} |
\end{eqnarray}

\eqref{EconstA} better is written as
{%\samepage
\begin{eqnarray}
\bef{E} = \aft{E} + \xi + U
\end{eqnarray}}

with $ \xi(x) $ beeing the loss from friction rock/rock and U the one from
rock/ice. $ E $ surely includes both kinetic and rotation energy.

		%% model with spin and loss, try II

\section{Improvements}

The loss of energy\index{energy!loss} on hits because of friction rock/ice
to me seems to be very important.

A loss of energy could be introduced by a reduction of the remaining
path to run $(\Delta s)$ as a reduction of the transferred speed (momentum)
$ v_\parallel $ ($ a $ is friction-caused acceleration):
%
\begin{eqnarray}
v & \approx & \sqrt{ v_0^2 - 2 \cdot \Delta s \cdot a }
\end{eqnarray}
%
This relation isn't the result of proper theoretical knowledge or derived,
but bases heuristically on more or less vague experience.

Evtl.\ also of big influence might be to examine the rock's exact behaviour
during contact. Esp.\ the displacement of the rocks \emph{during} the
contact I consider to be nonneglegible.

		%% takeout/improvements
% $Id$

\chapter{Very nice, but show me\dots}

For the above formulae beeing quite complex I made a C++ library with all you
need to simulate hits and running rocks. It compiles with Borland C++ 2.0, gcc
2.7.2 (sun) and djgpp 2.8.1. I also added means for quick and compact storing
of trajectories. Very simple graphics output classes also exist. I called the
package 'mrcurl' and would like to publish it, but don't know which way.

If you're interested in this lib (or the two primitive programs 'curlcomp' to
compute and 'curlview' to view) mail me. Also mail me if you'd like to help me
porting the stuff to Java.
			%% takeout/improvements
% $Id$
\chapter{Rock trajectory}
\section{Applying the 'Denny' model}

Mark Denny published a model in \cite{denny:98}, see \ref{denny}. To apply it
for simulation we need to
\begin{itemize}
\item compute the ice properties from the draw-to-tee time and curl.
\item transform from $t_0 = 0$ to $t_0 > 0$
\item transform the equations from rock-coordinates to world-coordinates
\item compute the initial speed of a rock from the hog-to-hog time and $y_0$
\end{itemize}

\subsection{Basic equations}
After substituting \eqref{denTau} the baseic equations are:
\begin{eqnarray}
\eqlab{dennyX}
x(t) &=& -\sgn(\omega_0){{b\,g\,\mu\,t^3\,\left(3\,g\,\mu\,t-4\,v_{0}\right)}\over{48\,
 \varepsilon\,R}} \\
x'(t) &=& -\sgn(\omega_0){{b\,g\,\mu\,t^2\,\left(g\,\mu\,t-v_{0}\right)}\over{4\,\varepsilon
 \,R}} \\
x''(t) &=& -\sgn(\omega_0){{b\,g\,\mu\,t\,\left(3\,g\,\mu\,t-2\,v_{0}\right)}\over{4\,
 \varepsilon\,R}} \\
\eqlab{dennyY}
y(t) &=& -{{t\,\left(g\,\mu\,t-2\,v_{0}\right)}\over{2}} \\
y'(t) &=& -\left(g\,\mu\,t-v_{0}\right) \\
y''(t) &=& -g\,\mu \\
\alpha(t) &=& {{\omega_{0}\,\varepsilon\,\left(g\,\mu\,t\,\left(-{{g\,\mu\,t-v_{0
 }}\over{v_{0}}}\right)^{{{1}\over{\varepsilon}}}-v_{0}\,\left(-{{g\,
 \mu\,t-v_{0}}\over{v_{0}}}\right)^{{{1}\over{\varepsilon}}}+v_{0}
 \right)}\over{\left(\varepsilon+1\right)\,g\,\mu}} \\
\alpha'(t) &=& \omega_{0}\,\left(-{{g\,\mu\,t-v_{0}}\over{v_{0}}}\right)^{{{1
 }\over{\varepsilon}}} \\
\alpha''(t) &=& {{\omega_{0}\,g\,\mu\,\left(-{{g\,\mu\,t-v_{0}}\over{v_{0}}}\right)
 ^{{{1}\over{\varepsilon}}}}\over{\varepsilon\,\left(g\,\mu\,t-v_{0}
 \right)}}
\end{eqnarray}

\subsection{The ice-properties $ \mu $ and $ b $}
To calculate $\mu$ and $b$ from the time $T$ and curl $B$ of a draw-to-tee
\begin{eqnarray}
\eqlab{dennyXTee}
\mbox{and} \quad x(T) &=& B \quad \mbox{(curl)} \\
\mbox{and} \quad x'(T) &=& 0 \\
\eqlab{dennyYTee}
\mbox{with} \quad y(T) &=& D \quad \mbox{(distance hog to tee)} \\
\mbox{and} \quad y'(T) &=& 0 \\
\Rightarrow \quad v_0 &=& g \mu t \\
\eqlab{dennyMu}
\mu &=& \frac{2 D}{g T^2} \\
\eqlab{dennyCurl}
b &=& -{{48\,B\,\varepsilon\,R}\over{g^2\,T^6}}
\end{eqnarray}

\subsection{Some initial speeds}

\textit{How hard do we have to throw a rock, that will take 12 seconds hog to
hog?}

To calculate $v_0$ at the far hog we don't use the time hog-to-tee for not
every rock reaches the tee-line. Here the time hog-to-hog ($T_H$) is better.
\begin{eqnarray}
\mbox{dist.~hog--hog:} \quad H &=& v_0 \left( T_H - \frac{T_H^2}{2\tau}\right) \\
&=& v_0 \left(T_H - \frac{T_H^2}{2 \frac{v_0}{\mu g} }\right) \\
\frac{H}{v_0} &=& T_H - \frac{T_H^2}{v_0} \frac{\mu g}{2} \\
T_H &=& \frac{1}{v_0} \left(H + T_H^2 \frac{\mu g}{2}\right) \\
\Longrightarrow\quad
v_0 &=& \frac{H}{T_H} + \frac{T_H\mu g}{2}
\end{eqnarray}

If we don't want $v_0$ at the far hog but at any given distance, things get a
little bit more complicated.

Here we use \eqref{denY} at three times

\medskip
\begin{tabular}{l@{ : }l}
$t_0          $ & our starting point \\
$t_1 \equiv 0 $ & the rock crosses the far hog \\
$T_H $          & time hog-to-hog
\end{tabular}
\medskip

\noindent
with two distances

\medskip
\begin{tabular}{l@{ : }l}
$y_0$ & distance start-to-far-hog \\
$H  $ & distance hog-to-hog
\end{tabular}
\medskip

\begin{eqnarray}
y(t_0 - t_0) &:& 0 = 0 \\
\eqlab{denVA}
y(t_1 - t_0) &:& v_0 \left(-t_0 - \frac{t_0^2 \mu g}{2 v_0}\right) = -y_0 \\
\eqlab{denVB}
y(T_H - t_0) &:& v_0 \left(T_H-t_0 - \frac{(T_H-t_0)^2\mu g}{2v_0}\right) = H-y_0 
\end{eqnarray}

\eqref{denVA} + \eqref{denVB}:
\begin{eqnarray}
H &=& v_0 T_H - T_H\frac{\mu g}{2} + T_H t_0 \mu g \\
\Longrightarrow\quad
t_0 &=& \frac{H - v_0 T_H + T_H^2\frac{\mu g}{2}}{ T_H \mu g }
\end{eqnarray}

Substituting this into \eqref{denVA} and solving for $v_0$ gives
\begin{eqnarray}
\Longrightarrow v_0 =  \frac{1}{2}\cdot 
    \sqrt{
	\left(\mu g T_H + 2\frac{H}{T_H}\right)^2 +
	4 \mu g H - 
	8 \mu g y_0
    }
\end{eqnarray}

If you prefer using $Y_0$ measured from the tee, use
\begin{eqnarray}
\Longrightarrow v_0 =  \frac{1}{2}\cdot 
    \sqrt{
	\left(\mu g T_H + 2\frac{H}{T_H}\right)^2 + 
	4 \mu g H - 
	8 \mu g (\mbox{far-hog-to-tee} - Y_0)
    }
\end{eqnarray}

\subsection{Coordinate transformation}

Because of Denny assumes the rock to start at
$(0,0)^T$ with $v_0$ pointing along the $\hat{y}$-axis, we need a
rotation and shift to get the general equations.

\begin{eqnarray}
{x \choose y} &:=& \frac{ \vec{v}_\mathrm{real} }{ | \vec{v}_\mathrm{real} | }
\end{eqnarray}

\begin{eqnarray}
\mbox{The required transformation is:}\quad
\left(\begin{array}{cc}
 y &  x \\
 x & -y
\end{array}\right)
\end{eqnarray}

Applying this trafo to e.g.\ ${a \choose b}$ results in
\begin{eqnarray}
\left(\begin{array}{cc}
 y &  x \\
 x & -y
\end{array}\right)
\cdot {a \choose b} &=&
{ay+bx \choose ax-by}
\end{eqnarray}
%
where $a$ and $b$ are polynomes of max.\ fourth degree:
%
\begin{eqnarray}
a &=:& At^4 + Bt^3 + Ct^2 + Dt + E \\
b &=:& \alpha t^4 + \beta t^3 + \gamma t^2 + \delta t + \phi
\end{eqnarray}

This leads to a x-component of
\begin{eqnarray}
&& y ( At^4 + Bt^3 + Ct^2 + Dt + E ) + \nonumber \\
&& x ( \alpha t^4 + \beta t^3 + \gamma t^2 + \delta t + \phi ) \\
&=& (Ay + \alpha x)t^4 + \ldots + (Ey + \phi x)
\end{eqnarray}

and a y-component of
\begin{eqnarray}
&&  x ( At^4 + Bt^3 + Ct^2 + Dt + E ) + \nonumber \\
&& -y ( \alpha t^4 + \beta t^3 + \gamma t^2 + \delta t + \phi ) \\
&=& (Ax - \alpha y)t^4 + \ldots + (Ex - \phi y)
\end{eqnarray}

Now just the shift is missing.

\clearpage
		%% thoughts about curl

\begin{appendix}
\chapter{Some physical/math. Basics}
\section{Rotation equations using cross-products}

%\cite[S.~20-5]{feynmanI:89} says (the names are changed to fit our system):

\begin{eqnarray}
\mbox{velocity:}\quad       \vec{v} &=& \vec{\omega} \times \vec{R} \\
\mbox{torque:}\quad         \vec{M} &=& \vec{R} \times \vec{F} \\
\mbox{angular mom.:}\quad   \vec{L} &=& \vec{R} \times \vec{p} = J\vec{\omega} \\
\mbox{energy:}\quad         E_{Rot} &=& \frac{1}{2} \vec{L} \cdot \vec{\omega} = \frac{|\vec{L}|^2}{2J}
\end{eqnarray}


\section{Force-splitting}

See \figref{kraftzerleg}.

\begin{figure}[htb]
\setlength{\unitlength}{10mm}
\begin{center}\large
\setcoordinatesystem units <\unitlength,\unitlength>
\mbox{\beginpicture
\small
\setplotarea x from -2.5 to 2.5, y from -2.5 to 2.5

\iffinalplot
\circulararc 360 degrees from -2 1 center at  0 0
\fi

\put {$\times$} 		     at 0     0   % Sp Marker

\put {\vector(0,-1){2}} 	[Bl] at 2.24  0   % Force at the edge
\put {\vector(0,1){2}}		[Bl] at 0     0   % Froce in Sp
\put {\vector(0,-1){2}} 	[Bl] at 0     0   % Opposition-froce
\put {\vector(1,0){2.24}}	[Bl] at 0     0   % Radius
\put {$\vec{F}_1$}		[Cl] at 2.24 -1   % Force at the edge
\put {$\vec{F}_3=\vec{F}_1$}	[Cr] at 0    -1   % Force in Sp
\put {$\vec{F}_2=-\vec{F}_1$}	[Cr] at 0     1   % Opposition force
\put {$\vec{ R }$}		[Bc] at 1.1   0   % Radius
\endpicture}
\caption[Force-splitting]{\figlab{kraftzerleg}%
Force-splitting: The force at the edge $ \vec{F}_1 $ can, after adding
$ \vec{F}_{2,3} $ be split into a central force $ \vec{F}_3 $ and
a torque-causing force-pair $ \vec{F}_{1,2} $.
The result is a force $ \vec{F}_2 $ in the center of mass and a
torque $ \vec{F}_1 \times \vec{R} $. See
\cite[P.~142f]{greinerII:85}, \cite[P.~34fff]{gross:95}. }
\end{center}
\end{figure}

%\clearpage
\section{Time-distance of two spheres on straight paths}

\begin{figure}[ht]
\begin{center}
\input{distance.em}
\end{center}
\caption{When do 2 objects hit? The setup.}
\end{figure}

\iffalse

Definition:
\begin{eqnarray}
\vec{v} &:=& \vec{v}_b - \vec{v}_a \\
\vec{x} &:=& \vec{x}_b - \vec{x}_a
\end{eqnarray}

Two objects hit, when the distance between their surfaces becomes 0.

Distance:
\begin{eqnarray}
d &=& |\vec{x}| - \underbrace{R}_{\makebox[0mm]{\tiny sum of radii}}
\end{eqnarray}

Speed of approach:
\begin{eqnarray}
\vec{v} \parallel \vec{x} &=& - \frac{ \vec{v}\vec{x} }{ |\vec{x}| }
\end{eqnarray}

Time until hit:
\begin{eqnarray}
t &=& \frac{ d }{ \vec{v} \parallel \vec{x} } \\
  &=& - |\vec{x}| \frac{ |\vec{x}| - R }{ \vec{v}\vec{x} }
\end{eqnarray}

If $\vec{v}\vec{x} = 0$ the rocks pass each other, for
$\vec{v}\vec{x} > 0$ they separate.

\else

Having one object's center at $\vec{a}$ with speed $\vec{v}_a$, the other at
$\vec{b}$ with speed $\vec{v}_b$, we introduce
%
\begin{eqnarray}
\vec{x} := \vec{b}   - \vec{a}   \\
\vec{v} := \vec{v}_b - \vec{a}_a 
\end{eqnarray}

Now we need the time until the centers' distance equals the two radii:

\begin{eqnarray}
|\vec{x} + t \cdot \vec{v} | &=& (r+r) \\
(\vec{x} + t \cdot \vec{v} ) \cdot (\vec{x} + t \cdot \vec{v} ) &=& (r+r)^2 \\
\vec{x}\vec{x} + 2 t \vec{v}\vec{x} + t^2 \vec{v}\vec{v} - (r+r)^2 &=& 0 
\end{eqnarray}

\begin{eqnarray}
t_{1,2} &=& \frac{ 
       	-\vec{v}\vec{x} \pm \sqrt{ \left(\vec{v}\vec{x})^2 - 
	|\vec{v}|^2\cdot(|\vec{x}|^2 - (r+r)^2\right) }
    }{ 
	|\vec{v}|^2 
    }
\end{eqnarray}

In our case the solution is the smaller (earlier) result:

\begin{eqnarray}
t &=& \frac{ 
       	-\vec{v}\vec{x} - \sqrt{ \left(\vec{v}\vec{x})^2 - 
	|\vec{v}|^2\cdot(|\vec{x}|^2 - (r+r)^2\right) }
    }{ 
	|\vec{v}|^2 
    }
\end{eqnarray}

If $\vec{v}\vec{x} = 0$ the rocks pass each other (or don't move at all), for
$\vec{v}\vec{x} > 0$ they separate.

\fi

		%% phys. & math. explanations
\end{appendix}

%\addcontentsline{toc}{chapter}{Index}
%\printindex

 \nocite{brach:92}
 \nocite{feynmanI:89}
 \nocite{gammert}
 \nocite{greinerII:85}
 \nocite{gross:95}
 \nocite{hertz:81}
 \nocite{hills:93}
 \nocite{hughesI:76}
 \nocite{hughesII:76}
 \nocite{lehmann:85}
 \nocite{meybergI:93}
 \nocite{van:89}
 %
 \nocite{sutor:88}
 \nocite{denny:98}
 \nocite{shegelski:96}
 \nocite{voyenli:85}
 \nocite{voyenli:86}
 \nocite{daniels:86}
 \nocite{johnston:81}
 %
 \addcontentsline{toc}{chapter}{\bibname}%
 \bibliography{sci}%
 \bibliographystyle{myalpha} % {plain}

\end{document}
